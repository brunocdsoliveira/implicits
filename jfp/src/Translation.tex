\section{Type-Directed Translation to System F}
\label{sec:trans}

In this section we explain the dynamic semantics of $\name$ in terms
of System F's dynamic semantics, by means of a type-directed translation. 
This translation turns implicit contexts into explicit parameters and
statically resolves all queries, much like Wadler and Blott's dictionary
passing translation for type classes~\cite{adhoc}. 
The advantage of this approach is that we simultaneously provide a meaning to
well-typed $\name$ programs and an effective implementation that resolves
all queries statically.

The translation follows the type system presented in Section~\ref{sec:ourlang}.
The additional machinery that is necessary (on top of the type system)
corresponds to the grayed parts of Figures~\ref{fig:type}, \ref{fig:resolution1} and \ref{fig:resolution2}. 

%-------------------------------------------------------------------------------
\subsection{Type-Directed Translation}
Figure~\ref{fig:type} presents the translation rules that convert $\name$
expressions into ones of System~F. 
The gray parts of the figure extend the type system with the necessary
information for the translation.

The syntax of System F is as follows: 
{\small
  \[ \begin{array}{llrl}
    \text{Types} & T & ::= & \alpha \mid T \arrow T 
    \mid \forall \alpha. T \\ 
    \text{Expressions} & E & ::=  & x \mid \lambda (x:T) . E \mid E\;E
    \mid \Lambda \alpha . E \mid E\;T 
  \end{array} \]}

The gray extension to the syntax of type environments annotates every
implicit rule type with explicit System F evidence in the form of a 
term variable $x$.

%- - - - - - - - - - - - - - - - - - - - - - - - - - - - - - - - - - - - - - - - 
\paragraph{Translation of Types}

The function $\ttrans$ takes 
$\name$ types $\rulet$ to System F types T: 
\begin{equation*}
\begin{array}{rcl@{\hspace{1cm}}rcl}
\ttrans[\alpha] & = & \alpha &
\ttrans[\forall \alpha. \rulet] & = & \forall \alpha. \ttrans[\rulet] \\
\ttrans[\rulet_1 \arrow \rulet_2] & = & \ttrans[\rulet_1] \arrow \ttrans[\rulet_2] &
\ttrans[\rulet_1 \iarrow \rulet_2] & = & \ttrans[\rulet_1] \arrow \ttrans[\rulet_2] %\\
\end{array}
\end{equation*}
Its reveals that implicit $\name$ arrows are translated to explicit System F function arrows.

%- - - - - - - - - - - - - - - - - - - - - - - - - - - - - - - - - - - - - - - - 
\paragraph{Translation of Terms}

The type-directed translation judgment, which extends the typing judgment, is 
\begin{center}
  $\wte{\tenv}{e}{\rulet}{E}$
\end{center}
This judgment states that the translation of $\name$ expression $e$ with
type $\rulet$ is System~F expression $E$, with respect to type environment
$\tenv$.

Variables, lambda abstractions and applications
are translated straightforwardly. Perhaps the only noteworthy 
rule is \rref{Ty-IAbs}. This rule associates the type $\rulet_1$ with 
the fresh variable $x$ in the type environment. 
This creates the necessary evidence that can be used by resolutions 
in the body of the rule abstraction to construct System F terms of type $\ttrans[\rulet_1]$. 

%- - - - - - - - - - - - - - - - - - - - - - - - - - - - - - - - - - - - - - - - 
\paragraph{Resolution}
The more interesting part of the translation happens when resolving queries.
Queries are translated by rule $\rref{Ty-Query}$ using the auxiliary resolution
judgment $\ivturns$:
\begin{equation*}
\dres{\tenv}{\rulet}{E}
\end{equation*}
which is shown, in deterministic form, in Figure~\ref{fig:resolution2}.  The
translation builds a System F term as evidence for the resolution.  

The mechanism that builds evidence dualizes the process of peeling off
abstractions and universal quantifiers: Rule \rref{R-IAbs}~wraps a lambda
binder with a fresh variable $x$ around a System F expression $E$, which is
generated from the resolution for the head of the rule ($\rulet_2$). Similarly,
rule \rref{R-TAbs}~wraps a type lambda binder around the System F expression
resulting from the resolution of $\rulet$.

For simple types $\type$ rule \rref{R-Simp} delegates the work of
building evidence, when a matching rule $\rulet$ type is found in the
environment, to rule \rref{L-RuleMatch}. The evidence consists of two parts:
$E$ is the evidence of matching $\type$ against $\rulet$. This match contains
placeholders $\bar{x}$ for the contexts whose resolution is postponed by rule
\rref{M-IAbs}. It falls to rule \rref{L-RuleMatch} to perform these
postponed resolutions, obtain their evidence $\bar{E}$ and fill in the
placeholders.

%- - - - - - - - - - - - - - - - - - - - - - - - - - - - - - - - - - - - - - - - 
\paragraph{Meta-Theory} The type-directed translation of $\name$ to System F exhibits a number
of desirable properties.

\begin{theorem}[Type-Preserving Translation]\label{thm:type:preservation} Let $e$ be a $\name$
  expression, $\rulet$ be a type, $\tenv$ be a type environment and $E$ be a System F expression. If
  $\wte{\tenv}{e}{\rulet}{E}$, then $\fwte{\etrans{\tenv}}{E}{\ttrans[\rulet]}$.
\end{theorem}
Here we define the translation of the type environment form $\name$ to System F as:
\begin{equation*}
\begin{array}{rcl@{\hspace{2cm}}rcl}
\etrans{\epsilon} & = & \epsilon & \etrans{\tenv,\alpha} & = & \etrans{\tenv}, \alpha \\
\etrans{\tenv,x : \rulet} & = & \etrans{\tenv}, x : \ttrans[\rulet] &
\etrans{\tenv, \rulet \leadsto x} & = & \etrans{\tenv}, x : \ttrans[\rulet]
\end{array}
\end{equation*}

An important lemma in the theorem's proof is the type preservation of 
resolution.
\begin{lemma}[Type-Preserving Resolution]
Let $\tenv$ be a type environment, $\rulet$ be a type and E be a System F expression.
If $\ares{\tenv}{\rulet}{E}$, then $\fwte{\etrans{\tenv}}{E}{\ttrans[\rulet]}$.
\end{lemma}
% Both the theorem and the lemma are proven in Appendix~\ref{proof:preservation}.

Moreover, we can express three key properties of Figure~\ref{fig:resolution2}'s
definition of resolution in terms of the generated evidence.
Firstly, the deterministic version of resolution is a sound variation on the original ambiguous resolution.
\begin{theorem}[Soundness]
Figure~\ref{fig:resolution2}'s definition of resolution is sound (but
incomplete) with respect to Figure~\ref{fig:resolution1}'s definition.
\[\forall \tenv, \rulet, E: \quad\quad \dres{\tenv}{\rulet}{E} \quad\Rightarrow\quad \ares{\tenv}{\rulet}{E} \]
\end{theorem}
Secondly, the deterministic resolution is effectively deterministic:
\begin{theorem}[Determinacy]
The generated evidence of resolution is uniquely determined.
\[\forall \tenv, \rulet, E_1, E_2: \quad\quad 
     \dres{\tenv}{\rulet}{E_1} ~\wedge~ 
     \dres{\tenv}{\rulet}{E_2} \quad\Rightarrow\quad E_1 = E_2 \]
\end{theorem}
It follows immediately that deterministic resolution is also coherent,
which is formulated as follows for our elaboration-based setting.
\begin{corollary}[Coherence]
\[\forall \tenv, \rulet, E_1, E_2: \quad\quad 
     \dres{\tenv}{\rulet}{E_1} ~\wedge~ 
     \dres{\tenv}{\rulet}{E_2} \quad\Rightarrow\quad 
     \fctx{\etrans{\tenv}}{E_1}{E_2}{\ttrans[\rulet]} \]
\end{corollary}
Here the judgment $\fctx{\tenv}{E_1}{E_2}{T}$
denotes \emph{contextual equivalence}, which captures that
$E_1$ and $E_2$ behave the same in any well-typed program context. We do not give the
formal definition here as it is not required to prove the corollary; we only
need to know that it is an equivalence relation, and, more spefically, that
it is reflexive, i.e., any well-typed System~F term is contextually equivalent
to itself. 

Thirdly, on top of the immediate coherence of deterministic resolution, 
an additional stability property holds.  
\begin{lemma}[Stability of Resolution]
Resolution is stable under type substitution.
\[\forall \tenv,\alpha,\tenv',\sigma,\rulet, E: \enskip\
\dres{\tenv,\alpha,\tenv'}{\rulet}{E} \enskip\wedge\enskip \tenv \vdash \sigma
\enskip\Rightarrow\enskip 
\dres{\tenv,\tenv'[\sigma/\alpha]}{\rulet[\sigma/\alpha]}{E[\ttrans[\sigma]/\alpha]} \]
\end{lemma}%
This is a key lemma to establish that static reduction of type application 
preserves typing and elaboration.
\begin{lemma}[Stability of Typing under Type Application Reduction]
Static reduction of type application preserves typing.
\[\forall \tenv,\alpha,\sigma,\rulet, E: \enskip\
\wte{\tenv}{(\Lambda \alpha.e)\,\suty}{\rulet}{(\Lambda \alpha.E)\,\ttrans[\suty]}
\enskip\Rightarrow\enskip 
\wte{\tenv}{e[\suty/\alpha]}{\rulet}{E[\ttrans[\suty]/\alpha]}
   \]
\end{lemma}%
This, together with another property of System~F's contextual equivalence:
\[ \forall \tenv, \alpha, E, T, T': \enskip\
     \fwte{\tenv}{(\Lambda \alpha.E)\,T}{T'} \enskip\Rightarrow\enskip 
     \fctx{\tenv}{(\Lambda \alpha.E)\,T}{E[T/\alpha]}{T'}\]
allows us to conclude the correctness of static reduction of type application.
\begin{theorem}[Correctness of Type Application Reduction]
If a type application and its reduced form elaborate to two System~F terms,
those terms are contextually equivalent.
\begin{equation*}
\forall \tenv, \alpha, e, \suty, E1, E2 : \enskip\\
\wte{\tenv}{(\Lambda \alpha.e)\,\suty}{\rulet}{E_1} ~\wedge~
\wte{\tenv}{e[\suty/\alpha]}{\rulet}{E_2}
\enskip\Rightarrow\enskip 
\fctx{\etrans{\tenv}}{E_1}{E_2}{\ttrans[\rulet]}
\end{equation*}
\end{theorem}

%-------------------------------------------------------------------------------
\subsection{Evidence Generation in the Algorithm}

The evidence generation in Figure~\ref{fig:algorithm} is largely similar to
that in the deterministic specification of resolution in
Figure~\ref{fig:resolution2}.
With the evidence we can state the correctness of the algorithm.

% The main difference, and complication, is due to the fact that the evidence for
% type instantiation and recursive resolution is needed before these operations
% actually take place, as the algorithm has to postpone them. For this reason, the
% algorithm first produces placeholders that are later substituted for the actual
% evidence.
% 
% The central relation is $\rulet; \bar{\rulet}; \bar{\alpha} \gbox{; \bar{\omega};
% E}\turns_{\mathit{match}} \tau \hookrightarrow \bar{\rulet}' \gbox{;
% \bar{\omega}'; E'}$. It captures the matching instantiation of context type
% $\rulet$ against simple type $\tau$.  The input evidence for $\rulet$ is $E$, and
% the output evidence for the instantiation is $E'$. The accumulating parameters
% $\bar{\alpha}$ and $\bar{\rulet}$ denote that the instantiation of type variables
% $\bar{\alpha}$ and the recursive resolution of $\bar{\rulet}$ have been
% postponed. We use the $\bar{\alpha}$ themselves as convenient placeholders for
% the instantiating types, and we use the synthetic $\bar{\omega}$ as placeholders for
% the evidence of the $\bar{\rulet}$. The rules \rref{MTC-Abs} and \rref{MTC-Abs} 
% introduce these two kinds of placeholders in the evidence. The former kind, $\bar{\alpha}$,
% are substituted in rule \rref{MTC-Simp} where the actual type instantiatons $\bar{\theta}$
% are computed. The latter kind, $\bar{\omega}$, are substituted later in rule \rref{Alg-Simp} where
% the recursive resolutions take place.

\begin{theorem}[Partial Correctness]
Let $\tenv$ be a type environment, $\rulet$ be a type and E be a System F expression.
Assume that $\unambig{\epsilon}{\rulet}$ and also $\forall \rulet_i \in \tenv: \unambig{\epsilon}{\rulet_i}$.
Then $\dres{\tenv}{\rulet}{E}$ if and only if $\adres{\tenv}{\rulet}{E}$,
provided that the algorithm terminates.
\end{theorem}

%-------------------------------------------------------------------------------
\subsection{Dynamic Semantics}
Finally, we define the dynamic semantics of $\name$ as the composition of
the type-directed translation and System F's dynamic semantics.  Following
Siek's notation~\cite{systemfg}, this dynamic semantics is:
\[ \mathit{eval}(e) = V \quad\quad \textit{where } \wte{\epsilon}{e}{\rulet}{E} \textit{ and } E \rightarrow^* V  \]
with $\rightarrow^*$ the reflexive, transitive closure of System F's standard single-step call-by-value reduction relation (see Chapter 23 of \cite{tapl}).

Now we can state the conventional type safety theorem for $\name$:
\begin{theorem}[Type Safety]
If $\wtep{\epsilon}{e}{\rulet}$, then $\mathit{eval}(e) = V$ for
some System F value $V$.
\end{theorem}
The proof follows trivially from Theorem~\ref{thm:type:preservation}.


