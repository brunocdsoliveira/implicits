\section{Discussion}

In the design of \name we had to take several design decisions. In
this section we discuss and justify several of those design decisions.
Mostly, many of these choices are motivated by the design of Haskell 
type classes or Scala implicits.

\bruno{Cite some of the papers that discuss various design options for Haskell?}


\subsection{Predicative Instantiation}

System F is an impredicative calculus, allowing the instantiation of
type variables with arbitrary (polymorphic) types. In contrast \name 
is predicative: instantiation of type
variables only allows \emph{monotypes}. The reason to depart from
traditional System F is due to three factors:

\begin{itemize}

\item {\bf Impredicative instantiation in resolution leads to additional
    ambiguity.} As discussed in Section~\ref{}, if instantiations of
  type variables during resolution are allowed to be impredicative,
  then additional ambiguity is possible. It is not obvious how to
  remove such ambiguity while retaining impredicativity. The
  restriction to predicative instantiation removes such ambiguity.

\item {\bf Impredicativity is complex when some implicit instantiation
  is allowed.} While traditional System F (where all type
instantiations are explicit) is simple, matters become much more
complicated when some implicit instantiation is allowed. Indeed the
design of System F-like calculi with implicit instantiation and/or
some form of type-inference is much more divided in terms of design
choices regarding (im)predicativity. One key complication has todo
with the most-general relation for implicit
polymorphism. The most-general relation allows systems with implicit 
polymorphism to decide whether a type is an instance of another type. 
For example the type |Int -> Int| is an instance of the polymorphic 
type |forall a . a -> a|. An algorithm for determining whether a type
is more general than another must perform implicit instantiation. 
Unfortunatelly it is well-known that for \emph{impredicative
  instantiation}, such most-general relation is undecidable~\cite{}. 
However, when only predicative instantiation is allowed, then 
the most-general relation is decidable~\cite{}. 
Resolution in \name is closely related to the most-general relation 
and we believe that, under impredicative instantiation, resolution is
indeed undecidable.\bruno{This may actually be a known result. See Rouvet?}

\item {\bf Predicative instantiation is not a big restriction in
    practice.} 
Due to the complications brought by impredicativity, many practical 
languages with type-inference only allow predicative instantiation.
For example, the key algorithm for type-inference currently employed 
by the GHC Haskell compiler is predicative~\cite{}. Other languages
with sophisticated forms of type-inference use similar
algorithms~\cite{}. Worth noting is that the original Hindley-Milner (HM)
system is where the predicativity restriction on polymorphic type
systems with type-inference was firstly put into place. 
Since \name is intended as a target for languages for languages with 
implicit polymorphism and type-inference, which often have
predicativity restrictions, restricting the core language to allow
predicative instantiation only is does not lose expressive power in practice.

\end{itemize}

\paragraph{Alternative Design Choices} One alternative design choice
worth mentioning for \name would be to allow impredicative
instantiation in type applications, but still retain the predicativity
restriction during resolution. This design would be less restrictive
than the design of \name, and we believe that it is a reasonable
design. We decided to also have the predicative instantiation even 
for the explicit type applications of \name for two reasons. Firstly, 
since \name is aimed to be a target for source languages with
type-inference, which often have predicative restrictions anyway, then 
there is not much to be gained by having impredicative instantiation
in the core. Secondly, and more importantly, some of the meta-theory
would be more involved if impredicative instantiation on type
applications was allowed. In particular, Lemma~\ref{} would need 
to be generalized to account for any types, rather than just
monotypes.

Allowing full impredicativity (both in type applications and
resolution) seems more complicated. We expect that such designs 
are possible, but necessaraly more complicated if ambiguity and
undecidability are to be avoided. We expect that the work on
impredicative type-inference~\cite{} is relevant, and perhaps some 
of the design choices employed in those works would be helpful 
in the design of a system with full impredicativity.  


\subsection{Committed Choice}

\name employs a committed choice for resolution. The motivation for this choice is 
largely due to the startegy employed by Haskell. Since early on it was decided that Haskell 
should not use backtracking during resolution. One key motivator for this is reasoning. Another 
one is performance.

\paragraph{Overlapping Instances}

Consider the following piece of code:


> data IntSet = C a deriving Eq
> 
> class C a where
>   m :: a -> a
>
> instance Eq a => C [a] where
> 
> instance Ord a => C [a] where
> 
> fun :: StablePtr Int -> [StablePtr Int]
> fun sp = m [sp]

The point here is that StablePtr supports equality but not ordering. That is there 
is a type class instance |Eq (StablePtr a)|, but no type class instance |Ord (StablePtr a)|. 

The question is should the above code type-check or not? In GHC Haskell the answer is no. 
Even though for this program there is no ambiguity: the only choice is to pick the first 
type class instance, the program is nevertheless rejected. 

This is because Haskell's resolution does not consider the contexts in the resolution process:
it only considers the head of a type class instance. 

\section{Superclasses}

Superclasses are not supported by \name. Here \name follows the design of Scala implicits, which 
do not have a concept similar to superclasses. In Scala a similar mechanism to 
superclasses can often be achieved with OO Subtyping and class hierarchies, although there are situations 
that don't work well. \bruno{Should I really go into this?} 

Superclasses look alot like overlapping instances, for example:

>

But resolution in the presence of superclasses behaves differently from overlapping instances. 

Superclasses have a ad-hoc treatement in GHC. 

\subsection{Coherence}

We use coherence as determinism in our paper. But, generally speaking, coherence can be interpreted
more broadly. 


 