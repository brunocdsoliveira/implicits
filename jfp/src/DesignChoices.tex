\section{Discussion}

In the design of \name we had to take several design decisions. In
this section we discuss and justify several of those design decisions.
Mostly, many of these choices are motivated by the design of Haskell 
type classes or Scala implicits.

\bruno{Cite some of the papers that discuss various design options for Haskell?}


\subsection{Predicative Instantiation}

System F is an impredicative calculus, allowing the instantiation of type variables with 
arbitrary (polymorphic) types. In contrast most functional languages only allow instantiation 
of \emph{monotypes}. Cochis is a predicative calculus. 

\subsection{Committed Choice}

\name employs a committed choice for resolution. The motivation for this choice is 
largely due to the startegy employed by Haskell. Since early on it was decided that Haskell 
should not use backtracking during resolution. One key motivator for this is reasoning. Another 
one is performance.

\paragraph{Overlapping Instances}

Consider the following piece of code:


> data IntSet = C a deriving Eq
> 
> class C a where
>   m :: a -> a
>
> instance Eq a => C [a] where
> 
> instance Ord a => C [a] where
> 
> fun :: StablePtr Int -> [StablePtr Int]
> fun sp = m [sp]

The point here is that StablePtr supports equality but not ordering. That is there 
is a type class instance |Eq (StablePtr a)|, but no type class instance |Ord (StablePtr a)|. 

The question is should the above code type-check or not? In GHC Haskell the answer is no. 
Even though for this program there is no ambiguity: the only choice is to pick the first 
type class instance, the program is nevertheless rejected. 

This is because Haskell's resolution does not consider the contexts in the resolution process:
it only considers the head of a type class instance. 

\section{Superclasses}

Superclasses are not supported by \name. Here \name follows the design of Scala implicits, which 
do not have a concept similar to superclasses. In Scala a similar mechanism to 
superclasses can often be achieved with OO Subtyping and class hierarchies, although there are situations 
that don't work well. \bruno{Should I really go into this?} 

Superclasses look alot like overlapping instances, for example:

>

But resolution in the presence of superclasses behaves differently from overlapping instances. 

Superclasses have a ad-hoc treatement in GHC. 

\subsection{Coherence}

We use coherence as determinism in our paper. But, generally speaking, coherence can be interpreted
more broadly. 


 