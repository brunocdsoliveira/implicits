Implicit Progamming (IP) mechanisms infer values by type-directed
resolution, making programs more compact and easier to read.
Examples of IP mechanisms include Haskell's type classes, Scala's
implicits, Agda's instance arguments, Coq's type classes, and Rust's
traits.  The design of IP mechanisms has led to heated debate:
proponents of one school argue for the desirability of strong
reasoning properties; while proponents of another
school argue for the power and flexibility of local scoping or
overlapping instances.  The current state of affairs seems to indicate that
the two goals are at odds with one another and cannot easily be
reconciled.

This paper presents \name, the Calculus Of CoHerent ImplicitS,
an improved variant of the implicit calculus that offers flexibility
while preserving two key properties: \emph{coherence} and
\emph{stability of substitutions}. \name supports
polymorphism, local scoping,
overlapping instances, first-class instances, and higher-order
rules, while remaining type safe, coherent and stable under substitution.

\name has a compact formulation.
We introduce a logical formulation of how to resolve implicits, which
is simple but ambiguous and incoherent, and a second formulation,
which is less simple but unambiguous, coherent and stable.  Every resolution
of the second formulation is also a resolution of the first, but not
conversely.  Parts of the second formulation bear a close resemblance
to a standard technique for proof search called focussing.
\bruno{We should consider changing the title. Actually our main
  novelty isn't coherent implicits: stability is more of the novelty
  here. }

%%%%%%%%%%%%%%%%%%%%%%%%%%%%%%%%%%%%%%%%%%%%%%%%%%%%%%%%%%%%%%%%%%%%%%%%

%% Haskell type classes are a highly successful mechanism for
%% type-directed overloading. Greatly inspired by type classes, 
%% many other languages (such as Scala, Agda,
%% Coq and Rust) developed their own mechanisms 
%% for type-directed overloaded. However, 

%% Many other languages (such as Scala, Agda,
%% Coq)\bruno{others?} include mechanisms inspired by type classes, but
%% with some significant differences.

%% \emph{Generic programming} (GP) is an increasingly important trend in
%% programming languages. Well-known GP mechanisms, such as type classes
%% and the C++0x concepts proposal, usually combine two features: 1) a special type of
%% interfaces; and 2) \emph{implicit instantiation} of implementations of
%% those interfaces.

%% Scala \emph{implicits} are a GP language mechanism, inspired by type
%% classes, that break with the tradition of coupling
%% implicit instantiation with a special type of interface. Instead,
%% implicits provide only implicit instantiation, which is generalized to
%% work for \emph{any types}. 
%% %%In particular, implicit instantiation works
%% %%with any interface types in the language. 
%% %% and those interface types 
%% %%can be used to model concepts. 
%% Scala implicits turn out to be quite
%% powerful and useful to address many limitations that show up in other
%% GP mechanisms.

%% This paper synthesizes the key ideas of implicits formally in a minimal
%% and general core calculus called the implicit calculus (\name),
%% and it shows how to build source languages supporting implicit
%% instantiation on top of it. A novelty of the calculus is its support
%% for \emph{partial resolution} and \emph{higher-order rules} (a feature 
%% that has been proposed before, but was never formalized or implemented).
%% Ultimately, the implicit calculus provides a formal model of implicits, 
%% which can be used by language designers to 
%% study and inform implementations of similar mechanisms in their own languages.

%%%%%%%%%%%%%%%%%%%%%%%%%%%%%%%%%%%%%%%%%%%%%%%%%%%%%%%%%%%%%%%%%%%%%%%%

%%The calculus is relevant because it offers an interesting
%% because it provides a simple model 

%% This paper presents a core calculus generic programming

%%Our paper addresses this lack of formalization; it paper presents a general,
%%yet minimal calculus of implicits. The calculus offers the basic mechanisms for
%%type-directed resolution and scoping of implicits. These mechanisms are the key
%%ingredients for modeling powerful forms of implicit instantiation for source
%%languages. A core feature of the calculus is the fact that resolution does not
%%require a special concept-like interface, but works for any type.

%%We present the syntax, operational semantics, type system, an efficient
%%implementation through a type-directed translation, as well as correctness
%%results. Finally, we show how generic programming constructs are translated
%%into our calculus.

%%\paragraph{Version 2:}
%%Implicit programming that infers values by using type
%%information has proven its benefits in practice. However, 
%%years of experience have exposed several limitations in its current practice.
%%For example, Haskell's type classes, which are the oldest and most
%%prominent form of implicit programming have several
%%limitations of its original design: no
%%ability to control the \emph{scoping} of rules (instances); the \emph{second
%%  class} nature of types classes compared to types; and type class
%%and type class rules being limited to {first-order} cases.

%%This paper presents a \textit{implicit calculus}: a powerful and expressive,
%%yet minimalistic, calculus that provides an implicit programming model
%%without the limitations of the current practice. The implicit calculus
%%offers control over scoping, does not suffer from the second class
%%nature of type classes, and incoorporates two features which
%%have been discussed in the literature but for which there is
%%currently no good solution: \emph{higher-order rules} and
%%\emph{coherent support for overlapping rules}.  We present the
%%syntax, operational semantics, type system, type-directed translation
%%into System F, as well as correctness results. Finally, we show how
%%Haskell and Scala's implicit programming constructs are translated
%%into our calculus.

%%%%%%%%%%%%%%%%%%%%%%%%%%%%%%%%%%%%%%%%%%%%%%%%%%%%%%%%%%%%%%%%%%%%%%%%

%% Implicit programming that infers values by using type
%% information has proven its benefits in practice. However, 
%% years of experience have exposed several limitations in its current practice.
%% For example, Haskell's type classes, which are the oldest and most
%% prominent form of implicit programming have several
%% limitations of its original design: no
%% ability to control the \emph{scoping} of instances; the \emph{second
%%   class} nature of types classes compared to types; and type class
%% constraints being limited to \emph{first-order predicates}.

%% This paper presents a \textit{implicit calculus}: a powerful and expressive,
%% yet minimalistic, calculus that provides an implicit programming model
%% without the limitations of the current practice. The implicit calculus
%% offers control over scoping, does not suffer from the second class
%% nature of type classes, and incoorporates two features which
%% have been discussed in the literature but for which there is
%% currently no good solution: \emph{higher-order predicates} and
%% \emph{coherent support for overlapping instances}.  We present the
%% syntax, operational semantics, type system, type-directed translation
%% into System F, as well as correctness results. Finally, we show how
%% Haskell and Scala's implicit programming constructs are translated
%% into our calculus.

%%%%%%%%%%%%%%%%%%%%%%%%%%%%%%%%%%%%%%%%%%%%%%%%%%%%%%%%%%%%%%%%%%%%%%%%

%%% Local Variables: 
%%% mode: latex
%%% TeX-master: "../Main"
%%% End: 
