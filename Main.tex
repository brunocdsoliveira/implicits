\documentclass[acmlarge, anonymous, review]{acmart}
%\documentclass[prodmode,acmtoplas]{acmsmall}


\usepackage{color}
\usepackage{xspace}
\usepackage{graphicx}
\usepackage{latexsym}
\usepackage{stmaryrd}
\usepackage{amsmath}
\usepackage{amssymb}
% \usepackage{amsthm}
\usepackage{stmaryrd}
\usepackage{mathpartir}
\usepackage{wasysym}
\usepackage{url}
\usepackage{algorithm, algpseudocode}
\usepackage{fancyvrb}
%\usepackage[[style=1]{mdframed}
\usepackage{comment}
\usepackage{soul}

\definecolor{light}{gray}{.75}

% \mdfdefinestyle{MyFrame}{%
%     linecolor=white,
% %%    outerlinewidth=2pt,
% %%    roundcorner=20pt,
% %%    innertopmargin=\baselineskip,
% %%    innerbottommargin=\baselineskip,
% %%    innerrightmargin=20pt,
% %%    innerleftmargin=20pt,
%     backgroundcolor=light}

\newcommand{\gbox}[1]{\colorbox{light}{$\!\!#1\!\!$}}

%%\bibliographystyle{plain}

%% for comments
\newcommand{\yank}[1]{\textbf{yank:}}
\newcommand{\todoc}[2]{{\textcolor{#1} {\textbf{[[#2]]}}}}
\newcommand{\todo}[1]{{\todoc{red}{\textbf{[[#1]]}}}}

\newcommand{\todored}[1]{\todoc{red}  {\textbf{[[#1]]}}}
\newcommand{\todoblue}[1]{\todoc{blue}{\textbf{[[#1]]}}}
\newcommand{\todogreen}[1]{\todoc{green}{\textbf{[[#1]]}}}

\newcommand{\TODO}[1]{\todored{#1}}

\newcommand{\wc}[1]{\todogreen{WC: #1}}
\newcommand{\wt}[1]{\todogreen{WT: #1}}
\newcommand{\tom}[1]{\marginpar{\textcolor{red}{TOM}}\textcolor{red}{#1}}
\newcommand{\bruno}[1]{\todogreen{BRUNO: #1}}
\newcommand{\klara}[1]{\todogreen{KLARA: #1}}

% \newcommand{\wc}[1]{}
% \newcommand{\wt}[1]{}
% \newcommand{\tom}[1]{}
% \newcommand{\bruno}[1]{}

%%% Local Variables: 
%%% mode: latex
%%% TeX-master: "Main"
%%% End: 

\newcommand{\arrow}{\rightarrow}

% \newtheorem{theorem}{Theorem}[section]
% \newtheorem{conjecture}{Conjecture}[section]
% \newtheorem{lemma}{Lemma}[section]
% \newtheorem{definition}{Definition}[section]
% \newtheorem{proposition}[theorem]{Proposition}
% \newtheorem{corollary}[theorem]{Corollary}

\newcommand{\suty}{\sigma}
\newcommand{\name}{\textsc{Cochis}\xspace}

\newcommand{\mca}{\mathcal{A}}
\newcommand{\mcb}{\mathcal{B}}
\newcommand{\mcc}{\mathcal{C}}
\newcommand{\mcd}{\mathcal{D}}
\newcommand{\mce}{\mathcal{E}}
\newcommand{\mcf}{\mathcal{F}}
\newcommand{\mcg}{\mathcal{G}}
\newcommand{\mch}{\mathcal{H}}
\newcommand{\mci}{\mathcal{I}}
\newcommand{\mcj}{\mathcal{J}}
\newcommand{\mck}{\mathcal{K}}
\newcommand{\mcl}{\mathcal{L}}
\newcommand{\mcm}{\mathcal{M}}
\newcommand{\mcn}{\mathcal{N}}
\newcommand{\mco}{\mathcal{O}}
\newcommand{\mcp}{\mathcal{P}}
\newcommand{\mcq}{\mathcal{Q}}
\newcommand{\mcr}{\mathcal{R}}
\newcommand{\mcs}{\mathcal{S}}
\newcommand{\mct}{\mathcal{T}}
\newcommand{\mcu}{\mathcal{U}}
\newcommand{\mcv}{\mathcal{V}}
\newcommand{\mcw}{\mathcal{W}}
\newcommand{\mcx}{\mathcal{X}}
\newcommand{\mcy}{\mathcal{Y}}
\newcommand{\mcz}{\mathcal{Z}}

\newcommand{\ourlang}{\name}
%\newcommand{\ourlang}{\lambda_{?}}
\newcommand{\ourpolylang}{\lambda_{\rho}^{\it poly}}

\newcommand{\figtwocol}[3]
{\begin{figure} #3 \caption{#2}\label{#1}\end{figure}}

\newcommand{\rref}[1]{\hyperlink{rule:#1}{\mylabel{#1}}}
\newcommand{\myrule}[3]{\inferrule*[left={\hypertarget{rule:#1}{\scriptsize\mylabel{#1}}}]{#2}{#3}\label{rule:#1}}
\newcommand{\myexrule}[3]{\inferrule*[left={\hyperlink{rule:#1}{\scriptsize\mylabel{#1}}}]{#2}{#3}\label{rule:#1}}
\newcommand{\myexruleL}[3]{\inferrule*[Left={\hyperlink{rule:#1}{\scriptsize\mylabel{#1}}}]{#2}{#3}\label{rule:#1}}
\newcommand{\myirule}[2]{\inferrule*{#1}{#2}}
%\newcommand{\myirule}[2]{{\renewcommand{\arraystretch}{1.2}\ba{c} #1
%                      \\ \hline #2 \ea}}

\newcommand{\ba}{\begin{array}}
\newcommand{\ea}{\end{array}}
\newcommand{\bda}{\[\ba}
\newcommand{\eda}{\ea\]}

\newcommand{\myset}[1]{\{#1\}}
\newcommand{\relation}[2]{#1:#2}
\newcommand{\myrelation}[2]{#1\!:\!#2}
\newcommand{\To}{\Rightarrow}
\newcommand{\ie}[2]{#1~\gbox{\leadsto #2}}

\newcommand{\Abs}[2]{\Lambda #1. #2}
\newcommand{\abs}[2]{\lambda #1. #2}
\newcommand{\tall}[2]{\forall #1. #2}
%\newcommand{\ruleabs}[2]{\langle\!|  #2  :  #1 |\!\rangle}
%\newcommand{\ruleabs}[2]{(\!|  #2  :  #1 |\!)}
\newcommand{\ruleabs}[2]{\lambda_? #1. #2}
\newcommand{\ruleapp}[2]{#1 {\bf ~with~} #2}
\newcommand{\iarrow}{\Rightarrow}
\newcommand{\ilambda}{\lambda_{?}}
\newcommand{\with}{{\bf ~with~}}
\newcommand{\query}{?}
\newcommand{\type}{\tau}
\newcommand{\tyint}{{\it Int}}
\newcommand{\tybool}{{\it Bool}}
\newcommand{\tychar}{{\it Char}}
\newcommand{\tystr}{{\it String}}
\newcommand{\tyunit}{()}
\newcommand{\unit}{()}
\newcommand{\btrue}{{\it True}}
\newcommand{\bfalse}{{\it False}}
\newcommand{\Eq}{{\tt Eq}}
\newcommand{\ShowC}{{\tt ShowC}}
\newcommand{\Monad}{{\tt Monad}}
\newcommand{\MonadTrans}{{\tt MonadTrans}}
\newcommand{\GTree}{{\tt GTree}}
\newcommand{\TT}{{\tt \#t}}
\newcommand{\FF}{{\tt \#f}}


\newcommand{\kind}{\kappa}
\newcommand{\rulet}{\rho}
\newcommand{\ruleschr}[3]
{\forall #1. #2 \To #3}
\newcommand{\rulesch}[3]
{\forall \vec{#1}. #2 \To #3}
\newcommand{\ruleset}{\bar{\rulet}}
\newcommand{\rulesetvar}{\ruleset}
\newcommand{\nil}{\varnothing}
\newcommand{\meta}[1]{{\it #1 }}
\newcommand{\finto}{\stackrel{\mathsf{fin}}{\to}}
\newcommand{\rulepgm}{\overline{\relation{\rulet}{v}}}
\newcommand{\rulepgmvar}{\eta}
\newcommand{\grulepgm}[2]{\overline{\relation{#1}{#2}}}
\newcommand{\rulesetexp}{\overline{\relation{e}{\rulet}}}

\newcommand{\FV}     {\meta{fv}}
\newcommand{\FTV}    {\meta{FTV}}
\newcommand{\ftv}[1] {\meta{ftv}(#1)}
\newcommand{\BV}     {\meta{bv}}
\newcommand{\BTV}    {\meta{BTV}}
\newcommand{\dom}    {\meta{dom}}

\newcommand{\tstate}{\mce}
\newcommand{\rclos}[1]{(#1)}
\newcommand{\lookup}[2]{{#1\langle{#2}\rangle}}
\newcommand{\kenv}{\Phi}
\newcommand{\denv}{\Delta}
\newcommand{\env}{\Delta}
\newcommand{\tenv}{\Gamma}
\newcommand{\eval}{\Downarrow}
\newcommand{\turns}{\vdash}
\newcommand{\fturns}{\vdash^\mathrm{f}_{\mathrm{r}}}
\newcommand{\vturns}{\vdash^\mathrm{a}_{\mathrm{r}}}
\newcommand{\ivturns}{\mathop{\turns_{\mathrm{r}}}}
\newcommand{\kturns}{\vdash_{\mathrm{k}}}
\newcommand{\vtyping}{\models}
\newcommand{\unamb}{\vdash_{\mathrm{unamb}}}


\newcommand{\case}[1]{\----{\bf case}~#1}
\newcommand{\facerule}[1]{{\tt\sc #1}}
\newcommand{\tlabel}[1]{\mbox{(#1)}}

\newcommand{\mylabel}[1]{\tlabel{\facerule{#1}}}

\newcommand{\OpQuery}{\tlabel{\facerule{OpQuery}}}
\newcommand{\OpRule}{\tlabel{\facerule{OpRule}}}
\newcommand{\OpInst}{\tlabel{\facerule{OpInst}}}
\newcommand{\OpRApp}{\tlabel{\facerule{OpRApp}}}
\newcommand{\OpRClos}{\tlabel{\facerule{OpRClos}}}

\newcommand{\RTAbs}{\tlabel{\facerule{RTAbs}}}
\newcommand{\RIAbs}{\tlabel{\facerule{RIAbs}}}
\newcommand{\RSimpl}{\tlabel{\facerule{RSimp}}}

\newcommand{\IDone}{\tlabel{\facerule{IDone}}}
\newcommand{\ITAbs}{\tlabel{\facerule{ITAbs}}}
\newcommand{\IIAbs}{\tlabel{\facerule{IIAbs}}}

\newcommand{\LHere}{\tlabel{\facerule{LHead}}}
\newcommand{\LThere}{\tlabel{\facerule{LTail}}}

\newcommand{\MDone}{\tlabel{\facerule{MDone}}}
\newcommand{\MTAbs}{\tlabel{\facerule{MTAbs}}}
\newcommand{\MIAbs}{\tlabel{\facerule{MIAbs}}}

\newcommand{\TrRTAbs}{\tlabel{\facerule{TrRTAbs}}}
\newcommand{\RRho}{\tlabel{\facerule{RRho}}}
\newcommand{\TrRIAbs}{\tlabel{\facerule{TrRIAbs}}}
\newcommand{\TrRSimpl}{\tlabel{\facerule{TrRSimp}}}

\newcommand{\TrIDone}{\tlabel{\facerule{TrIDone}}}
\newcommand{\TrITAbs}{\tlabel{\facerule{TrITAbs}}}
\newcommand{\TrIIAbs}{\tlabel{\facerule{TrIIAbs}}}

\newcommand{\TrLHere}{\tlabel{\facerule{TrLHead}}}
\newcommand{\TrLThere}{\tlabel{\facerule{TrLTail}}}


\newcommand{\TrInt}{\tlabel{\facerule{TrInt}}}
\newcommand{\TrVar}{\tlabel{\facerule{TrVar}}}
\newcommand{\TrAbs}{\tlabel{\facerule{TrAbs}}}
\newcommand{\TrApp}{\tlabel{\facerule{TrApp}}}
\newcommand{\TrTAbs}{\tlabel{\facerule{TrTAbs}}}
\newcommand{\TrTApp}{\tlabel{\facerule{TrTApp}}}
\newcommand{\TrIAbs}{\tlabel{\facerule{TrIAbs}}}
\newcommand{\TrIApp}{\tlabel{\facerule{TrIApp}}}
\newcommand{\TrQuery}{\tlabel{\facerule{TrQuery}}}

\newcommand{\TermSimpl}{\tlabel{\facerule{T-Simp}}}
\newcommand{\TermTyVar}{\tlabel{\facerule{TermTyVar}}}
\newcommand{\TermInt}{\tlabel{\facerule{TermInt}}}
\newcommand{\TermForall}{\tlabel{\facerule{T-Forall}}}
\newcommand{\TermFun}{\tlabel{\facerule{TermFun}}}
\newcommand{\TermRule}{\tlabel{\facerule{T-Rule}}}

\newcommand{\ResNil}{\tlabel{\facerule{ResNil}}}
\newcommand{\Res}{\tlabel{\facerule{Res}}}
\newcommand{\SimpleRes}{\tlabel{\facerule{SimpleRes}}}
\newcommand{\RuleRes}{\tlabel{\facerule{RuleRes}}}
\newcommand{\StaRes}{\tlabel{\facerule{TyRes}}}
\newcommand{\DynRes}{\tlabel{\facerule{DynRes}}}

\newcommand{\TrRes}{\tlabel{\facerule{TrRes}}}

\newcommand{\Canon}[1]{\mathsf{canon}(#1)}
\newcommand{\unify}[3]{\mcu_{#3}(#1, #2)}

\newcommand{\KiVar}{\tlabel{\facerule{KiVar}}}
\newcommand{\KiApp}{\tlabel{\facerule{KiApp}}}
\newcommand{\KiForall}{\tlabel{\facerule{KiForall}}}

\newcommand{\err}{{\it err}}

\def\ruleform#1{{\setlength{\fboxrule}{1pt}\fbox{\normalsize $#1$}}}

\newcommand{\Ex}[1]{\[\begin{array}{l}#1\end{array}\]}
\newcommand{\Cmnt}[1]{{\tt #1}}

\newcommand{\subst}[2]{[ #1\!\mapsto\!#2 ]}
\newcommand{\substone}{\subst{\alpha}{\type}}

\newcommand{\leteq}{\stackrel{{\sf let}}{=}}
\newcommand{\defeq}{\stackrel{{\sf def}}{=}}

\newcommand{\overlap}{{\sf overlap}}
\newcommand{\nonoverlap}{{\sf nonoverlap}}
\newcommand{\wellformed}{{\sf wellformed}}
%\newcommand{\coherent}{{\sf coherent}}
\newcommand{\distinct}{{\sf distinct}}
\newcommand{\unrelated}{{\sf unrelated}}
\newcommand{\disjoint}{{\sf distinct}}
\newcommand{\converge}{{\sf converge}}
\newcommand{\welldefined}{{\sf welldefined}}

\newcommand{\distinctrs}{{\sf distinct\_rs}}
\newcommand{\distinctctx}{{\sf distinct\_ctx}}
\newcommand{\distinctwith}{{\sf distinct\_with}}
\newcommand{\unambiguous}{{\sf unambiguous}}

\definecolor{gray}{gray}{0.7}
\definecolor{dark-gray}{gray}{0.4}
\definecolor{light-gray}{gray}{0.8}

%\newcommand{\shade}[1]{\colorbox{gray}{\color{dark-gray} {$#1$}}}
%\newcommand{\shade}[1]{\colorbox{light-gray}{\color{dark-gray} {$#1$}}}
\newcommand{\shade}[1]{\colorbox{light-gray}{\color{black}{$#1$}}}
\newcommand{\hide}[1]{}
\def\myruleform#1{
\setlength{\fboxrule}{0.5pt}\fbox{\normalsize $#1$}
}

\newcommand{\WFIntTy}{\tlabel{\facerule{WF-IntTy}}}
\newcommand{\WFVarTy}{\tlabel{\facerule{WF-VarTy}}}
\newcommand{\WFAbsTy}{\tlabel{\facerule{WF-UnivTy}}}
\newcommand{\WFFunTy}{\tlabel{\facerule{WF-FunTy}}}
\newcommand{\WFRulTy}{\tlabel{\facerule{WF-RulTy}}}

\newcommand{\TyTyAbs}{\tlabel{\facerule{TyTyAbs}}}
\newcommand{\TyTyApp}{\tlabel{\facerule{TyTyApp}}}
\newcommand{\TyUnit}{\tlabel{\facerule{TyUnit}}}
\newcommand{\TyInt}{\tlabel{\facerule{Ty-Int}}}
\newcommand{\TyVar}{\tlabel{\facerule{Ty-Var}}}
\newcommand{\TyIntL}{\tlabel{\facerule{TyIntL}}}
\newcommand{\TyLVar}{\tlabel{\facerule{TyLVar}}}
\newcommand{\TyIVar}{\tlabel{\facerule{TyIVar}}}
\newcommand{\TyIClass}{\tlabel{\facerule{TyIClass}}}
\newcommand{\TyRec}{\tlabel{\facerule{TyRec}}}
\newcommand{\TyAbs}{\tlabel{\facerule{Ty-Abs}}}
\newcommand{\TyTAbs}{\tlabel{\facerule{Ty-TAbs}}}
\newcommand{\TyIAbs}{\tlabel{\facerule{Ty-IAbs}}}
\newcommand{\TyApp}{\tlabel{\facerule{Ty-App}}}
\newcommand{\TyTApp}{\tlabel{\facerule{Ty-TApp}}}
\newcommand{\TyIApp}{\tlabel{\facerule{Ty-IApp}}}
\newcommand{\TyLet}{\tlabel{\facerule{TyLet}}}
\newcommand{\TyImp}{\tlabel{\facerule{TyImp}}}
\newcommand{\TyWith}{\tlabel{\facerule{TyWith}}}
\newcommand{\TyAnn}{\tlabel{\facerule{TyAnn}}}

\newcommand{\eqv}{\Longleftrightarrow}


% Judgement Forms
% ~~~~~~~~~~~~~~~

% well-formed types
\newcommand{\wfty}[2]{\hello{#1 \turns #2}}

% well-typed expressions
\newcommand{\wte}[4]{\hello{#1 \turns \relation{#2}{#3}~\gbox{\leadsto #4}}}
\newcommand{\wtep}[3]{\hello{#1 \turns \relation{#2}{#3}}}

% ambiguous resolution
\newcommand{\ares}[3]{\hello{#1 \vturns #2~\gbox{\leadsto #3}}}
\newcommand{\aresp}[2]{\hello{#1 \vturns #2}}

% focusing resolution
\newcommand{\frres}[3]{\hello{#1 \fturns [#2]~\gbox{\leadsto #3}}}
\newcommand{\fmres}[6]{\hello{#1; [#2]~\gbox{\leadsto #3} \fturns #4~\gbox{\leadsto #5}; #6}}

% lookup
\newcommand{\lres}[4]{\hello{#1;[#2] \ivturns #3~\gbox{\leadsto #4}}}

% deterministic resolution
\newcommand{\dres}[3]{\hello{#1 \ivturns #2~\gbox{\leadsto #3}}}
\newcommand{\drres}[4]{\hello{#1; #2 \fturns [#3]~\gbox{\leadsto #4}}}
\newcommand{\dlres}[5]{\hello{#1; #2; [#3] \ivturns #4~\gbox{\leadsto #5}}}
\newcommand{\dmres}[6]{\hello{#1; [#2]~\gbox{\leadsto #3} \ivturns #4~\gbox{\leadsto #5}; #6}}

% algorithm
\newcommand{\alg}{\turns_{\mathit{alg}}}
\newcommand{\adres}[3]{\hello{#1 \alg #2 ~\gbox{\leadsto #3}}}
\newcommand{\adrres}[4]{\hello{#1; #2 \alg #3 ~\gbox{\leadsto #4}}}
\newcommand{\adlres}[5]{\hello{#1; #2; [#3] \alg #4~\gbox{\leadsto #5}}}
\newcommand{\admres}[8]{\hello{#1; #2; [#3] ~\gbox{\leadsto #4}; #5 \alg #6 ~\gbox{\leadsto #7}; #8}}
\newcommand{\mgu}[3][\bar{\alpha}]{\hello{\textit{unify}_{\tenv;#1}(#2,#3)}}
\newcommand{\mgun}[4][\tenv]{\hello{\textit{unify'}_{#2}(#3,#4)}}
\newcommand{\before}[3][\tenv]{\hello{#2 >_{#1} #3}}

% type variables
\newcommand{\tyvars}[1]{\hello{\mathsf{tyvars}(#1)}}

% valid substitutions
\newcommand{\validsubst}[3]{\hello{#1; #2 \vdash #3}}

% stability
\newcommand{\stable}[4]{\hello{{\sf stable}(#1;#2~\gbox{\leadsto #3};#4)}}
\newcommand{\dstable}[5]{\hello{{\sf stable}(#1;#2;#3~\gbox{\leadsto #4};#5)}}

% coherent
\newcommand{\coh}{\turns_{\mathit{coh}}}
\newcommand{\coherent}[4]{\hello{#1; #2; #3 \coh #4}}
\newcommand{\incoherent}[4]{\hello{#1; #2; #3 \not\coh #4}}

% unambiguity condition
\newcommand{\unambig}[2]{\hello{#1 \unamb #2}}

% annotated dmres
\newcommand{\dmresa}[7]{\hello{#1; #2; [#3]~\gbox{\leadsto #4} \ivturns #5~\gbox{\leadsto #6}; #7}}

%termination
\newcommand{\term}[1]{\hello{\turns_\mathit{term} #1}}
\newcommand{\occ}[2]{\hello{\mathit{occ}_{#1}(#2)}}
\newcommand{\tnorm}[1][\cdot]{\hello{\|#1\|}}
\newcommand{\head}[1]{\hello{\mathrm{hd}(#1)}}

%elaboration
\newcommand{\ttrans}[1][\cdot]{\hello{|#1|}}
\newcommand{\etrans}[1]{\hello{|#1|}}

% System F
\newcommand{\fwte}[3]{\hello{#1 \turns_{\scriptscriptstyle \hspace{-1mm}\mathtt{F}} #2 : #3}}

%%% Local Variables: 
%%% mode: latex
%%% TeX-master: "Main"
%%% End: 

%%include lhs2TeX.fmt
%% ODER: format ==         = "\mathrel{==}"
%% ODER: format /=         = "\neq "
%
%
\makeatletter
\@ifundefined{lhs2tex.lhs2tex.sty.read}%
  {\@namedef{lhs2tex.lhs2tex.sty.read}{}%
   \newcommand\SkipToFmtEnd{}%
   \newcommand\EndFmtInput{}%
   \long\def\SkipToFmtEnd#1\EndFmtInput{}%
  }\SkipToFmtEnd

\newcommand\ReadOnlyOnce[1]{\@ifundefined{#1}{\@namedef{#1}{}}\SkipToFmtEnd}
\usepackage{amstext}
\usepackage{amssymb}
\usepackage{stmaryrd}
\DeclareFontFamily{OT1}{cmtex}{}
\DeclareFontShape{OT1}{cmtex}{m}{n}
  {<5><6><7><8>cmtex8
   <9>cmtex9
   <10><10.95><12><14.4><17.28><20.74><24.88>cmtex10}{}
\DeclareFontShape{OT1}{cmtex}{m}{it}
  {<-> ssub * cmtt/m/it}{}
\newcommand{\texfamily}{\fontfamily{cmtex}\selectfont}
\DeclareFontShape{OT1}{cmtt}{bx}{n}
  {<5><6><7><8>cmtt8
   <9>cmbtt9
   <10><10.95><12><14.4><17.28><20.74><24.88>cmbtt10}{}
\DeclareFontShape{OT1}{cmtex}{bx}{n}
  {<-> ssub * cmtt/bx/n}{}
\newcommand{\tex}[1]{\text{\texfamily#1}}	% NEU

\newcommand{\Sp}{\hskip.33334em\relax}


\newcommand{\Conid}[1]{\mathit{#1}}
\newcommand{\Varid}[1]{\mathit{#1}}
\newcommand{\anonymous}{\kern0.06em \vbox{\hrule\@width.5em}}
\newcommand{\plus}{\mathbin{+\!\!\!+}}
\newcommand{\bind}{\mathbin{>\!\!\!>\mkern-6.7mu=}}
\newcommand{\rbind}{\mathbin{=\mkern-6.7mu<\!\!\!<}}% suggested by Neil Mitchell
\newcommand{\sequ}{\mathbin{>\!\!\!>}}
\renewcommand{\leq}{\leqslant}
\renewcommand{\geq}{\geqslant}
\usepackage{polytable}

%mathindent has to be defined
\@ifundefined{mathindent}%
  {\newdimen\mathindent\mathindent\leftmargini}%
  {}%

\def\resethooks{%
  \global\let\SaveRestoreHook\empty
  \global\let\ColumnHook\empty}
\newcommand*{\savecolumns}[1][default]%
  {\g@addto@macro\SaveRestoreHook{\savecolumns[#1]}}
\newcommand*{\restorecolumns}[1][default]%
  {\g@addto@macro\SaveRestoreHook{\restorecolumns[#1]}}
\newcommand*{\aligncolumn}[2]%
  {\g@addto@macro\ColumnHook{\column{#1}{#2}}}

\resethooks

\newcommand{\onelinecommentchars}{\quad-{}- }
\newcommand{\commentbeginchars}{\enskip\{-}
\newcommand{\commentendchars}{-\}\enskip}

\newcommand{\visiblecomments}{%
  \let\onelinecomment=\onelinecommentchars
  \let\commentbegin=\commentbeginchars
  \let\commentend=\commentendchars}

\newcommand{\invisiblecomments}{%
  \let\onelinecomment=\empty
  \let\commentbegin=\empty
  \let\commentend=\empty}

\visiblecomments

\newlength{\blanklineskip}
\setlength{\blanklineskip}{0.66084ex}

\newcommand{\hsindent}[1]{\quad}% default is fixed indentation
\let\hspre\empty
\let\hspost\empty
\newcommand{\NB}{\textbf{NB}}
\newcommand{\Todo}[1]{$\langle$\textbf{To do:}~#1$\rangle$}

\EndFmtInput
\makeatother
%
%
%
%
%
%
% This package provides two environments suitable to take the place
% of hscode, called "plainhscode" and "arrayhscode". 
%
% The plain environment surrounds each code block by vertical space,
% and it uses \abovedisplayskip and \belowdisplayskip to get spacing
% similar to formulas. Note that if these dimensions are changed,
% the spacing around displayed math formulas changes as well.
% All code is indented using \leftskip.
%
% Changed 19.08.2004 to reflect changes in colorcode. Should work with
% CodeGroup.sty.
%
\ReadOnlyOnce{polycode.fmt}%
\makeatletter

\newcommand{\hsnewpar}[1]%
  {{\parskip=0pt\parindent=0pt\par\vskip #1\noindent}}

% can be used, for instance, to redefine the code size, by setting the
% command to \small or something alike
\newcommand{\hscodestyle}{}

% The command \sethscode can be used to switch the code formatting
% behaviour by mapping the hscode environment in the subst directive
% to a new LaTeX environment.

\newcommand{\sethscode}[1]%
  {\expandafter\let\expandafter\hscode\csname #1\endcsname
   \expandafter\let\expandafter\endhscode\csname end#1\endcsname}

% "compatibility" mode restores the non-polycode.fmt layout.

\newenvironment{compathscode}%
  {\par\noindent
   \advance\leftskip\mathindent
   \hscodestyle
   \let\\=\@normalcr
   \(\pboxed}%
  {\endpboxed\)%
   \par\noindent
   \ignorespacesafterend}

\newcommand{\compaths}{\sethscode{compathscode}}

% "plain" mode is the proposed default.

\newenvironment{plainhscode}%
  {\hsnewpar\abovedisplayskip
   \advance\leftskip\mathindent
   \hscodestyle
   \let\\=\@normalcr
   \(\pboxed}%
  {\endpboxed\)%
   \hsnewpar\belowdisplayskip
   \ignorespacesafterend}

% Here, we make plainhscode the default environment.

\newcommand{\plainhs}{\sethscode{plainhscode}}
\plainhs

% The arrayhscode is like plain, but makes use of polytable's
% parray environment which disallows page breaks in code blocks.

\newenvironment{arrayhscode}%
  {\hsnewpar\abovedisplayskip
   \advance\leftskip\mathindent
   \hscodestyle
   \let\\=\@normalcr
   \(\parray}%
  {\endparray\)%
   \hsnewpar\belowdisplayskip
   \ignorespacesafterend}

\newcommand{\arrayhs}{\sethscode{arrayhscode}}

% The mathhscode environment also makes use of polytable's parray 
% environment. It is supposed to be used only inside math mode 
% (I used it to typeset the type rules in my thesis).

\newenvironment{mathhscode}%
  {\parray}{\endparray}

\newcommand{\mathhs}{\sethscode{mathhscode}}

% texths is similar to mathhs, but works in text mode.

\newenvironment{texthscode}%
  {\(\parray}{\endparray\)}

\newcommand{\texths}{\sethscode{texthscode}}

% The framed environment places code in a framed box.

\def\codeframewidth{\arrayrulewidth}
\RequirePackage{calc}

\newenvironment{framedhscode}%
  {\parskip=\abovedisplayskip\par\noindent
   \hscodestyle
   \arrayrulewidth=\codeframewidth
   \tabular{@{}|p{\linewidth-2\arraycolsep-2\arrayrulewidth-2pt}|@{}}%
   \hline\framedhslinecorrect\\{-1.5ex}%
   \let\endoflinesave=\\
   \let\\=\@normalcr
   \(\pboxed}%
  {\endpboxed\)%
   \framedhslinecorrect\endoflinesave{.5ex}\hline
   \endtabular
   \parskip=\belowdisplayskip\par\noindent
   \ignorespacesafterend}

\newcommand{\framedhslinecorrect}[2]%
  {#1[#2]}

\newcommand{\framedhs}{\sethscode{framedhscode}}

% The inlinehscode environment is an experimental environment
% that can be used to typeset displayed code inline.

\newenvironment{inlinehscode}%
  {\(\def\column##1##2{}%
   \let\>\undefined\let\<\undefined\let\\\undefined
   \newcommand\>[1][]{}\newcommand\<[1][]{}\newcommand\\[1][]{}%
   \def\fromto##1##2##3{##3}%
   \def\nextline{}}{\) }%

\newcommand{\inlinehs}{\sethscode{inlinehscode}}

% The joincode environment is a separate environment that
% can be used to surround and thereby connect multiple code
% blocks.

\newenvironment{joincode}%
  {\let\orighscode=\hscode
   \let\origendhscode=\endhscode
   \def\endhscode{\def\hscode{\endgroup\def\@currenvir{hscode}\\}\begingroup}
   %\let\SaveRestoreHook=\empty
   %\let\ColumnHook=\empty
   %\let\resethooks=\empty
   \orighscode\def\hscode{\endgroup\def\@currenvir{hscode}}}%
  {\origendhscode
   \global\let\hscode=\orighscode
   \global\let\endhscode=\origendhscode}%

\makeatother
\EndFmtInput
%
\newcommand{\lempredstable}{
  \begin{lemma}[Conditions are stable under substitution] \
    \label{aplem:pred-stable}
    \begin{enumerate}
    \item 
      Let $\rulet_1, \rulet_2$ be a rule type and $\theta$ be a type
      substitution. \\
      If $\nonoverlap(\rulet_1, \rulet_2)$, 
      then $\nonoverlap(\theta \rulet_1, \theta \rulet_2)$.
    \item 
      Let $\rulesetvar_1$ and $\rulesetvar_2$ be a context and $\theta$
      be a type substitution. \\
      If $\disjoint(\rulesetvar_1, \rulesetvar_2)$,
      then $\disjoint(\theta \rulesetvar_1, \theta \rulesetvar_2)$.
    \item 
      Let $\env$ be a type environment, $\type$ be a type and $\theta$
      be a type substitution. \\
      If $\coherent(\env, \type)$, 
      then $\coherent(\theta \env, \theta \type)$.
    \item 
      Let $e_1, \dots, e_n$ be a expression, 
      $v_1, \dots, v_n$ be a value, 
      $\rulet_1, \dots, \rulet_n$ be a rule type and 
      $\theta$ be a type substitution.
      \begin{enumerate}
      \item 
        If $\distinctwith(\rulesetexp)$, then 
        $\distinctwith(
        \overline{\relation{\theta e}{\theta \rulet}})$.
      \item 
        If $\distinctrs(\rulepgm)$, 
        then $\distinctrs(\grulepgm{\theta \rulet}{\theta v})$. 
      \item 
        If $\distinctctx(\rulesetvar)$, 
        then $\distinctctx(\theta \rulesetvar)$. 
      \end{enumerate}
    \end{enumerate}
  \end{lemma}
}

\newcommand{\lemresstable}[1]{
  \begin{lemma}[Static resolution is stable under substitution]
    \label{#1}
    Let $\env$ be a type environment, $\rulet$ be a rule type and
    $\theta$ be a type substitution. If $\env \vturns \rulet$, then
    $\theta \env \vturns \theta \rulet$.
  \end{lemma}
}

\newcommand{\lemtystable}[1]{
  \begin{lemma}[Expression typing is stable under substitution]
    \label{#1}
    Let $\env$ be a type environment, $e$ be an expression, $\type$
    be a type and $\theta$ be a type substitution. If $\env \turns
    \relation{e}{\type}$, then
    $\theta \env \turns \relation{\theta e}{\theta \type}$.
  \end{lemma}
}

\newcommand{\lemsemstable}[1]{
  \begin{lemma}[Semantic typing is stable under substitution]
    \label{#1}
    Let $\tstate$ be an environment, $\env$ be a type environment,
    $\rulepgmvar$ be a rule set, $\rulesetvar$ be a context, $v$
    be a value, $\rulet$ be a rule type and $\theta$ be a
    substitution. Then, 
    \begin{itemize}
    \item 
      If $\vtyping \relation{\tstate}{\env}$, then for all $\theta$, 
      $\vtyping \relation{\theta \tstate}{\theta \env}$;
    \item 
      If $\vtyping \relation{\rulepgmvar}{\rulesetvar}$, then for all
      $\theta$, $\vtyping \relation{\theta \rulepgmvar}{\theta
        \rulesetvar}$;
    \item 
      If $\vtyping \relation{v}{\tau}$, then for all $\theta$,
      $\vtyping \relation{\theta v}{\theta \tau}$.
    \end{itemize}
  \end{lemma}
}

\newcommand{\lemlookup}[1]{
  \begin{lemma}
    \label{#1}
    Let $\tstate$ be an environment, $\env$ be a type environment, and
    $\type$ be a type such that $\vtyping \relation{\tstate}{\env}$.
    If $\lookup{\env}{\type} = \rulet$, then $\lookup{\tstate}{\type}
    = v$ and $\vtyping \relation{v}{\rulet}$.
  \end{lemma}
}

\newcommand{\lemrespreserve}[1]{
  \begin{lemma}[Preservation of static resolution]
    \label{#1}
    Let $\tstate$ be environment, $\env$ be type environment, and
    $\rulet$ be a rule type such that $\rulet$ is unambiguous, $\env
    \vturns \rulet$ and $\vtyping \relation{\tstate}{\env}$. If
    $\tstate \vturns \rulet \eval v$, then $\vtyping
    \relation{v}{\rulet}$.
  \end{lemma}
}

\newcommand{\lemtypreserve}[1]{
  \begin{lemma}[Preservation]
    \label{#1}
    Let $\env$ be an environment, $\tstate$ be a type environment, $e$
    be a expression and $\type$ be a type such that $\tstate \turns
    \relation{e}{\type}$ and $\vtyping \relation{\tstate}{\env}$. If
    $\env \turns e \eval v$, then$\vtyping \relation{v}{\type}$.
  \end{lemma}
}

\newcommand{\thmsoundness}{
  \begin{theorem}[Type Soundness]
    Let $e$ be an expression and $\type$ be a type such that $\epsilon
    \turns \relation{e}{\type}$. If $\epsilon \turns e \eval v$, then
    $\vtyping \relation{v}{\type}$.
  \end{theorem}
}

\newcommand{\lemtrestypreserve}[1]{
  \begin{lemma}[\TrRes{} preserves type]
    \label{#1}
    Let $\rulet$ be a rule type, $\denv$ be a
    translation environment and $E$ be a System F expression. If
    $\denv \vturns \rulet \leadsto E$, then $|\denv| \turns
    \relation{E}{|\rulet|}$.
  \end{lemma}
}

\newcommand{\lemtranstypreserve}[1]{
 \begin{lemma}[Translation rules preserve type]
    \label{#1}
   Let e be a $\ourlang$ expression, $\type$ be a type, $\denv$ be a
   translation environment and $E$ be a System F expression. If $\denv
   \turns \relation{e}{\type} \leadsto E$, then $|\denv| \turns
   \relation{E}{|\rulet|}$.
 \end{lemma}
}

\newcommand{\thmtranstypreserve}{
  \begin{theorem}[Type-preserving translation]\label{thm:type:preservation} Let $e$ be a $\ourlang$
    expression, $\rho$ be a type and $E$ be a System F expression. If
    $\epsilon\mid\epsilon \turns \relation{e}{\rho} \leadsto E$, then $\epsilon \turns
    \relation{E}{|\rho|}$.
  \end{theorem}
}

\newcommand{\thmtranssempreserve}{
  \begin{conjecture}[Semantic-preserving translation]
    Let $e$ be a $\ourlang$ expression, $\type$ be a type and $E$ be a
    System F expression such that $\epsilon \turns \relation{e}{\type}
    \leadsto E$. If $\epsilon \turns E \eval V$, then $\epsilon \turns e
    \eval v$ where $\turns v \leadsto V$.
  \end{conjecture}
}

%%%%%%%%%%%%%%%%%%%%%%%%%%%%%%%%%%%%%%%%%%%%%%%%%%%%%%%%%%%%%%%%%%%%%%
\newcommand{\defunrelated}{
  \begin{definition}
    \[
    \myirule
    {
      \unify{\tau_1}{\tau_2}{\overline{\alpha_1},\overline{\alpha_2}} 
      = {\tt error}
    }
    {
      \unrelated
      (\rulesch{\alpha_1}{\pi_1}{\tau_1},
      \rulesch{\alpha_2}{\pi_2}{\tau_2})
    }
    \]
  \end{definition}
}

\newcommand{\lemunrelated}{
  \begin{lemma}
    \[
    \unrelated(\rho_1,\rho_2) 
    \Leftrightarrow \neg{\overlap}(\rho_1,\rho_2)
    \]
  \end{lemma}
}

\newcommand{\defresolvable}{
  \begin{definition}[Resolvability Check] \ 
    \begin{enumerate}
    \item 
      Let $P = \{ \rulet \in \env \mid \env(\rhs{\rulet}) =
      \rulet \}$ bet the set of non-occluded rule types.
    \item
      Consider all pairs of rule types $\rulet_1,\rulet_2 \in P$.
      \begin{enumerate}
      \item 
        Let $\theta$ be the most general unifier of $\rhs{\rulet_1}$ and $\rhs{\rulet_2}$
        with respect to $\ftv{\env}$ and $\qtv{\rulet_1}$. 
        If $\theta$ does not exist, ignore the pair $\rulet_1,\rulet_2$.
      \item Consider all substitutions $\theta'$ such that $\exists \theta''. \theta'' \cdot \theta' = \theta$,
        and $\theta'(\ftv{\env}) = \theta(\ftv{\env})$.
      \item Consider all $\pi \subseteq \lhs{\rulet_1}$.
      \item Check that, if
        $\env \vturns \theta'(\pi \Rightarrow \rhs{\rulet_1})$, then
        $\env \vturns \theta(\pi \Rightarrow \rhs{\rulet_2})$.
        If not, report failure of resolvability.
      \end{enumerate}
    \end{enumerate}
  \end{definition}
}

\newcommand{\thmalgoterm}{
  \begin{theorem}
    The algorithm terminates.
  \end{theorem}
}

\newcommand{\thmalgosound}{
  \begin{theorem}
    If the algorithm accepts $\env$, then $\welldefined(\env)$ holds.
  \end{theorem}
}

% \newcommand{\occ}[2]{\mathit{occ}_{#1}(#2)}
% \newcommand{\tnorm}[1]{|#1|}

\newcommand{\defterm}{
  \begin{definition}[Termination Condition]
    An implicit environment $\env$ satisfies the condition, denoted $\mathit{term}(\env)$, iff $\mathit{term}(\rho)$ for every $\rho = (\rulesch{\alpha}{\bar{\rho}}{\tau}) \in \mathit{dom}(\env)$, where:
    \begin{eqnarray*}
      \mathit{term}(\rho) 
      & \stackrel{\mathit{def}}{\Leftrightarrow} 
      & \occ{\alpha}{\tau'}\leq\occ{\alpha}{\tau} \\
      & & ~~~(\forall (\rulesch{\alpha'}{\bar{\rho}'}{\tau'}) \in \bar{\rho},\forall \alpha \in \mathit{ftv}(\tau,\tau_i)) \setminus \vec{\alpha}') \\
      & \wedge & \tnorm{\tau_i} < \tnorm{\tau}~~~(\forall \tau_i \in \bar{\rho}) \\
      & \wedge & \mathit{term}(\rho')~~~(\forall \rho' \in \bar{\rho})
    \end{eqnarray*}
    where
    \begin{eqnarray*}
      \occ{\alpha}{\tyint} & = & 0 \\
      \occ{\alpha}{\alpha} & = & 1 \\
      \occ{\alpha}{\alpha'} & = & 0\hspace{3cm}(\alpha \neq \alpha') \\
      \occ{\alpha}{\type_1 \rightarrow \type_2} & = & \occ{\alpha}{\type_1} + \occ{\alpha}{\type_2} \\
      \occ{\alpha}{\rulesch{\alpha}{\bar{\rho}}{\tau}} & = &  \occ{\alpha}{\tau} + \sum_{\rho\in\bar{\rho}}\occ{\alpha}{\rho} \\
      \tnorm{\tyint} & = & 1 \\
      \tnorm{\alpha} & = & 1 \\
      \tnorm{\type_1 \rightarrow \type_2} & = & 1 + \tnorm{\type_1} + \tnorm{\type_2} \\
      \tnorm{\rulesch{\alpha}{\bar{\rho}}{\tau}} & = &  1 + \tnorm{\tau} + \sum_{\rho\in\bar{\rho}}\tnorm{\rho}.
    \end{eqnarray*} 
  \end{definition}
}

\newcommand{\thmresterm}{
  \begin{theorem}[Terminating Resolution]
    For every implicit environment $\env$ such that $\mathit{term}(\env)$, and for every $\rulet$,
    all derivations of $\env \vturns \rulet$ are finite.
  \end{theorem}
}

%%% Local Variables: 
%%% mode: latex
%%% TeX-master: "Main"
%%% End: 


% \newenvironment{example}{\par\noindent\textit{Example}\quad}{}

% Metadata Information
%\acmVolume{1}
%\acmNumber{1}
%\acmArticle{1}
%\acmYear{2012}
%\acmMonth{1}

% Document starts
\begin{document}

% Page heads
\markboth{B. Oliveira et al.}{The Implicit Calculus: A New Foundation for Generic Programming}

% Title portion
\title{Deterministic Coherent Implicity}
\bruno{I find the proposed title a bit weird. I think I would prefers: 
``Coherent Implicits'' or ``Deterministic and coherent implicits''.
Perhaps Phillip has a cunning suggestion for the title?}
\author{Ommitted for anonymous submission}

\begin{abstract}
Implicit Progamming (IP) mechanisms infer values by a type-directed
resolution process, making programs more compact and easier to read.
Examples of IP mechanisms include Haskell's type classes, Scala's
implicits, Agda's instance arguments, Coq's type classes, and Rust's
traits.  The design of IP mechanism has led to heated debate:
proponents of one school argue the desirability of coherence, ensuring
each implicit has a unique resolution; while proponents of another
school argue for the power and flexibility of local scoping or
overlapping instances.  The current state-of-affairs seems to indicate
the two goals are at odds with one another, and cannot easily be
reconciled.

This paper presents \name, the Calculus Of CoHerent ImplicitS,
an improved variant of the implicit calculus that offers flexibility
while preserving coherence and avoiding ambiguity.  \name supports local scoping,
overlapping instances, first-class instances, and higher-order
rules, while remaining type safe and coherent.

\name has a compact formulation.
We introduce a logical formulation of how to resolve implicits, which
is simple but ambiguous and incoherent, and a second formulation,
which is less simple but unambiguous and coherent.  Every resolution
of the second formulation is also a resolution of the first, but not
conversely.  Parts of the second formulation bear a close resemblance
to a standard technique for proof search called focussing.

%%%%%%%%%%%%%%%%%%%%%%%%%%%%%%%%%%%%%%%%%%%%%%%%%%%%%%%%%%%%%%%%%%%%%%%%

%% Haskell type classes are a highly successful mechanism for
%% type-directed overloading. Greatly inspired by type classes, 
%% many other languages (such as Scala, Agda,
%% Coq and Rust) developed their own mechanisms 
%% for type-directed overloaded. However, 

%% Many other languages (such as Scala, Agda,
%% Coq)\bruno{others?} include mechanisms inspired by type classes, but
%% with some significant differences.

%% \emph{Generic programming} (GP) is an increasingly important trend in
%% programming languages. Well-known GP mechanisms, such as type classes
%% and the C++0x concepts proposal, usually combine two features: 1) a special type of
%% interfaces; and 2) \emph{implicit instantiation} of implementations of
%% those interfaces.

%% Scala \emph{implicits} are a GP language mechanism, inspired by type
%% classes, that break with the tradition of coupling
%% implicit instantiation with a special type of interface. Instead,
%% implicits provide only implicit instantiation, which is generalized to
%% work for \emph{any types}. 
%% %%In particular, implicit instantiation works
%% %%with any interface types in the language. 
%% %% and those interface types 
%% %%can be used to model concepts. 
%% Scala implicits turn out to be quite
%% powerful and useful to address many limitations that show up in other
%% GP mechanisms.

%% This paper synthesizes the key ideas of implicits formally in a minimal
%% and general core calculus called the implicit calculus (\name),
%% and it shows how to build source languages supporting implicit
%% instantiation on top of it. A novelty of the calculus is its support
%% for \emph{partial resolution} and \emph{higher-order rules} (a feature 
%% that has been proposed before, but was never formalized or implemented).
%% Ultimately, the implicit calculus provides a formal model of implicits, 
%% which can be used by language designers to 
%% study and inform implementations of similar mechanisms in their own languages.

%%%%%%%%%%%%%%%%%%%%%%%%%%%%%%%%%%%%%%%%%%%%%%%%%%%%%%%%%%%%%%%%%%%%%%%%

%%The calculus is relevant because it offers an interesting
%% because it provides a simple model 

%% This paper presents a core calculus generic programming

%%Our paper addresses this lack of formalization; it paper presents a general,
%%yet minimal calculus of implicits. The calculus offers the basic mechanisms for
%%type-directed resolution and scoping of implicits. These mechanisms are the key
%%ingredients for modeling powerful forms of implicit instantiation for source
%%languages. A core feature of the calculus is the fact that resolution does not
%%require a special concept-like interface, but works for any type.

%%We present the syntax, operational semantics, type system, an efficient
%%implementation through a type-directed translation, as well as correctness
%%results. Finally, we show how generic programming constructs are translated
%%into our calculus.

%%\paragraph{Version 2:}
%%Implicit programming that infers values by using type
%%information has proven its benefits in practice. However, 
%%years of experience have exposed several limitations in its current practice.
%%For example, Haskell's type classes, which are the oldest and most
%%prominent form of implicit programming have several
%%limitations of its original design: no
%%ability to control the \emph{scoping} of rules (instances); the \emph{second
%%  class} nature of types classes compared to types; and type class
%%and type class rules being limited to {first-order} cases.

%%This paper presents a \textit{implicit calculus}: a powerful and expressive,
%%yet minimalistic, calculus that provides an implicit programming model
%%without the limitations of the current practice. The implicit calculus
%%offers control over scoping, does not suffer from the second class
%%nature of type classes, and incoorporates two features which
%%have been discussed in the literature but for which there is
%%currently no good solution: \emph{higher-order rules} and
%%\emph{coherent support for overlapping rules}.  We present the
%%syntax, operational semantics, type system, type-directed translation
%%into System F, as well as correctness results. Finally, we show how
%%Haskell and Scala's implicit programming constructs are translated
%%into our calculus.

%%%%%%%%%%%%%%%%%%%%%%%%%%%%%%%%%%%%%%%%%%%%%%%%%%%%%%%%%%%%%%%%%%%%%%%%

%% Implicit programming that infers values by using type
%% information has proven its benefits in practice. However, 
%% years of experience have exposed several limitations in its current practice.
%% For example, Haskell's type classes, which are the oldest and most
%% prominent form of implicit programming have several
%% limitations of its original design: no
%% ability to control the \emph{scoping} of instances; the \emph{second
%%   class} nature of types classes compared to types; and type class
%% constraints being limited to \emph{first-order predicates}.

%% This paper presents a \textit{implicit calculus}: a powerful and expressive,
%% yet minimalistic, calculus that provides an implicit programming model
%% without the limitations of the current practice. The implicit calculus
%% offers control over scoping, does not suffer from the second class
%% nature of type classes, and incoorporates two features which
%% have been discussed in the literature but for which there is
%% currently no good solution: \emph{higher-order predicates} and
%% \emph{coherent support for overlapping instances}.  We present the
%% syntax, operational semantics, type system, type-directed translation
%% into System F, as well as correctness results. Finally, we show how
%% Haskell and Scala's implicit programming constructs are translated
%% into our calculus.

%%%%%%%%%%%%%%%%%%%%%%%%%%%%%%%%%%%%%%%%%%%%%%%%%%%%%%%%%%%%%%%%%%%%%%%%

%%% Local Variables: 
%%% mode: latex
%%% TeX-master: "../Main"
%%% End: 

\end{abstract}

\begin{comment}
\category{D.3.2}{Programming Languages}   
                {Language Classifications}
                [Functional Languages, Object-Oriented Languages]
\category{F.3.3}{Logics and Meanings of Programs}   
                {Studies of Program Constructs}
                []

\terms{Languages}

\keywords{
Implicit parameters, type classes, C++ concepts, generic programming,
Haskell, Scala}

\acmformat{Oliveira, B. C. d. S., Schrijvers, T., Choi, W., Lee, W., Yi, K., Wadler, P.
201?. The implicit calculus: a new foundation for generic programming.}

\begin{bottomstuff}
Author's addresses: 
B. C. d. S. Oliveira, Department of Computer Science, Hong Kong University; 
T. Schrijvers, Department of Applied Mathematics and Computer Science, Ghent University;
W. Choi {and} W. Lee {and} K. Yi, \ldots, Seoul National University;
P. Wadler, \ldots, University of Edinburgh.
\end{bottomstuff}
\end{comment}

\maketitle



\newcommand{\Meta}[1]{{\it #1\/}}
\newcommand{\m}[1]{\ensuremath{#1}}

\newcommand{\qlam}[2]{\m{\lambda\{#1\}.#2}}
\newcommand{\qask}[1]{\m{\Meta{?}#1}}
\newcommand{\qapp}[2]{\m{(#1)\;\texttt{with}\;#2}}
\newcommand{\qlet}[2]{\texttt{implicit}\;#1\;\texttt{in}\;#2}
\newcommand{\qLam}[2]{\m{\Lambda#1.#2}} 
  
\newcommand{\ty}[1]{\Meta{#1}}
\newcommand{\tyInt}{\Meta{int}}
\newcommand{\tyBool}{\Meta{bool}}
\newcommand{\rulety}[2]{\m{\{#1\}\!\Rightarrow\!#2}}

\section{Introduction}
\label{sec:intro}

%%* Implicit programming
%%* Type classes, C++ concepts, Scala implicits, JavaGI
%%* resolution, type-based

Implicit programming infers values by using type information. It is
``implicit'' because we can omit expressions (e.g. those for function's
parameters) and instead implicitly rely on its type-directed inference
of necessary values. 
%mentioning a type does not explicitly pinpoint a
%value by an expression but just means a value of that type.
Implicit yet necessary values are either fetched by their types from
the current environment or constructed by type-directed rules. 

Implicit programming has proven its benefits in practice. 
Haskell's \emph{type classes}, Scala's \emph{implicits} , JavaGI, C++'s
\emph{concepts} are all implicit programming mechanisms. These
practices have popularized the benefits of implicit programming: it
significantly reduces the amount of tedious, explicitly passed
expressions, it can be used to model a {constrained polymorphism}, and
it offers a powerful and disciplined form of {overloading}. 

\paragraph{Implicit Programming}
We first introduce, in a general manner, implicit programming by
enumerating its essence without being constrained by existing
practice. For this purpose we will use our calculus notation. For the
readers who are familiar with implicit programming practice, we will
sometimes add corresponding examples in notation reminescent of the
practice.

We then expose problems of existing implicit programming practice 
from this general point of view toward implicit programming, and we 
summarize our contribution. 

\begin{itemize}
\item {Implicit programming} fetches values by types, not by names.
  For example, an increment function's appliction to an implicit
  argument would be:
\[
\texttt{inc}\; (\qask{\tyInt}).
\]
Function \texttt{inc} is applied to an implicit argument (we call ``implicit
query'') that queries ``$\qask{\tyInt}$'' by mentioning just its
type \tyInt.  The int-typed entry in the current implicit
environment is looked up  to provide an integer value. 

In practice, the implicit query ``$\qask{\tyInt}$'' can even be
omitted thanks to type inference. Our calculus makes implicit queries
always manifest in text. 

\item Implicit programming computes values, whose steps are prescribed
  by programmer-defined, type-directed value-resolution rules  (\'{a}
  la functions). Value-resolution rules (rule abstractions) define how
  to compute, from implicit arguments, the values of an intended
  type. For example, a rule that computes a pair of int and bool from
  implicit int and bool values would be:
\[
\qlam{\tyInt, \tyBool}
     {\texttt{(\qask{\tyInt}+1, not\;\qask{\tyBool})}}.
\]
We write the above rule's type as
$\rulety{\tyInt,\tyBool}{\tyInt\times\tyBool}$. 

Hence, where a value of $\tyInt\times\tyBool$ is needed (expressed by 
an implicit query $\qask{(\tyInt\times\tyBool)}$), 
the above rule can be used if int and bool
values are available in the current implicit environment. Under such
environment, the rule returns a pair of the int and bool values after 
respective increment and negation. 

Building implicit environment is by rule applications
(analogous to building environment by function applications).
In our notation, rule application is expressed as, for example:
\[
\qapp{\qlam{\tyInt, \tyBool}
      {\texttt{(\qask{\tyInt}+1, not\;\qask{\tyBool})}}
     }{\{\texttt{1},\texttt{true}\}}.
\]
Using a syntactic sugar, which is analogous to the \texttt{let}-expression, the
above application is expressed as:
\[
\qlet{\{\texttt{1},\texttt{true}\}}
     {\texttt{(\qask{\tyInt}+1, not\;\qask{\tyBool})}}
\]
which returns $(2,\Meta{false})$. 

\item Implicit programming with higher-order rules. For example, a rule
\[
\qlam{\tyInt,\rulety{\tyInt}{\tyInt\times\tyInt}}{\qask{(\tyInt\times\tyInt)}}
\]
of type
$\rulety{\tyInt,\rulety{\tyInt}{\tyInt\times\tyInt}}{\tyInt\times\tyInt}$ 
is for computing an integer pair given an integer and a rule to
compute an integer pair from an integer.

Hence, the following example will return $(3, 4)$:
\[
\qlet{\{\texttt{3},\qlam{\tyInt}{\texttt{(\qask{\tyInt},\qask{\tyInt}+1)}}\}}
     {\qask{(\tyInt\times\tyInt)}}.
\]
Resolving (computing a value for) the implicit query
$\qask{(\tyInt\times\tyInt)}$ follows 
a simple inference: the goal is to have a pair of integers, yet the current
environment has no such pair but has an integer $3$ and a rule 
$\qlam{\tyInt}{\texttt{(\qask{\tyInt},\qask{\tyInt}+1)}}$
to compute a pair from an integer. Hence we can apply the pair
construction rule to $3$ to construct the $(3,4)$ pair for the implicit query. 

\item Implicit programming with polymorphic rules. For example, rule 
\[
\qLam{\alpha}{\qlam{\alpha}{(\qask{\alpha},\qask{\alpha})}},
\]
whose type is $\forall\alpha.\rulety{\alpha}{\alpha\times\alpha}$, can be
instantiated to multiple rules of monomorphic types
\[
\rulety{\tyInt}{\tyInt\times\tyInt}, 
\rulety{\tyBool}{\tyBool\times\tyBool}, \cdots.
\]
Hence multiple monomorphic queries can be resolved by the same
rule. For example, the following expression returns
$((3,3),(\Meta{true},\Meta{true}))$: 
\[
\begin{array}{l}
\texttt{implicit}\\
\mbox{\ \ \ \ }\{\texttt{3},\;\texttt{true},\;
      \qLam{\alpha}{\qlam{\alpha}{(\qask{\alpha},\qask{\alpha})}}\}\\
\texttt{in}\\
\mbox{\ \ \ \ }
\texttt{(\qask{(\tyInt\times\tyInt)},\qask{(\tyBool\times\tyBool)})}.
\end{array}
\]

\item Implicit programming with higher-order and polymorphic
  rules in combination. For example, the following rule 
\[
\qlam{\tyInt,\forall\alpha.\rulety{\alpha}{\alpha\times\alpha}}
 {\qask{((\tyInt\times\tyInt)\times(\tyInt\times\tyInt))}}
\]
prescribes how to build a pair of integer pairs, inductively from an
integer value. Begining from an integer value, consecutively applying the 
argument rule of type
\[
\forall\alpha.\rulety{\alpha}{\alpha\times\alpha}
\]
twice: first to an integer, then to its result (an
integer pair) again.

That is, the following expressiong returns $((3,3),(3,3))$:
\[
\begin{array}{l}
\texttt{implicit}\\
\mbox{\ \ \ \ }\{\texttt{3},\;
      \qLam{\alpha}{\qlam{\alpha}{(\qask{\alpha},\qask{\alpha})}}\}\\
\texttt{in}\\
\mbox{\ \ \ \ }
\texttt{\qask{((\tyInt\times\tyInt)\times(\tyInt\times\tyInt))}}.
\end{array}
\]
The $(3,3)$ pair is from applying the rule 
$\qLam{\alpha}{\qlam{\alpha}{(\qask{\alpha},\qask{\alpha})}}$
to $3$, and the final answer $((3,3),(3,3))$ from applying the same
rule to $(3,3)$.

\item Locally and lexically scoped rules. Suppose the rule 
\[
\qlam{\tyInt}
 {\qlam{\tyBool, \rulety{\tyBool}{\tyInt}}
       {\qask{\tyInt}}
 }
\]
is applied to an integer and then to a boolean and a rule
of type $\rulety{\tyBool}{\tyInt}$. Resolving the implicit query
$\qask{\tyInt}$ always selects the lexically nearest possible
resolution.

That is, resolution of the implicit query $\qask{\tyInt}$ is not from
fetching the integer value but from applying to the boolean value the
rule that returns an integer from a boolean. Following example thus
returns $2$ not $1$:
\[
\begin{array}{l}
\texttt{implicit}\;\{\texttt{1}\}\;\texttt{in}\\
\mbox{\ \ \ }\texttt{implicit}\;\{\texttt{true},\;
\qlam{\tyBool}{\texttt{if}\;\qask{\tyBool}\;\texttt{then 2}}\}\;\texttt{in}\\
\mbox{\ \ \ \ \ \ }\qask{\tyInt}.
\end{array}
\]

\item Overlapping rules and coherency. Two rules overlap if their
  return types intersects, hence they can both be used 
  to resolve the same implicit query. 

  The coherency principle states
  under what conditions the behavior of resolution is consistently
  predictable in the presence of overlapping rules. 
  Without coherency, programming with implicit values is a tricky
  business, fragile and unpredictable. 

  The conherency: the most concrete resolution rule is
  always chosen modulo the lexical scoping. For example, consider the
  following code: 
\[
\begin{array}{l}
\texttt{let}\;\texttt{f}:\forall\beta.\beta\to\beta = \\
\mbox{\ \ \ }\texttt{implicit}\;
  \{\lambda x.x:\forall\alpha.\alpha\to\alpha\}\;\texttt{in}\\
\mbox{\ \ \ \ \ \ }\texttt{implicit}\;
   \{\lambda n.n+1:\tyInt\to\tyInt\}\;\texttt{in}\\
\mbox{\ \ \ \ \ \ \ \ \ }\qask{(\beta\to\beta)}\\
\texttt{in}
\mbox{\ \ \ }\texttt{f} [\tyInt] \texttt{1}\;\cdots\;
             \texttt{f} [\tyBool] \texttt{true}.
\end{array}
\]
The definition of \texttt{f} uses two nested scopes to introduce two
overlapping values in the implicit environment.
According to the consistency principle, resolving the 
implicit query $\qask{\beta\to\beta}$ is determined at \texttt{f}'s
instantiations: the first $f$ must be $\lambda
n.n+1$ and the second $f$ must be $\lambda x.x$. 
\end{itemize}

%This \emph{resolution} mechanism to automatically infer
%function arguments based on available type information.  User-supplied
%\emph{rules} determine which values resolution infers. 

\paragraph{Problem}
Current implicit programming practice (e.g. Haskell's type classes,
Scala's implicits, C++'s concepts) fails to support 
the above implicit programming features in full combination.

\begin{itemize}

\item Haskell {type classes}~\cite{adhoc}, which is the
most prominent implicit programming mechanism, explore
only a particular corner of the large design space of implicit
programming~\cite{designspace}. 

Haskell's resolution rules are global. Hence, there can only be a
single rule for any given
type~\cite{named_instance,overloadingCamarao,implicit_explicit,modular}. No local scoping of rules are available. 
This is an noticeable limitation that already received several proposals for its
fix: rules can be named~\cite{named_instance} or locally
scoped~\cite{overloadingCamarao,implicit_explicit,modular}.
% Kahl and Scheczyk~\cite{named_instance} proposed to address this
%problem with \emph{named instances}, offering the possibility to name
%instances and manually select instances for resolution using these
%names. Others have suggested mechanisms that allowed
%\emph{local scoping} of
%instances~\cite{overloadingCamarao,implicit_explicit,modular}, thus
%offering control over which instances are on scope rather than having
%a single global scope. 
%
Haskell can't have higher-order rules~\cite{restricted}, though their
need in practice is real~\cite{sybclass, derivable, scalageneric}.
The consistency is broken. For the above consistency example, Haskell
assigns $\lambda x.x$ to both calls to $f$, though $\lambda n.n+1$ is
more concrete evidence for $f\;1$.

%Finally, type class
%constraints consist, essentially, of first-order rules, %predicates on types,
%but for some applications \emph{higher-order rules} are %predicates} are
%needed~\cite{derivable,scalageneric}.

\item  Scala's \emph{implicits}~\cite{implicits}  mechanism is only informally
described~\cite{implicits,scala} and some of 
the design decisions are quite ad-hoc. In particular, 
the \textit{coherency} principle is not guaranteed.
Scala's \emph{implicits} allows locally scoped
rules but overlapping rules are not lexically resolved but uses a
distance measure that determines the closest rule in an ad-hoc
manner. Static type inference is not available. 

%differs from Haskell
%type classes in several ways and has its own limitations too.
%In Scala, any value can be made implicit: Scala does not
%segregate type classes and types like Haskell and, as a consequence it
%is possible to abstract over the types representing type
%classes. 
%Secondly, implicit values (which roughly correspond to type
%class instances) are \textit{locally scoped}. 
%%%This allows for the same
%%%type to resolve to different values at different points in the
%%%program.  
%Thirdly, the \textit{scope nesting} prioritizes
%implicits. It enables an important and much desired feature without
%satisfactory solution in the globally-scoped type class approach: how
%to override general rules with ad-hoc cases, known as
%\textit{overlapping rules}.  The Scala solution to overlapping rules is
%  to place the ad-hoc cases in a nested scope, prioritizing them over
%  the general rules in an outer scope. 

%% Finally, it is important to 
%% note that while implicits offer many advantages over type classes, 
%% they have significantly less support for type-inference.


\item Other implicit programming mechanisms sit in between Haskell
  type classes and Scala implicits in terms of design choices. For
  instance, many proposals for \emph{concepts}, from the (C++) generic
  programming community, share with type 
classes the segregation between (concept) interfaces and other types, but
emphasize local scoping like Scala implicits. Similarly, the proposal of Dreyer et al.~\cite{modular}
for combining ML-style modules and type classes distinguishes
modules (that model type classes) and other types, and uses local
scoping. Most of these proposals, however, do not allow overlapping rules nor
higher-order ones.

\end{itemize}

There is a need to shed more light on the implicit
programming design space from a formal point of view. The existing
literature only addresses features like nested scoping, 
overlapping rules and higher-order predicates independently, but 
it does not provide an account of the combination of these features.
Furthermore the important issue of coherency remains essentially unresolved. 

\paragraph{Contribution} 
Our contributions are as follows.
\begin{itemize}
\item We present an \emph{implicit calculus} and a polymorphic type
  system that supports all the
  above mentioned features of implicit  programming: 
our type-safe implicit calculus seamlessly integrates nested scoping,
overlapping rules and higher-order predicates, while at the same time
ensuring coherency in the presence of overlapping rules. 

\item Despite its expressivness, our implicit calculus is a 
minimal system, which makes it ideal for the formal study of
resolution and scoping issues that occur in implicit programming
mechanisms. Furthermore, unlike most existing approaches to implicit
programming, whose semantics are defined by translation, 
our implicit calculus' operational semantics enables a direct
soundness proof and programs can be understood independently of the
type system.  

\item We present a type-preserving translation from our calculus to
  System F, as a way to implement our calculus. 

\item Finally, we present a small, but realistic source language,
  built on top of our calculus, that captures the essence and, in some
  ways, goes beyond the Haskell and Scala implicit programming
  practice. We also sketch the type-directed translation from the
  source to implicit calculus and discuss design decisions related to
  type-inference.

\end{itemize}

%%At the same time, it strives to be
%%a minimal core calculus consisting of only three constructs:
%%1) a query operator, 
%%2) a mechanism for assuming rules in scope, and 
%%3) another mechanism to supply evidence for assumptions.

% simple core calculus that essentially models a query
% language, resembling a simple logic programming language.  It consists of only
% four basic constructs: a query operator, a mechanism for introducing
% facts\tom{rules is a better word, facts in LP are rules without LHS} in scope,
% and another mechanism for binding evidence to facts and eliminating facts. 
% The rule calculus integrates all 4 features discussed by us and, consequently,
% provides a good foundation for studying the issues of scoping and resolution in
% implicit programming mechanisms.

%Technical results

%%We present and compare two semantics for the implicit calculus: a type-directed
%%elaboration semantics and an operational semantics.  While the former is the
%%traditional approach, the latter is the first of its kind for implicit
%%programming: it enables a direct soundness proof and programs can be understood
%%independent of the type system. Moreover, because resolution happens at runtime,
%%a static type system for the operational semantics can accept more programs
%%than for the elaboration semantics, which performs all of resolution statically.

%%In the static type system, we pay particular attention to the different issues
%%that arise with \textit{overlapping rules}, especially with respect to coherency. 
%%Our setting of locally
%%scoped rules and the dynamic resolution in the operational semantic, allows us
%%to formulate alternative solutions for dealing with overlapping rules and
%%avoiding the associated pitfalls.


%%a situation where multiple rules
%%are eligible for resolving an implicit value. Such overlapping rules have shown
%%their use in Haskell and Scala, but
%%the semantic issues are essentially left unaddressed.  


% We present the type system, an operational semantics as well as 
% an elaboration semantics for the polymorphic rule calculus. The 
% operational semantics is proved sound with respect to the type 
% system, and we show that the elaboration semantics is type 
% preserving. Furthermore, both semantics are shown to have 
% the coherency property. Finally, we show that the operational semantics 
% proposed by us is a \emph{conservative extension} of the elaboration 
% semantics, allowing more programs to be accepted, while still preserving 
% coherency. 

% With respect to the operational semantics, it is worth remarking that
% it is the first operational semantics in the literature for an
% implicit programming mechanism. Traditionally, the semantics or
% implementation of implicit programming mechanisms has been through
% elaboration. The large majority of type-class related proposals, Scala
% implicits or proposals for \emph{concepts} in C++ is given a semantics
% in this way.  In this style of semantics there is a source language
% which is translated via a type-directed translation into a core
% language. As a result, soundness can only be proved indirectly, as a
% consequence of the type-preservation of the translation. A consequence 
% of the elaboration semantics is that it tightly couples the dynamic and
% static semantics and it is essentially not possible to understand the
% behavior of programs independently of the type system. Furthermore, 
% it leads to somewhat weaker metatheoretical results~\cite{}.

% Although our operational semantics requires type annotations in
% certain constructs to guide the dynamic execution of programs, 
% it is independent from the static semantics. The operational semantics 
% allows us to prove soundness in a direct style.  Furthermore, it
% offers an alternative to resolution, which is necessarily static 
% in the elaboration semantics. In contrast, our operational semantics 
% allows resolution to happen at run-time, which offers the possibility 
% to accept more programs. Indeed, as we shall see, the operational 
% semantics conservatively extends the elaboration semantics. 

%%We also sketch a simple, but realistic source language that translates to the
%%calculus in a type-directed manner. This source language relinquishes some
%%expressivity to suit type-inference. We discuss the remaining type inference
%%issues and relate them to the literature~\cite{outsidein}.
  


% With respect to related work, the most notable difference of the rule
% calculus is that it focuses on providing a core calculus that directly
% supports features related to implicit programming. Most previous work,
% has typically studied each feature in isolation and has been built
% directly on source languages or core languages that have no built-in
% constructs for supporting implicit programming. For example, with
% respect to local instances, \cite{} proposes an elegant design of type
% classes on top of ML-style modules, which allows instances to be
% selectively activated in a certain local scope. However, none of the
% other 3 features are addressed and the local instances proposed in
% that work cannot be nested (which avoids further issues with respect to
% overlapping instances). 

% \paragraph{Summary}
% The contributions of this paper are:
% \begin{itemize}
% 
% \item \textbf{The rule calculus:} A core calculus aimed at modeling 
% the scoping and resolution features of implicit programming
% mechanisms. 
% 
% \item \textbf{Operational Semantics and dynamic resolution:}
% 
% \item \textbf{Source language and discussion about type-inference:}
% 
% \end{itemize}
% Instead type class interfaces are encoded using
% standard object-oriented interfaces and type class instances correspond to
% implicit objects that implement such interfaces. 

% In comparison to Scala --- which supports a form of local and
% overlapping instances, as well as type classes as types --- the key
% difference is that there is no formalism that models the Scala implicits
% mechanism in the literature. The design of the implicits mechanism is
% only informally described in the Scala reference \cite{}, and in a
% recent paper~\cite{}. More technically, the support for nested local 
% instances is ad-hoc and the current version of Scala does not ensure 
% coherent selection of instances. Furthermore, higher-order predicates 
% are not supported and support for type-inference in Scala is 
% very limited.  
% Notably, Haskell type
% classes assume that instances are globally visible and that there is only one
% possible instance for each type. Furthermore, type classes and types are
% considered distinct entities, the former being predicates (or relations) on the
% latter.

% In particular, the
% following topics have been discussed before in the literature:
% 
% %% * Limitations of type class mechanisms: 
% %%  higher-order predicates, local scoping, first-class type classes, 
% %%  any values can be implicit, overlapping instances. 
% 
% \paragraph{Local scoping} With Haskell type classes instances are
%   globally scoped, which essentially means that there can be only 
%   one instance per type in a program. While there are merits in the
%   choice of global scoping (for example with respect to principal
%   types \cite{}), for certain applications\bruno{mention references,
%     generic programming?} being restricted to a
%   single instance is too limiting and better control over scoping 
%   is desirable. As such, it is not a surprise that several alternative design 
%   proposals include some notion of local scoping
%   \cite{}. \bruno{Want to talk about Scala 
%   here but Scala should first be introduced, come back to this, or see
%   whet. Maybe I can talk about modular type classes later.} 
%    
% \paragraph {Overlapping instances and coherency} Related to scoping is the issue
%   of overlapping instances. Because selection of instances is
%   type-based, there are two important questions related to the
%   selection of instances. Firstly, what should happen if two instances of 
%   the same type are in scope? Secondly, in polymorphic systems like
%   Haskell, what should happen if two instances are in scope but one 
%   is more specific than the other? With global instances the standard 
%   answer to the first question is that programs with two instances of 
%   the same type should be rejected, whereas it is a much debated
%   question of what to do with respect to the second question. With
%   local instances the issue of overlapping instances is even more
%   subtle because for both questions it is debatable what the most 
%   appropriate thing to do is.   
% 
%   Ultimately, the issue of overlapping instances is one of the most
%   controversial in the design of implicit programming
%   mechanisms. While certain implementations of Haskell offer some
%   support for overlapping instances, the semantic foundations of
%   overlapping instances are considered to be quite tricky and there is
%   very little work in that area. Notably a desirable property when
%   dealing with overlapping instances is \emph{coherency}.\bruno{Why 
%   is coherency desirable?}
%   Nevertheless, the experience with
%   Haskell libraries has shown that overlapping instances are useful in
%   practice.\bruno{Again, it may be worth mentioning Scala here,
%     because of the prioritized overlapping implicits, we may also 
%   want to point out that Scala implicits ignore the issue of coherency.}
% 
% \paragraph{Type classes as types} A primary characteristic in the 
% design of Haskell type classes is that regular types (such as
% integers, lists or records) and type classes are segregated. 
% Values of regular types are explicitly passed, and such types 
% can be abstracted through the use of polymorphism. However, 
% with type classes, the evidence (which corresponds to the values) 
% is implicitly passed and type classes cannot be abstracted. 
% While the Haskell design can be justified with the view that 
% type classes as predicates on types, the segregation between 
% type classes and regular types as been shown limiting for certain 
% applications. \bruno{Scala takes a different view}.  
% 
% \paragraph{Higher-order predicates} Finally, in Haskell, constraints 
% can only be first-order predicates. However, in practice the need 
% for higher-order predicates arises. For example, it is well-known
% that instances for datatypes involving parametrization on type constructors, 
% would benefit from higher-order predicates~\cite{}. Also some 
% sophisticated, yet widely used libraries such as the monad transformer 
% library (MTL), are limited in their design partly because of the lack 
% of higher-order predicates. As a result some ad-hoc tricks need to be 
% used to overcome such limitations\bruno{We probably need to
% substantiate this argument. Forward reference to an example in the paper?}. 
% 
% \subsection{The Rule Calculus}
%
% \bruno{What is resolution?}  The main contribution of this paper is
% the \textit{rule caculus}: a calculus aimed at the study of resolution and
% scoping issues that occur in implicit programming mechanisms. From the
% four topics discussed above --- local scoping, overlapping instances,
% type classes as types, and higher-order predicates --- the first 2
% items are related to scoping, whereas the last 2 are related to
% resolution. While, every topic has been discussed before in the
% literature, there are no formalisms capable of providing all the
% features at once (or even some combinations) in the literature. 
% While every feature has shown its use in practice,
% it is not know whether all features can be integrated in a single, coherent
% system. 
% 
% The rule calculus is a simple core calculus that essentially models a query
% language, resembling a simple logic programming language.  It consists of only
% four basic constructs: a query operator, a mechanism for introducing
% facts\tom{rules is a better word, facts in LP are rules without LHS} in scope,
% and another mechanism for binding evidence to facts and eliminating facts. The
% rule calculus integrates all 4 features discussed by us and, consequently,
% provides a good foundation for studying the issues of scoping and resolution in
% implicit programming mechanisms.
% 
% %Technical results
% 
% We present the type system, an operational semantics as well as 
% an elaboration semantics for the polymorphic rule calculus. The 
% operational semantics is proved sound with respect to the type 
% system, and we show that the elaboration semantics is type 
% preserving. Furthermore, both semantics are shown to have 
% the coherency property. Finally, we show that the operational semantics 
% proposed by us is a \emph{conservative extension} of the elaboration 
% semantics, allowing more programs to be accepted, while still preserving 
% coherency. 
% 
% With respect to the operational semantics, it is worth remarking that
% it is the first operational semantics in the literature for an
% implicit programming mechanism. Traditionally, the semantics or
% implementation of implicit programming mechanisms has been through
% elaboration. The large majority of type-class related proposals, Scala
% implicits or proposals for \emph{concepts} in C++ is given a semantics
% in this way.  In this style of semantics there is a source language
% which is translated via a type-directed translation into a core
% language. As a result, soundness can only be proved indirectly, as a
% consequence of the type-preservation of the translation. A consequence 
% of the elaboration semantics is that it tightly couples the dynamic and
% static semantics and it is essentially not possible to understand the
% behavior of programs independently of the type system. Furthermore, 
% it leads to somewhat weaker metatheoretical results~\cite{}.
% 
% Although our operational semantics requires type annotations in
% certain constructs to guide the dynamic execution of programs, 
% it is independent from the static semantics. The operational semantics 
% allows us to prove soundness in a direct style.  Furthermore, it
% offers an alternative to resolution, which is necessarily static 
% in the elaboration semantics. In contrast, our operational semantics 
% allows resolution to happen at run-time, which offers the possibility 
% to accept more programs. Indeed, as we shall see, the operational 
% semantics conservatively extends the elaboration semantics. 
% 
% To illustrate the usefulness of the rule calculus to model realistic
% source languages with implicit programming constructs, we also 
% develop a simple Haskell-like source language which supports 
% the 4 features. This source language can be translated into the rule 
% calculus via a type-directed translation, which is shown to be 
% type-preserving. In the translation the various source language 
% constructs are modeled by combining the rule calculus constructs 
% with the standard constructs of the simply-typed lambda calculus. 
% The resulting language is less expressive than the rule calculus
% itself but, on the other hand, is amenable to type-inference. 
% A discussion about type-inference for this language, as well as
% several examples from the literature are presented.
%   
% With respect to related work, the most notable difference of the rule
% calculus is that it focuses on providing a core calculus that directly
% supports features related to implicit programming. Most previous work,
% has typically studied each feature in isolation and has been built
% directly on source languages or core languages that have no built-in
% constructs for supporting implicit programming. For example, with
% respect to local instances, \cite{} proposes an elegant design of type
% classes on top of ML-style modules, which allows instances to be
% selectively activated in a certain local scope. However, none of the
% other 3 features are addressed and the local instances proposed in
% that work cannot be nested (which avoids further issues with respect to
% overlapping instances). 




% It is not a surprise that alternatives have been proposed. 

%Why is Scala interesting

%%While Scala implicits are implemented and widely used, there is a lot
%%less research exploring the area of the design space covered by Scala.


%%However, important areas of the design space related to scoping and
%%resolution of type class instances remain relatively unexplored.

%%As pointed out by several researchers

%%As an attempt to provide a coherent framework in
%%which many type class extensions can be explained, \cite{} proposed
%%that if type classes are viewed as a particular use mode of ML-modules
%%then many extensions naturally follow for free.

%%Multiple
%%parameter type classes, functional dependencies, associated types 
%%fall in this category.   

%%As noted by several researchers, some of 

%%a lot of the focus of research about type class

%%has been on the ``type classes'' 
%%  (interfaces/ group of methods) part; less has been done on 
%%  scoping and resolution.

%% Following up on the observation that type classes and ML-modules
%% are closely related~\cite{}, \cite{} proposed a module system in which
%% modules can optionally be implicitly passed.

%%* Lots of work on type classes (and also qualified types)
%%  - However, a lot of the focus has been on the ``type classes'' 
%%  (interfaces/ group of methods) part; less has been done on 
%%  scoping and resolution.

%%* While Scala implicits are implemented and widely used, 
%%they have not been formalized.

%%Our solution

%%* The rule calculus. Just 3 constructs that deal with: 
%%  - scoping of implicits (local/lexical/nested scoping/overlapping instances)
%%  - resolution (higher-order predicates/any values of any type) 

%%* Any values of any Types can be implicitly passed (serve as evidence)

%%* Higher-order predicates 

%%* Local Scoping 

%%* First-class type classes

%%* Overlapping instances
%%  + incoherence

%%Operational Semantics

%% * Operational Semantics
%%  - Independent of the static semantics, although types are used to
%%  guide resolution. 

%% * Dynamic Resolution/Static Resolution
%% - coherency   


% \subsection{Old Stuff}
% 
% A recent trend in many programming languages is support for
% \emph{implicit programming}.  \emph{Implicit programming} denotes a
% programming style where some values are automatically inferred by
% using available type information. Implicit programming is 
% useful because it allows implicitly passing many
% values that are tedious to explicitly pass it makes 
% the code more readable 
% 
% . because they simply follow 
% the type structure of another argument, or 
% 
% Haskell type classes~\cite{} are the
% most prominent example of implicit programming. With type classes,
% \emph{type-class implementations} are inferred while performing
% type-inference. For example, in Haskell, the type of the |sort|
% function is as follows:
% 
% > sort :: Ord a => [a] -> [a]
% 
% In this case |Ord a| is a \emph{type-class constraint}. This
% constraint is used to ensure that implementations of comparison
% functions for values of type |a| exist. Such implementations, 
% called type-class instances, are defined by the user 
% 
% In its essence, the type class
% |Ord| is just an interface specifying a number of methods for
% comparing values of some type |a|. 
% 
% The |Ord| type class
% can be implemented for various types such as integers or lists of some
% type |a|. When the |sort| function is called the right implementation
% of |Ord| is selected by the compiler based on the instantiation of the
% type variable |a|. For example, calls like:
% 
% > sort [1,5,3]
% > sort [[1,5],[4,3]]
% 
% describes a particular programming style
% in which certain values used by a program can be \emph{inferred} by
% exploiting type-inference\footnote{This should probably be generalized
% to entail selection based on types, rather than just type-inference}.
% The most prominent example of a mechanism that supports implicit
% programming is given by Haskell type classes. With type classes,
% \emph{type-class dictionaries} (which are a specific kind of records) 
% are inferred while performing type-inference. A canonical
% example is given by a sorting function for lists in Haskell. The type
% of the |sort| function is as follows:
% 
% Implicit programming is to call functions without giving actual
% arguments.
% 
% % Haskell type classes have been widely used and proven to be one
% % effective programming feature. It is originated by the requirement of
% % supporting ad-hoc polymorphism in Hindley-Milner type system. It turns
% % out that we can simulate the things that are considered to only be
% % possible with dependent types with the recent extension of functional
% % dependencies or type families.
% 
% % The key ingredients of Haskell type class are two things: overloading
% % and resolution, whose combination is the essence of what we call {\it
% %   implicit programming}. The use of overloaded entity raises {\it
% %   overloading constraints} which is reflected on the type of
% % expression. For the function with overloading constraints to be used
% % in the later call site, constraints should be {\it resolved} using
% % {\it resolution rules} registered by programmer.
% 
% % Inspired by Haskell type class, many systems with {\it implicit
% %   programming} feature have been proposed.
% 
% % The previous systems have explored the design space of overloading and
% % resolution. \cite{designspace} discusses many design considerations in
% % Haskell type classes. \cite{logic_over,theory_over} studied the
% % resolution. \cite{typeclasses_implicits} studied how Scala-style
% % overloading can support Haskell-style overloading. \cite{without}
% % studied type class-like system in untyped language, Scheme and
% % proposed predicate-based overloading scheme. \cite{modular} proposes
% % an interesting system which harmoniously embeds type class-like
% % feature into ML module system by adding additional usage mode of
% % modules.
% 
% % However, there has not been any work which proposes comprehensive
% % framework on implicit programming. Many systems only supports global
% % scope of resolution rules. Most of them uses
% % static resolution.
% 
% % Especially, the previous systems suffers problem in modular
% % compilation when resolution rules can overlap each
% % other~\cite{designspace}.
% 
% % We propose a general system that is expressive enough to encompass
% % current practices and proposals and solves the intricacy incurred by
% % the overlapping.
% 
% \paragraph{Contributions} Our contributions are as follows.
% 
% \begin{itemize}
% \item We propose a core calculus $\lambda_{\rho}$ which supports the
%   following features of implicit programming:
% 
%   \begin{itemize}
% 
%   \item Flexible implicit values: values of any types can be used as
%     implicit values. This is more general than Haskell type
%     classes~\cite{adhoc} and its extension to named
%     instances~\cite{named_instance, implicit_explicit} where only the
%     dictionaries can be passed implicitly.
% 
%     % Note that in
%     % Haskell type classes~\cite{adhoc} and its extension to named
%     % instances~\cite{named_instance, implicit_explicit}, only
%     % dictionaries can be passed to functions implicitly. This is
%     % generalized in ~\cite{modular} such that modules are used as
%     % implicit values, but this is still limited because modules are not
%     % first-class objects in their system.
% 
%   \item Lexical scoping: an implicit value has lexical scope where it
%     is used for resolution. Lexical scopes of implicit values can be
%     arbitrarily nested. Again, this is more general than the systems
%     with only global scope~\cite{adhoc, theory_over, snd_over,
%       logic_over}.
% 
%   \item Overlapping instances: two implicit instances can overlap each
%     other in the same implicit scope unless they make
%     ambiguity. Overlapping instances is regarded as a useful feature
%     but has not yet been handled properly~\cite{designspace}.
%     Especially, the combination of overlapping instances and lexical
%     scoping has not yet been discussed.
% 
%     % \item First-class instances: implicit instances can be freely
%     %   created, passed and registered to implicit scope at run
%     %   time. First-class instances are necessary for generic
%     %   programming~\cite{syb, sybclass}, so they have been partially
%     %   supported by Glasgow Haskell Compiler~\cite{ghc}. However, there
%     %   is no system that fully supports this.
% 
%     % \item High-order implicit constraints: as an additional benefit of
%     %   flexible implicit values, we can use types with higher-order
%     %   implicit constraints. 
% 
%   \end{itemize}
% 
%   These features have not yet been combined in any single system
%   proposed in the literature.
%   
% \item For our language, we provide two alternatives for resolving
%   implicit constraints; dynamic resolution, which always picks the
%   most specific instance for the given constraint but is not modular,
%   and static resolution, which is modular but fails to pick the most
%   specific one when there are overlapping instances. We explain in
%   detail about for which set of programs both resolution coincide and
%   for which they do not.
% 
% \item Based on our core calculus, we make a simple language that
%   supports all the features of core calculus with some restrictions
%   and still enjoys the benefits of type inference. There has been no
%   system that has all these features with type inference.
% 
% % \item High-order implicit constraints:
% 
% \end{itemize}

% \paragraph{Organization} Section \ref{sec:overview} is an overview of
% our calculus $\ourlang$ and shows how our calculus supports various
% features of implicit programming using examples. Section
% \ref{sec:notation} defines notations to be used in the rest of this
% paper. Section \ref{sec:calculus} presents the formal system of
% $\ourlang$; operational semantics, polymorphic type system and
% type-directed translation are explained. Section \ref{sec:resolution}
% shows a detailed study on resolution in our system. Section
% \ref{sec:example} gives an example Haskell-like programming language
% as an instance of our general framework. Section \ref{sec:discussion}
% discusses some issues. Section \ref{sec:related} presents our
% exhaustive study on the previous work and comparison results. Section
% \ref{sec:conclusion} concludes.


%%% Local Variables: 
%%% mode: latex
%%% TeX-master: "../Main"
%%% End: 


\input{src/Overview}

\input{src/Types}

\section{Type-Directed Translation to System F}
\label{sec:trans}

In this section we explain the dynamic semantics of $\name$ in terms
of System F's dynamic semantics, by means of a type-directed translation. 
This translation turns implicit contexts into explicit parameters and
statically resolves all queries, much like Wadler and Blott's dictionary
passing translation for type classes~\cite{adhoc}. 
The advantage of this approach is that we simultaneously provide a meaning to
well-typed $\name$ programs and an effective implementation that resolves
all queries statically.

The translation follows the type system presented in Section~\ref{sec:ourlang}.
The additional machinery that is necessary (on top of the type system)
corresponds to the grayed parts of Figures~\ref{fig:type}, \ref{fig:resolution1} and \ref{fig:resolution2}. 

%-------------------------------------------------------------------------------
\subsection{Type-Directed Translation}
Figure~\ref{fig:type} presents the translation rules that convert $\name$
expressions into ones of System F. 
The gray parts of the figure extend the type system with the necessary
information for the translation.

The syntax of System F is as follows: 
{\small
  \[ \begin{array}{llrl}
    \text{Types} & T & ::= & \alpha \mid T \arrow T 
    \mid \forall \alpha. T \\ 
    \text{Expressions} & E & ::=  & x \mid \lambda (x:T) . E \mid E\;E
    \mid \Lambda \alpha . E \mid E\;T 
  \end{array} \]}

The gray extension to the syntax of type environments annotates every
implicit rule type with explicit System F evidence in the form of a 
term variable $x$.

%- - - - - - - - - - - - - - - - - - - - - - - - - - - - - - - - - - - - - - - - 
\paragraph{Translation of Types}

The function $|\cdot|$ takes 
$\name$ types $\rulet$ to System F types T: 
\begin{equation*}
\begin{array}{rcl@{\hspace{1cm}}rcl}
|\alpha| & = & \alpha &
|\forall \alpha. \rulet| & = & \forall \alpha. |\rulet| \\
|\rulet_1 \arrow \rulet_2| & = & |\rulet_1| \arrow |\rulet_2| &
|\rulet_1 \iarrow \rulet_2| & = & |\rulet_1| \arrow |\rulet_2| %\\
\end{array}
\end{equation*}
Its reveals that implicit $\name$ arrows are translated to explicit System F function arrows.

%- - - - - - - - - - - - - - - - - - - - - - - - - - - - - - - - - - - - - - - - 
\paragraph{Translation of Terms}

The type-directed translation judgment, which extends the typing judgment, is 
\begin{center}
  $\tenv \turns \relation{e}{\rulet}~\gbox{\leadsto E}$
\end{center}
This judgment states that the translation of $\name$ expression $e$ with
type $\rulet$ is System~F expression $E$, with respect to type environment
$\tenv$.

Variables, lambda abstractions and applications
are translated straightforwardly. Perhaps the only noteworthy 
rule is \TyIAbs. This rule associates the type $\rulet_1$ with 
the fresh variable $x$ in the type environment. 
This creates the necessary evidence that can be used by resolutions 
in the body of the rule abstraction to construct System F terms of type $|\rulet_1|$. 

%- - - - - - - - - - - - - - - - - - - - - - - - - - - - - - - - - - - - - - - - 
\paragraph{Resolution}
The more interesting part of the translation happens when resolving queries.
Queries are translated by rule $\TyQuery$ using the auxiliary resolution
judgment $\ivturns$:
\begin{equation*}
\tenv \ivturns \rulet~\gbox{\leadsto E}
\end{equation*}
which is shown, in deterministic form, in Figure~\ref{fig:resolution2}.  The
translation builds a System F term as evidence for the resolution.  

The mechanism that builds evidence dualizes the process of peeling off
abstractions and universal quantifiers: Rule \mylabel{R-IAbs}~wraps a lambda
binder with a fresh variable $x$ around a System F expression $E$, which is
generated from the resolution for the head of the rule ($\rulet_2$). Similarly,
rule \mylabel{R-TAbs}~wraps a type lambda binder around the System F expression
resulting from the resolution of $\rulet$.

For simple types $\type$ rule \mylabel{R-Simp} delegates the work of
building evidence, when a matching rule $\rulet$ type is found in the
environment, to rule \mylabel{L-RuleMatch}. The evidence consists of two parts:
$E$ is the evidence of matching $\type$ against $\rulet$. This match contains
placeholders $\bar{x}$ for the contexts whose resolution is postponed by rule
\mylabel{M-IAbs}. It falls to rule \mylabel{L-RuleMatch} to perform these
postponed resolutions, obtain their evidence $\bar{E}$ and fill in the
placeholders.

%- - - - - - - - - - - - - - - - - - - - - - - - - - - - - - - - - - - - - - - - 
\paragraph{Meta-Theory} The type-directed translation of $\name$ to System F exhibits a number
of desirable properties.

\begin{theorem}[Type-preserving translation]\label{thm:type:preservation} Let $e$ be an $\name$
  expression, $\rulet$ be a type, $\tenv$ a type environment and $E$ be a System F expression. If
  $\tenv \turns \relation{e}{\rulet} \leadsto E$, then $|\tenv| \turns \relation{E}{|\rulet|}$.
\end{theorem}
Here we define the translation of the type environment form $\name$ to System F as:
\begin{equation*}
\begin{array}{rcl@{\hspace{2cm}}rcl}
|\epsilon| & = & \epsilon & |\tenv,\alpha| & = & |\tenv|, \alpha \\
|\tenv,x : \rulet| & = & |\tenv|, x : |\rulet| &
|\tenv, \rulet \leadsto x| & = & |\tenv|, x : |\rulet|
\end{array}
\end{equation*}

An important lemma in the theorem's proof is the type preservation of 
resolution.
\begin{lemma}[Type-Preserving Resolution]
Let $\tenv$ be a type environment, $\rulet$ be a type and E be a System F expression.
If $\tenv \vturns \rulet \leadsto E$, then $|\tenv| \vdash E : |\rulet|$.
\end{lemma}
% Both the theorem and the lemma are proven in Appendix~\ref{proof:preservation}.

Moreover, we can express three key properties of Figure~\ref{fig:resolution2}'s
definition of resolution in terms of the generated evidence.
\begin{lemma}[Determinacy]
The generated evidence of resolution is uniquely determined.
\[\forall \tenv, \rulet, E_1, E_2: \quad\quad \tenv \ivturns \rulet \leadsto E_1 ~\wedge~ \tenv \ivturns \rulet \leadsto E_2 \quad\Rightarrow\quad E_1 = E_2 \]
\end{lemma}
\begin{lemma}[Soundness]
Figure~\ref{fig:resolution2}'s definition of resolution is sound (but
incomplete) with respect to Figure~\ref{fig:resolution1}'s definition.
\[\forall \tenv, \rulet, E: \quad\quad \tenv \ivturns \rulet \leadsto E \quad\Rightarrow\quad \tenv \vturns \rulet \leadsto E \]
\end{lemma}
\begin{lemma}[Coherence]
Resolution is stable under substitution.
\[\forall \tenv,\alpha,\tenv',\sigma,\rulet, E: \quad\quad 
\tenv,\alpha,\tenv' \ivturns \rulet \leadsto E \quad\wedge\quad \tenv \vdash \sigma
\quad\Rightarrow\quad 
\tenv,\tenv'[\sigma/\alpha] \ivturns \rulet[\sigma/\alpha] \leadsto E[|\sigma|/\alpha] \]
\end{lemma}
\bruno{Not entirely sure that the naming for the lemmas matches with
  what's on the literature, as I argued in my email}

%-------------------------------------------------------------------------------
\subsection{Evidence Generation in the Algorithm}

The evidence generation in Figure~\ref{fig:algorithm} is largely similar to
that in the deterministic specification of resolution in
Figure~\ref{fig:resolution2}.
With the evidence we can state the correctness of the algorithm.

% The main difference, and complication, is due to the fact that the evidence for
% type instantiation and recursive resolution is needed before these operations
% actually take place, as the algorithm has to postpone them. For this reason, the
% algorithm first produces placeholders that are later substituted for the actual
% evidence.
% 
% The central relation is $\rulet; \bar{\rulet}; \bar{\alpha} \gbox{; \bar{\omega};
% E}\turns_{\mathit{match}} \tau \hookrightarrow \bar{\rulet}' \gbox{;
% \bar{\omega}'; E'}$. It captures the matching instantiation of context type
% $\rulet$ against simple type $\tau$.  The input evidence for $\rulet$ is $E$, and
% the output evidence for the instantiation is $E'$. The accumulating parameters
% $\bar{\alpha}$ and $\bar{\rulet}$ denote that the instantiation of type variables
% $\bar{\alpha}$ and the recursive resolution of $\bar{\rulet}$ have been
% postponed. We use the $\bar{\alpha}$ themselves as convenient placeholders for
% the instantiating types, and we use the synthetic $\bar{\omega}$ as placeholders for
% the evidence of the $\bar{\rulet}$. The rules \mylabel{MTC-Abs} and \mylabel{MTC-Abs} 
% introduce these two kinds of placeholders in the evidence. The former kind, $\bar{\alpha}$,
% are substituted in rule \mylabel{MTC-Simp} where the actual type instantiatons $\bar{\theta}$
% are computed. The latter kind, $\bar{\omega}$, are substituted later in rule \mylabel{Alg-Simp} where
% the recursive resolutions take place.

\begin{theorem}[Partial Correctness]
Let $\tenv$ be a type environment, $\rulet$ be a type and E be a System F expression.
Assume that $\epsilon \vdash_{\mathit{unamb}} \rulet$ and also $\forall \rulet_i \in \tenv: \epsilon \vdash_{\mathit{unamb}} \rulet_i$.
Then $\tenv \ivturns \rulet \leadsto E$ if and only if $\tenv \vdash_{\mathit{alg}} \rulet \leadsto E$,
provided that the algorithm terminates.
\end{theorem}

%-------------------------------------------------------------------------------
\subsection{Dynamic Semantics}
Finally, we define the dynamic semantics of $\name$ as the composition of
the type-directed translation and System F's dynamic semantics.  Following
Siek's notation~\cite{systemfg}, this dynamic semantics is:
\[ \mathit{eval}(e) = V \quad\quad \textit{where } \epsilon \turns \relation{e}{\rulet} \leadsto E \textit{ and } E \rightarrow^* V  \]
with $\rightarrow^*$ the reflexive, transitive closure of System F's standard single-step call-by-value reduction relation (see \cite[Chapter 23]{tapl}).

Now we can state the conventional type safety theorem for $\name$:
\begin{theorem}[Type Safety]
If $\epsilon \turns \relation{e}{\rulet}$, then $\mathit{eval}(e) = V$ for
some System F value $V$.
\end{theorem}
The proof follows trivially from Theorem~\ref{thm:type:preservation}.




%\input{src/SourceLang}

\input{src/Related}

\section{Conclusion}
\label{sec:conclusion}

Our main contribution is the development of the implicit
calculus $\ourlang$. This calculus isolates and formalizes the key
ideas of Scala implicits and provides a simple model for language designers 
interested in developing similar mechanisms for their own languages. 
In addition, $\ourlang$ supports higher-order rules and partial resolution, 
which add considerable expressiveness to the calculus.

Implicits provide an interesting alternative to conventional GP 
mechanisms like type classes or concepts. By decoupling resolution 
from a particular type of interfaces, implicits make resolution 
more powerful and general. Furthermore, this decoupling has other benefits too. 
For example, by modeling concept interfaces as conventional types, those interfaces can 
be abstracted as any other types, avoiding the issue of second class interfaces 
that arise with type classes or concepts. 

Ultimately, all the expressiveness offered by $\ourlang$
offers a wide-range of possibilities for new generic programming applications.

%%In particular, 
%%resolution for any types local and nested scoping for implicits 
 
%%features for
%%generic programming, nested scoping and resolution of any types.

%%We present a formal type system for the calculus and provide semantics via a
%%translation to System F. We prove that the translation preserves types and thus
%%establish type soundness for $\ourlang$.

%%The calculus provides a formal platform for the development of
%%a realistic source language. Our small source language already shows 
%%how to add implicit instantiation on top of the calculus. In further
%%work we intend to develop this into a full-fledged language.



% integrates key features in GP, while at the same
% time being minimalistic. The calculus supports nested scoping,
% overlapping rules, higher-order rules and rule resolutions. 
%
% and encodings of existing GP practice. 
% 
% In the calculus, two key features -- scoping and resolution -- of many
% GP mechanisms were put in the spotlight, and we
% offered answers to two challenging, and essentially unsolved issues in
% the literature: how to support coherence for overlapping rules in
% nested scoping; and how to support resolution with higher-order
% rules. Such properties of scoping and resolution have not received
% appropriate attention in the past. 
% 
% of GP, which has been an
% increasingly popular trend in programming languages such as Haskell, Scala,
% Java, C++ and Agda. 


%%Both of these shortcommings hinder the development of applications 
%%that rely on sophisticated implicit programming mechanisms, as
%%document previously in the literature~\cite{derivable,}.

%Future work includes the design of a source language with a powerful 
%modularity construct (which could be a form of modules, OO classes 
%or records with associated types) to model ``type-class'' style
%interfaces. Type inference and additional source-level constructs on top of
%$\ourlang$ also deserve further attention.

%%% Local Variables: 
%%% mode: latex
%%% TeX-master: "../Main"
%%% End: 




%\appendixhead{OLIVEIRA}

% Acknowledgments
\begin{acks}
We are grateful to Ben Delaware, Derek Dreyer, Jeremy Gibbons, Scott
Kilpatrick, eta Ziliani, the members of ROPAS and the
anonymous reviewers for their comments and suggestions.  This work was
partially supported by Korea Ministry of Education, Science and
Technology/Korea Science and Enginering Foundation's ERC grant
R11-2008-007-01002-0, Brain Korea 21, Mid-career Research Program 2010-0022061, and
by Singapore Ministry of Education research grant MOE2010-T2-2-073.
\end{acks}

% Bibliography
\bibliographystyle{ACM-Reference-Format}
\bibliography{papers}

% Appendix
\appendix
\section*{APPENDIX}
\setcounter{section}{1}



\figtwocol{fig:ftype}{System F Type System}{
\begin{center}
\framebox{
\begin{minipage}{\textwidth}
\bda{lc}
\multicolumn{2}{c}{\myruleform{\Gamma \turns T}} \\ \\
  (\texttt{F-WF-VarTy}) & 
\myirule{
           \alpha \in \Gamma
 }{
            \Gamma \turns \alpha
} \\ \\
  (\texttt{F-WF-FunTy}) & 
\myirule{
            \Gamma \turns T_1 \quad\quad \Gamma \turns T_2
 }{
            \Gamma \turns T_1 \arrow T_2
} \\ \\
  (\texttt{F-WF-AbsTy}) & 
\myirule{
            \Gamma, \alpha \turns T
 }{
            \Gamma \turns \forall \alpha.T
} \\ \\
\multicolumn{2}{c}{\myruleform{\Gamma \turns E : T}} \\ \\
  (\texttt{F-Var}) & 
\myirule{
           (x : T) \in \Gamma
 }{
            \Gamma \turns x : T
} \\ \\

  (\texttt{F-Abs}) & 
\myirule{
           \Gamma, x : T_1 \turns E : T_2 \quad\quad \Gamma \turns T_1
 }{
           \Gamma \turns \lambda x:T_1.E : T_1 \rightarrow T_2
} \\ \\

  (\texttt{F-App}) & 
\myirule{
  \Gamma \turns E_1 : T_2 \rightarrow T_1 \quad\quad
           \Gamma \turns E_2 : T_2
          }{
           \Gamma \turns E_1 \, E_2 : T_1
} \\ \\

  (\texttt{F-TApp}) & 
\myirule{
  \Gamma \turns E : \forall \alpha. T_2 \quad\quad \Gamma \turns T_1
           }{
            \Gamma \turns E \, T_1 : T_2[T_1/\alpha]
} \\ \\

  (\texttt{F-TAbs}) & 
\myirule{
   \Gamma, \alpha \turns E : T
            }{
             \Gamma \turns \Lambda \alpha.E : \forall \alpha. T 
} \\ \\
\eda
\end{minipage}
}
\end{center}
}

%###############################################################################
\section{Proofs}

Throughout the proofs we refer to the type system rules of System F listed
in Figure~\ref{fig:ftype}.

%-------------------------------------------------------------------------------
\subsection{Type Preservation}\label{proof:preservation}

Lemma~\ref{lemma:tp:0} states that the translation preserves the well-formedness of types. 

{\centering
\fbox{
\begin{minipage}{0.95\columnwidth}
\begin{lemma}\label{lemma:tp:0}
  If 
\begin{equation*}
    \tenv \turns \rulet
\end{equation*}
  then
\begin{equation*}
    |\tenv| \turns |\rulet|
\end{equation*}
\end{lemma}
\end{minipage}
}}

\begin{proof}
By structural induction on the expression and corresponding inference rule.
\begin{description}
\renewcommand{\itemsep}{10mm}
%===============================================================================
\item[\fbox{\texttt{(WF-VarTy)}}\quad$\tenv \turns \alpha$] \ \\
%===============================================================================
  It follows from the rule that $\alpha \in \tenv$. Hence, obviously $\alpha
  \in |\tenv|$. Finally, by rule \mylabel{F-WF-VarTy}, and taking into account
  that $|\alpha| = \alpha$, we conclude
\begin{equation*}
  |\tenv| \turns \alpha
\end{equation*}

%===============================================================================
\item[\fbox{\texttt{(WF-FunTy)}}\quad$\tenv \turns \rulet_1 \arrow \rulet_2$] \ \\
%===============================================================================
  It follows from the induction hypotheses and the hypotheses of the rule that
\begin{equation*}
  |\tenv| \turns |\rulet_1| \quad \wedge \quad |\tenv| \turns |\rulet_2|
\end{equation*}

  Hence, by rule \mylabel{F-WF-FunTy}, and taking into account that
  $|\rulet_1| \arrow |\rulet_2| = |\rulet_1 \arrow \rulet_2|$, we conclude that
\begin{equation*}
  |\tenv| \turns |\rulet_1 \arrow \rulet_2|
\end{equation*}

%===============================================================================
\item[\fbox{\texttt{(WF-RulTy)}}\quad$\tenv \turns \rulet_1 \iarrow \rulet_2$] \ \\
%===============================================================================
  It follows from the induction hypotheses and the hypotheses of the rule that
\begin{equation*}
  |\tenv| \turns |\rulet_1| \quad \wedge \quad |\tenv| \turns |\rulet_2|
\end{equation*}

  Hence, by rule \mylabel{F-WF-FunTy}, and taking into account that
  $|\rulet_1| \arrow |\rulet_2| = |\rulet_1 \iarrow \rulet_2|$, we conclude that
\begin{equation*}
  |\tenv| \turns |\rulet_1 \iarrow \rulet_2|
\end{equation*}

%===============================================================================
\item[\fbox{\texttt{(WF-AbsTy)}}\quad$\tenv \turns \forall\alpha.\rulet$] \ \\
%===============================================================================
  It follows from the induction hypothesis and the hypothesis of the rule that
\begin{equation*}
  |\tenv,\alpha| \turns |\rulet|
\end{equation*}
  As $|\tenv,\alpha| = |\tenv|,\alpha$, we can simplify this to
\begin{equation*}
  |\tenv|,\alpha \turns |\rulet|
\end{equation*}

  Hence, by rule \mylabel{F-WF-AbsTy}, and taking into account that
  $\forall\alpha.|\rulet| = |\forall\alpha.\rulet|$, we conclude that
\begin{equation*}
  |\tenv| \turns |\forall\alpha.\rulet|
\end{equation*}

\end{description}
\end{proof}


Lemma~\ref{lemma:tp:1} states that the translation of expressions to System F preserves
types. Its proof relies on Lemma~\ref{lemma:tp:2}, which states that the translation
of resolution preserves types.

{\centering
\fbox{
\begin{minipage}{0.95\columnwidth}
\begin{lemma}\label{lemma:tp:1}
  If 
\begin{equation*}
    \tenv \turns e : \rho \leadsto E
\end{equation*}
  then
\begin{equation*}
    |\tenv| \turns E : |\rho|
\end{equation*}
\end{lemma}
\end{minipage}
}}

\begin{proof}
By structural induction on the expression and corresponding inference rule.
\begin{description}
\renewcommand{\itemsep}{10mm}
%===============================================================================
\item[\fbox{\texttt{(Ty-Var)}}\quad$\tenv \turns x : \rulet \leadsto x$] \ \\
%===============================================================================

 It follows from \mylabel{Ty-Var} that 
\begin{equation*} 
    (x : \rulet) \in \tenv
\end{equation*} 

Based on the definition of $|\cdot|$  it follows 
\begin{equation*} 
   (x : |\rulet|) \in |\tenv| 
\end{equation*} 

Thus we have by (\texttt{F-Var}) that
\begin{equation*} 
   |\tenv| \turns x : |\rulet|
\end{equation*} 

%===============================================================================
\item[\fbox{\texttt{(Ty-Abs)}}\quad$\tenv \turns \lambda x:\rho_1.e : \rho_1 \rightarrow \rho_2 \leadsto \lambda x:|\rho_1|.E$] \ \\
%===============================================================================

  The first hypothesis of (\texttt{Ty-Abs}) is that
\begin{equation*} 
    \tenv, x : \rho_1 \turns e : \rho_2 \leadsto E
\end{equation*} 
  and thus by the induction hypothesis we have that
\begin{equation*} 
    |\tenv|, x : |\rho_1| \turns E : |\rho_2|
\end{equation*} 

  The second hypothesis of (\texttt{Ty-Abs}) is that
\begin{equation*} 
    \tenv \turns |\rho_1|
\end{equation*} 
  and thus by Lemma~\ref{lemma:tp:0} we have that
\begin{equation*} 
    |\tenv| \turns |\rho_1|
\end{equation*} 

  Hence, by \mylabel{F-Abs} we conclude 
\begin{equation*} 
    |\tenv| \turns \lambda x:|\rho_1|.E : |\rho_1 \rightarrow \rho_2|
\end{equation*} 

%===============================================================================
\item[\fbox{\texttt{(Ty-App)}}\quad$\tenv \turns e_1\,e_2 : \rulet_1 \leadsto E_1\,E_2$] \ \\
%===============================================================================

  By the induction hypothesis, we have:
\begin{equation*} 
   |\tenv| \turns E_1 : |\rulet_2 \rightarrow \rulet_1| \quad\wedge\quad |\tenv| \turns E_2 : |\rulet_2|
\end{equation*} 
  and, because $|\rulet_2 \arrow \rulet_1| = |\rulet_2| \arrow |\rulet_1|$, we can write the former as
\begin{equation*} 
   |\tenv| \turns E_1 : |\rulet_2| \arrow |\rulet_1|
\end{equation*} 

  Then it follows by (\texttt{F-App}) that
\begin{equation*} 
   |\tenv| \turns E_1\, E_2 : |\rulet_1|
\end{equation*} 

%===============================================================================
\item[\fbox{\texttt{(Ty-TAbs)}}\quad$\tenv \turns \Lambda \alpha.e : \forall \alpha. \rulet \leadsto \Lambda \alpha.E$]\ \\
%===============================================================================

  Based on (\texttt{Ty-TAbs}) and the induction hypothesis, we have
\begin{equation*} 
    |\tenv, \alpha| \turns E : |\rulet|
\end{equation*}

  Thus, based on (\texttt{F-TAbs}) and because $|\tenv,\alpha| = |\tenv|,\alpha$, we have
\begin{equation*} 
    |\tenv| \turns \Lambda \alpha. E : \forall \alpha.|\rulet|
\end{equation*}
  or, because $|\forall\alpha.\rulet|=\forall\alpha.|\rulet|$, we conclude
\begin{equation*} 
    |\tenv| \turns \Lambda \alpha. E : |\forall \alpha.\rulet|
\end{equation*}

%===============================================================================
\item[\fbox{\texttt{(Ty-TApp)}}]\quad$\tenv \turns e\,\rulet_1 : \rulet_2[\rulet_1/\alpha] \leadsto E\,|\rulet_1|$\ \\
%===============================================================================
 
  By the first hypothesis of the rule and the induction hypothesis of the
  lemma, it follows that
\begin{equation*} 
    |\tenv| \turns E : |\forall \alpha.\rulet_2|
\end{equation*} 
  From this we have by definition of $|\cdot|$
\begin{equation*} 
    |\tenv| \turns E : \forall{\alpha}.|\rulet_2|
\end{equation*} 

  By the second hypothesis of the rule and Lemma~\ref{lemma:tp:0} we also have
\begin{equation*} 
    |\tenv| \turns |\rulet_1|
\end{equation*} 

  It then follows from (\texttt{F-TApp}) that
\begin{equation*} 
    |\tenv| \turns E\,|\rulet_1| : |\rulet_2|[|\rulet_1|/\alpha]
\end{equation*} 
  This is easily seen to be equivalent to 
\begin{equation*} 
    |\tenv| \turns E\,|\rulet_1| : |\rulet_2[\rulet_1/\alpha]|
\end{equation*} 

%===============================================================================
\item[\fbox{\texttt{(Ty-IAbs)}}\quad$\tenv \turns \ilambda \rulet_1.e : \rulet_1 \iarrow \rulet_2 \leadsto \lambda x:|\rulet_1|.E$]\ \\
%===============================================================================

  Based on the first hypothesis of the rule and the induction hypothesis, we have
\begin{equation*} 
    |\tenv, \rulet_1 \leadsto x| \turns E : |\rulet_2|
\end{equation*}
  or, using the definition of $|\cdot|$,
\begin{equation*} 
    |\tenv|, x : |\rulet_1| \turns E : |\rulet_2|
\end{equation*}

  Based on the second hypothesis of the rule and Lemma~\ref{lemma:tp:0} we have
\begin{equation*} 
    |\tenv| \turns |\rulet_1|
\end{equation*}

  Thus, based on (\texttt{F-Abs}) we have
\begin{equation*} 
    |\tenv| \turns \lambda x : |\rulet_1|.E : |\rulet_1| \arrow |\rulet_2|
\end{equation*}
  or, using the definition of $|\cdot|$ again,
\begin{equation*} 
    |\tenv| \turns \lambda x : |\rulet_1|.E : |\rulet_1 \iarrow \rulet_2|
\end{equation*}
 
%===============================================================================
\item[\fbox{\texttt{(Ty-IApp)}}\quad$\tenv \turns e_1 \with e_2 : \rulet_1 \leadsto E_1\,E_2$] \ \\
%===============================================================================

  From the hypotheses of the rule and the induction hypothesis we have:
\begin{equation*}
    |\tenv| \turns E_1 : |\rulet_2 \iarrow \rulet_1| \quad\wedge\quad |\tenv| \turns E_2 : |\rulet_2|
\end{equation*}

  Based on the definition of $|\cdot|$, the first of these means 
\begin{equation*}
    |\tenv| \turns E_1 : |\rulet_2| \arrow |\rulet_1|
\end{equation*}

  Finally, based on (\texttt{F-App}), we know
\begin{equation*}
    |\tenv| \turns E_1\,E_2 : |\rulet_1|
\end{equation*}

%===============================================================================
\item[\fbox{\texttt{(Ty-Query)}}\quad$\tenv \turns ?\rulet : \rulet \leadsto E$] \ \\
%===============================================================================

  Based on the first hypothesis of the rule and Lemma~\ref{lemma:resolution} we know
\begin{equation*} 
    |\tenv| \turns E : |\rulet|
\end{equation*} 

\end{description}
\end{proof}

%###############################################################################
{\centering
\fbox{
\begin{minipage}{0.95\columnwidth}
\begin{lemma}\label{lemma:resolution}\label{lemma:tp:2}
  If 
\begin{equation*}
    \tenv \vdash_r \rulet \leadsto E
\end{equation*}
  and
\begin{equation*}
    \tenv \turns \rulet 
\end{equation*}
  then
\begin{equation*}
    |\tenv| \turns E : |\rulet|
\end{equation*}
\end{lemma}
\end{minipage}
}}

\begin{proof}

  By induction on the derivation.
\begin{description}
%===============================================================================
\item[\fbox{\texttt{(R-TAbs)}}\quad$\tenv \vdash_r \forall \alpha.\rulet \leadsto \Lambda \alpha.E$] \ \\
%===============================================================================

  From the hypothesis of the rule and the induction hypothesis, we have
\begin{equation*}
  |\tenv, \alpha| \vdash E : |\rulet|
\end{equation*}
  or alternatively, based on the definition of $|\cdot|$,
\begin{equation*}
  |\tenv|, \alpha \vdash E : |\rulet|
\end{equation*}
 
  Then, rule \mylabel{F-TAbs} allows us to conclude
\begin{equation*}
  |\tenv| \vdash \Lambda \alpha. E : \forall \alpha.|\rulet|
\end{equation*}
  or, again based on the definition of $|\cdot|$,
\begin{equation*}
  |\tenv| \vdash \Lambda \alpha. E : |\forall \alpha.\rulet|
\end{equation*}

%===============================================================================
\item[\fbox{\texttt{(R-TApp)}}]\quad$\tenv \vdash_r \rulet [\suty/\alpha] \leadsto E\,|\suty|$ \ \\
%===============================================================================

  From the first hypothesis of the rule and the induction hypothesis, we have
\begin{equation*}
  |\tenv| \vdash E : |\forall \alpha. \rulet|
\end{equation*}
  or alternatively, based on the definition of $|\cdot|$,
\begin{equation*}
  |\tenv| \vdash E : \forall \alpha. |\rulet|
\end{equation*}

  From the second hypothesis of the rule and Lemma~\ref{lemma:tp:0}, we have
\begin{equation*}
  |\tenv| \vdash |\suty|
\end{equation*}

  Then, rule \mylabel{F-TApp} allows us to conclude
\begin{equation*}
  |\tenv| \vdash E\,|\suty| : |\rulet|[|\suty|/\alpha]
\end{equation*}
  or, again based on the definition of $|\cdot|$,
\begin{equation*}
  |\tenv| \vdash E\,|\suty| : |\rulet[\suty/\alpha]|
\end{equation*}

%===============================================================================
\item[\fbox{\texttt{(R-IVar)}}]\quad$\tenv \vdash_r \rulet \leadsto x$ \ \\
%===============================================================================

  From the hypothesis of the rule and the definition of $|\cdot|$, we have
\begin{equation*}
  (x : |\rulet|) \in |\tenv|
\end{equation*}

  Thus, using rule \mylabel{F-Var}, we can conclude
\begin{equation*}
  |\tenv| \vdash x : |\rulet|
\end{equation*}

%===============================================================================
\item[\fbox{\texttt{(R-IAbs)}}]\quad$\tenv \vdash_r \rulet_1 \iarrow \rulet_2 \leadsto \lambda x:|\rulet_1|.E$ \ \\
%===============================================================================

  From the first hypothesis of the rule and the induction hypothesis, we have
\begin{equation*}
  |\tenv, \rulet_1 \leadsto x| \vdash E : |\rulet_2|
\end{equation*}
  or alternatively, based on the definition of $|\cdot|$,
\begin{equation*}
  |\tenv|, x : |\rulet_1| \vdash E : |\rulet_2|
\end{equation*}

  Then, rule \mylabel{F-Abs} allows us to conclude
\begin{equation*}
  |\tenv| \vdash \lambda x:|\rulet_1|.E : |\rulet_1| \arrow |\rulet_2|
\end{equation*}

  or, again based on the definition of $|\cdot|$,
\begin{equation*}
  |\tenv| \vdash \lambda x:|\rulet_1|.E : |\rulet_1 \iarrow \rulet_2|
\end{equation*}

%===============================================================================
\item[\fbox{\texttt{(R-IApp)}}]\quad$\tenv \vdash_r \rulet_2 \leadsto E_2\,E_1$ \ \\
%===============================================================================

  From the hypotheses of the rule and the induction hypothesis, we have
\begin{equation*}
  |\tenv| \vdash E_1 : |\rulet_1| \quad\wedge\quad
  |\tenv| \vdash E_2 : |\rulet_1 \iarrow \rulet_2|
\end{equation*}

  The second conjunct can be reformulated, based on the definition of $|\cdot|$, to
\begin{equation*}
  |\tenv| \vdash E_2 : |\rulet_1| \arrow |\rulet_2|
\end{equation*}

  Then, rule \mylabel{F-App} allows us to conclude
\begin{equation*}
  |\tenv| \vdash E_2\,E_1 : |\rulet_2|
\end{equation*}

\end{description}
\end{proof}


%-------------------------------------------------------------------------------
\subsection{Auxiliary Lemmas About Non-Determistic Resolution}

The non-deterministic resolution judgement enjoys a number of
typical binder-related properties.

The first lemma is the weakening lemma: that states that an extended context
preserves all the derivations of the original context.
%###############################################################################
{\centering
\fbox{
\begin{minipage}{0.95\columnwidth}
\begin{lemma}[Weakening]\label{lemma:weakening}
  If 
\begin{equation*}
  \tenv, \tenv' \vturns \rulet \leadsto E 
\end{equation*}
  then
\begin{equation*}
    \tenv, \tenv'', \tenv' \vturns \rulet \leadsto E
\end{equation*}
\end{lemma}
\end{minipage}
}}

\begin{proof}
The proof proceeds by straightfoward induction on the derivation of the hypothesis.
\end{proof}

The second lemma is the substitution lemma which states that
we can drop an axiom from the context if it is already implied
by the remainder of the context.

%###############################################################################
{\centering
\fbox{
\begin{minipage}{0.95\columnwidth}
\begin{lemma}[Substitution]\label{lemma:substitution}
  If 
\begin{equation*}
  \tenv, \rulet \leadsto x, \tenv' \vturns \rulet' \leadsto E'
\end{equation*}
  and
\begin{equation*}
  \tenv \vturns \rulet \leadsto E 
\end{equation*}
  then
\begin{equation*}
    \tenv, \tenv' \vturns \rulet' \leadsto E'[E/x]
\end{equation*}
\end{lemma}
\end{minipage}
}}

\begin{proof}
The proof proceeds by straightfoward induction on the derivation of the first
hypothesis.

The key case is the one for rule \mylabel{R-IVar} where $\rulet' = \rulet$ and
$E' = x$. In this case the second hypothesis gives us 
\begin{equation*}
\tenv \vturns \rulet \leadsto E
\end{equation*}
As $E = x[E/x]$, this also means
\begin{equation*}
\tenv \vturns \rulet \leadsto x[E/x]
\end{equation*}
Finally, we can apply the Weakening Lemma~\ref{lemma:weakening} to obtain the
desired result.
\begin{equation*}
\tenv, \tenv' \vturns \rulet \leadsto x[E/x]
\end{equation*}

All other cases are straightforward.
\end{proof}

%-------------------------------------------------------------------------------
\subsection{Soundness of Deterministic Resolution}

Lemma~\ref{lemma:s23:4} states that deterministic resolution is
sound with respect to non-deterministic resolution. 

%###############################################################################
{\centering
\fbox{
\begin{minipage}{0.95\columnwidth}
\begin{lemma}\label{lemma:s23:4}
  If 
\begin{equation*}
  \tenv \ivturns \rulet \leadsto E 
\end{equation*}
  then
\begin{equation*}
    \tenv \vturns \rulet \leadsto E
\end{equation*}
\end{lemma}
\end{minipage}
}}

\begin{proof}
The lemma immediately follows from Lemma~\ref{lemma:s23:3}.
\end{proof}

%###############################################################################
{\centering
\fbox{
\begin{minipage}{0.95\columnwidth}
\begin{lemma}\label{lemma:s23:3}
  If 
\begin{equation*}
  \bar{\alpha}; \tenv \ivturns \rulet \leadsto E 
\end{equation*}
  then
\begin{equation*}
    \tenv \vturns \rulet \leadsto E
\end{equation*}
\end{lemma}
\end{minipage}
}}

\begin{proof}
The proof proceeds by induction on the derivation.
\begin{description}
\setlength{\itemsep}{1em}
%===============================================================================
\item[\fbox{\texttt{(R-IAbs)}$^3$}]\quad$\bar{\alpha};\tenv \ivturns \rulet_1 \iarrow \rulet_2 \leadsto \lambda x:|\rulet_1|.E$ \ \\
%===============================================================================
  From the first assumption of rule \mylabel{R-IAbs}$^3$ and the induction hypothesis, we have
\begin{equation*}
  \tenv, x : \rulet_1 \vturns \rulet_2 \leadsto E
\end{equation*}

  Hence, from rule \mylabel{R-IAbs}$^2$ and the freshness condition on $x$ in
rule \mylabel{R-IAbs}$^3$ it follows that
\begin{equation*}
  \tenv \vturns \rulet_1 \iarrow \rulet_2 \leadsto \lambda x:|\rulet_1|.E
\end{equation*}

%===============================================================================
\item[\fbox{\texttt{(R-TAbs)}$^3$}]\quad$\bar{\alpha};\tenv \ivturns \forall \alpha.\rulet \leadsto \Lambda \alpha.E$ \ \\
%===============================================================================
  From the precondition of rule \mylabel{R-TAbs}$^3$ and the induction hypothesis, we have
\begin{equation*}
  \tenv,\alpha \vturns \rulet \leadsto E
\end{equation*}

  Hence, from rule \mylabel{R-TAbs}$^2$ it follows that
\begin{equation*}
  \tenv \vturns \forall \alpha.\rulet \leadsto \Lambda \alpha.E
\end{equation*}

%===============================================================================
\item[\fbox{\texttt{(R-Simp)}$^3$}]\quad$\bar{\alpha};\tenv \ivturns \tau \leadsto E$ \ \\
%===============================================================================

  From the precondition of the rule, Lemma~\ref{lemma:s23:2} and the simple fact that
  $\tenv \subseteq \tenv$, it follows that
\begin{equation*}
  \tenv \vturns \tau \leadsto E
\end{equation*}
\end{description}
\end{proof}


The above proof relies on the following auxiliary lemma for the 
resolution of simple types. The proof of this auxiliary lemma proceeds
by mutual induction with the proof of the main lemma.

%###############################################################################
{\centering
\fbox{
\begin{minipage}{0.95\columnwidth}
\begin{lemma}\label{lemma:s23:2}
  If 
\begin{equation*}
   \bar{\alpha}; \tenv; \tenv' \ivturns \tau \leadsto E
\end{equation*}
  and
\begin{equation*}
  \tenv' \subseteq \tenv 
\end{equation*}
  then
\begin{equation*}
  \tenv \vturns \tau \leadsto E
\end{equation*}
\end{lemma}
\end{minipage}
}}

\begin{proof}
The proof proceeds by induction on the derivation, mutually 
with the previous proof.
\begin{description}
\setlength{\itemsep}{1em}
%===============================================================================
\item[\fbox{\texttt{(L-RuleMatch)}}]\quad$\bar{\alpha}; \tenv; \tenv', \rulet \leadsto x \ivturns \type \leadsto E[\bar{E}/\bar{x}]$ \ \\
%===============================================================================
  The first assumption of the rule is
\begin{equation*}
  \tenv; \rulet \leadsto x \ivturns \type \leadsto E; \bar{E} \leadsto \bar{x}
\end{equation*}

  From the lemma's assumption $(\tenv', \rulet \leadsto x) \subseteq \tenv$ we conclude
  $(\rulet \leadsto x) \in \tenv$. Hence, by rule \mylabel{R-Simp}$^2$ we have
\begin{equation*}
  \tenv \vturns \rulet \leadsto x
\end{equation*}

  From the second precondition of the rule and the (mutual) induction hypothesis, we also have
\begin{equation*}
  \tenv \vturns \bar{\rulet} \leadsto \bar{E}
\end{equation*}

  The above three observations allow us to invoke the auxiliary Lemma~\ref{lemma:s23:1}
  and conclude
\begin{equation*}
  \tenv \vturns \tau \leadsto E[\bar{E}/\bar{x}]
\end{equation*}

%===============================================================================
\item[\fbox{\texttt{(L-Var), (L-TyVar), (L-RuleNoMatch)}}] \ \\
%===============================================================================
  Trivially by applying the induction hypothesis on the precondition of the rule.
\end{description}
\end{proof}

The above proof relies on the following auxiliary lemma.

%###############################################################################
{\centering
\fbox{
\begin{minipage}{0.95\columnwidth}
\begin{lemma}\label{lemma:s23:1}
  If 
\begin{equation*}
   \tenv; \rulet \leadsto E \ivturns \type \leadsto E'; \bar{\rulet} \leadsto \bar{x}
\end{equation*}
  and
\begin{equation*}
  \tenv \vturns \rulet \leadsto E
\end{equation*}
  and
\begin{equation*}
  \tenv \vturns \bar{\rulet} \leadsto \bar{E}
\end{equation*}
  then
\begin{equation*}
  \tenv \vturns \tau \leadsto E'[\bar{E}/\bar{x}]
\end{equation*}
\end{lemma}
\end{minipage}
}}

\begin{proof}
The proof proceeds by induction on the derivation of the first assumption.
\begin{description}
\setlength{\itemsep}{1em}
%===============================================================================
\item[\fbox{\texttt{(M-Simp)}}]\quad$\tenv;\type\leadsto E \ivturns \type \leadsto E; \epsilon$ \ \\
%===============================================================================
  The first assumption of the lemma is the desired conclusion
\begin{equation*}
  \tenv \vturns \type \leadsto E
\end{equation*}

%===============================================================================
\item[\fbox{\texttt{(M-IApp)}}]\quad$\tenv; \rulet_1 \iarrow \rulet_2 \leadsto E \ivturns \type \leadsto E'; \bar{\rulet} \leadsto \bar{x}, \rulet_1 \leadsto x$ \ \\
%===============================================================================
  The third hypothesis of the lemma then is
\begin{equation*}
  \tenv \vturns \bar{\rulet} \leadsto \bar{E} \quad\wedge\quad \tenv \vturns \rulet_1 \leadsto E_1
\end{equation*}
  The Weakening Lemma~\ref{lemma:weakening} turns the first conjunct into
\begin{equation*}
  \tenv, \rulet_1 \leadsto x \vturns \bar{\rulet} \leadsto \bar{E}
\end{equation*}

  The second hypothesis of the lemma then is
\begin{equation*}
  \tenv \vturns \rulet_1 \iarrow \rulet_2 \leadsto E
\end{equation*}
  By applying the Weakening Lemma~\ref{lemma:weakening} we get
\begin{equation*}
  \tenv, \rulet_1 \leadsto x \vturns \rulet_1 \iarrow \rulet_2 \leadsto E
\end{equation*}
  From rule \mylabel{R-IVar}$^2$ we can also conclude
\begin{equation*}
  \tenv, \rulet_1 \leadsto x \vturns \rulet_1 \leadsto x
\end{equation*}
  These two facts allow us to derive from rule \mylabel{R-IApp}
\begin{equation*}
  \tenv, \rulet_1 \leadsto x \vturns \rulet_2 \leadsto E\,x 
\end{equation*}

  We now have the necessary ingredients to invoke the induction hypothesis
  on the hypothesis of the rule and obtain
\begin{equation*}
  \tenv, \rulet_1 \leadsto x \vturns \type \leadsto E'[\bar{E}/\bar{x}]
\end{equation*}

  Finally, we use the second conjunct of the third hypothesis to invoke
  the Substitution Lemma~\ref{lemma:substitution} on the above and
  reach our desired conclusion
\begin{equation*}
  \tenv \vturns \type \leadsto E'[\bar{E}/\bar{x}][E_1/x]
\end{equation*}

%===============================================================================
\item[\fbox{\texttt{(M-TApp)}}]\quad$\tenv; \forall\alpha.\rulet \leadsto E \ivturns \type \leadsto E'; \bar{\rulet} \leadsto \bar{x}$ \ \\
%===============================================================================
  Then the second hypothesis of the lemma is
\begin{equation*}
  \tenv \vturns \forall\alpha.\rulet \leadsto E
\end{equation*}
  This allows us to conclude by rule \mylabel{R-TApp} that
\begin{equation*}
  \tenv \vturns \rulet[\suty/\alpha] \leadsto E\,|\suty|
\end{equation*}

  The third hypothesis of the lemma is
\begin{equation*}
  \tenv \vturns \bar{\rulet} \leadsto \bar{E}
\end{equation*}

  We now have the necessary ingredients to invoke the induction hypothesis
  on the hypothesis of the rule and obtain the desired conclusion
\begin{equation*}
  \tenv \vturns \type \leadsto E'[\bar{E}/\bar{x}]
\end{equation*}

\end{description}
\end{proof}

% %-------------------------------------------------------------------------------
% \subsection{Correctness of the Resolution Algorithm}
% 
% Lemma~\ref{lemma:sa:3} states that the resolution algorithm
% $\vdash_{\mathit{alg}}$ is sound with respect to the deterministic resolution
% specification $\vdash_r$.
% 
% Its proof relies on Lemma~\ref{lemma:sa:2}, which states that
% the auxiliary relation $\vdash_{\mathit{match1st}}$ is sound, whose
% proof in turn relies on Lemma~\ref{lemma:sa:1} which states that
% the auxiliary relation $\vdash_{\mathit{match}}$ is sound.
% 
% The completeness proof proceeds in a similar fashion.
% 
% %###############################################################################
% {\centering
% \fbox{
% \begin{minipage}{0.95\columnwidth}
% \begin{lemma}\label{lemma:sa:1}
%   If 
% \begin{equation*}
%     \rho;\bar{\rho};\bar{\alpha};\bar{\omega};E \vdash_{\mathit{match}} \tau \hookrightarrow \bar{\rho}';\bar{\omega'};E'
% \end{equation*}
%   then there exist $\bar{\rho}''$ with 
% \begin{equation*}
%   \bar{\rho}\theta \subseteq \bar{\rho'}
% \end{equation*}
%   where $\theta = [\bar{\rho''}/\bar{\alpha}]$,
% 
%   and 
% \begin{equation*}
%   \bar{\omega} \subseteq \bar{\omega}'
% \end{equation*}
% 
%   such that forall $\env$ and $\bar{E}''$: 
% 
%   if
% \begin{equation*}
%   \bar{\alpha}; \env \vdash_r \rho_i' \leadsto E_i''\quad\quad (\forall \rho_i' \in \bar{\rho}')
% \end{equation*}
% 
%   then 
% \begin{equation*}
%   \rho\theta \lhd \tau
% \end{equation*}
%   and
% \begin{equation*}
%   \bar{\alpha}; \env; \rho\theta \leadsto E|\theta|\eta \vdash_\downarrow \tau \leadsto E'\eta
% \end{equation*}
%   where $\eta = [\bar{E}''/\bar{\omega}']$
% \end{lemma}
% \end{minipage}
% }}
% 
% \begin{proof}
% The proof proceeds by induction on the derivation.
% \begin{description}
% %===============================================================================
% \item[(\texttt{MTC-Simp})]\quad$\tau';\bar{\rho};\bar{\alpha};\bar{\omega};E \vdash_{\mathit{match}} \tau \hookrightarrow \bar{\rho}\theta;\bar{\omega};E|\theta|$ \ \\
% %===============================================================================
%   Obviously, we have that 
% \begin{equation*}
%   \bar{\omega} \subseteq \bar{\omega}
% \end{equation*}
% 
%   and
% \begin{equation*}
%   \bar{\rho}\theta \subseteq \bar{\rho}\theta
% \end{equation*}
% 
%   From rule \mylabel{MTC-Simp} we have
% \begin{equation*}
%   \theta = \mathit{mgu}_{\bar{\alpha}}(\tau,\tau')
% \end{equation*}
% 
%   This means 
% \begin{equation*}
%   \tau'\theta = \tau
% \end{equation*}
% 
%   Hence, from rule \mylabel{M-Simp} it follows that
% \begin{equation*}
%   \tau'\theta \lhd \tau
% \end{equation*}
% 
%   Also, from rule \mylabel{I-Simp} it follows that
% \begin{equation*}
%   \bar{\alpha}; \env; \tau \leadsto E|\theta|\eta  \vdash_\downarrow \tau \leadsto E|\theta|\eta
% \end{equation*}
% 
%   or, equivalently,
% \begin{equation*}
%   \bar{\alpha}; \env; \tau'\theta \leadsto E|\theta|\eta \vdash_\downarrow \tau \leadsto E|\theta|\eta
% \end{equation*}
% 
% %===============================================================================
% \item[(\texttt{MTC-IAbs})]\quad$\rho_1 \iarrow \rho_2;\bar{\rho};\bar{\alpha};\bar{\omega};E \vdash_{\mathit{match}} \tau \hookrightarrow \bar{\rho}';\bar{\omega};E'$ \ \\
% %===============================================================================
% 
%   From the rule \mylabel{MTC-IAbs} and the induction hypothesis, it follows that
% \begin{equation*}
%   \bar{\omega}, \omega \subseteq \bar{\omega'}
% \end{equation*}
% 
%   Hence,
% \begin{equation*}
%   \bar{\omega} \subseteq \bar{\omega'}
% \end{equation*}
% 
%   Similarly, it follows that
% \begin{equation*}
%   (\bar{\rho},\rho_1)\theta \subseteq \bar{\rho}'
% \end{equation*}
% 
%   Hence,
% \begin{equation*}
%   \bar{\rho}\theta \subseteq \bar{\rho}'
% \end{equation*}
% 
%   Also, from the rule \mylabel{MTC-IAbs} and the induction hypothesis, it follows that
% \begin{equation*}
%   \rho_2\theta \lhd \tau
% \end{equation*}
% 
%   and, by rule \mylabel{M-IApp}, we hence have
% \begin{equation*}
%   \rho_1\theta \iarrow \rho_2\theta \lhd \tau
% \end{equation*}
% 
%   or, more succinctly,
% \begin{equation*}
%   (\rho_1 \iarrow \rho_2)\theta \lhd \tau
% \end{equation*}
% 
%   Finally, from the rule \mylabel{MTC-IAbs} and the induction hypothesis, it follows that
% \begin{equation*}
%  \bar{\alpha}; \env; \rho_2\theta \leadsto (E\,\omega)|\theta|\eta \vdash_\downarrow \tau \leadsto E'\eta 
% \end{equation*}
% 
%   or
% \begin{equation*}
%  \bar{\alpha}; \env; \rho_2\theta \leadsto (E\theta\eta)\,(\omega\eta) \vdash_\downarrow \tau \leadsto E'\eta 
% \end{equation*}
% 
%   Using rule \mylabel{I-IAbs} we may then conclude
% \begin{equation*}
%  \bar{\alpha}; \env; \rho_1\theta \iarrow \rho_2\theta \leadsto E\theta\eta \vdash_\downarrow \tau \leadsto E'\eta 
% \end{equation*}
% 
%   or, equivalently,
% \begin{equation*}
%  \bar{\alpha}; \env; (\rho_1 \iarrow \rho_2)\theta \leadsto E\theta\eta \vdash_\downarrow \tau \leadsto E'\eta 
% \end{equation*}
% 
% %===============================================================================
% \item[(\texttt{MTC-TAbs})]\quad$\forall \alpha.\rho;\bar{\rho};\bar{\alpha};\bar{\omega};E \vdash_{\mathit{match}} \tau \hookrightarrow \bar{\rho}';\bar{\omega};E'$ \ \\
% %===============================================================================
% 
%   From the rule \mylabel{MTC-TAbs} and the induction hypothesis, it follows that
% \begin{equation*}
%   \bar{\omega} \subseteq \bar{\omega'}
% \end{equation*}
% 
%   Similarly, it follows that
% \begin{equation*}
%   \bar{\rho}\theta \subseteq \bar{\rho}'
% \end{equation*}
% 
%   Also it follows that
% \begin{equation*}
%   \rho\theta \lhd \tau
% \end{equation*}
% 
%   or, equivalently,
% \begin{equation*}
%   \rho[\bar{\alpha}\theta/\bar{\alpha}][\alpha\theta/\alpha] \lhd \tau
% \end{equation*}
% 
%   Hence, following rule \mylabel{MTC-TAbs}, we have that
% \begin{equation*}
%   \forall \alpha.(\rho[\bar{\alpha}\theta/\bar{\alpha}]) \lhd \tau
% \end{equation*}
% 
%   or, equivalently,
% \begin{equation*}
%   (\forall \alpha.\rho)[\bar{\alpha}\theta/\bar{\alpha}] \lhd \tau
% \end{equation*}
% 
%   Finally, from the rule \mylabel{MTC-TAbs} and the induction hypothesis, it follows that
% \begin{equation*}
%   \bar{\alpha}; \env; \rho\theta \leadsto (E\,\alpha)|\theta|\eta \vdash_\downarrow \tau \leadsto E'\eta 
% \end{equation*}
% 
%   or, equivalently,
% \begin{equation*}
%   \bar{\alpha}; \env; \rho\theta \leadsto (E|\theta|[\alpha|\theta|/\alpha]\eta)\,(\alpha|\theta|) \vdash_\downarrow \tau \leadsto E'\eta 
% \end{equation*}
% 
%   Hence, following rule \mylabel{I-TAbs}, we get
% \begin{equation*}
%   \bar{\alpha}; \env; (\forall \alpha.\rho)\theta \leadsto E|\theta|\eta \vdash_\downarrow \tau \leadsto E'\eta 
% \end{equation*}
% \end{description}
% \end{proof}
% 
% %###############################################################################
% {\centering
% \fbox{
% \begin{minipage}{0.95\columnwidth}
% \begin{lemma}\label{lemma:sa:2}
%   If 
% \begin{equation*}
%    \bar{\alpha}; \env \vdash_{\mathit{match1st}} \tau \hookrightarrow \bar{\rho}';\bar{\omega};E
% \end{equation*}
%   then there exist $\rho$ and $x$ such that
% 
% \begin{equation*}
%   \elookup{\env}{\tau} = \rho \leadsto x
% \end{equation*}
% 
% and for all $\env' \supseteq \env$
% and for all $\bar{E}'$ such that
% \begin{equation*}
%   \bar{\alpha}; \env' \vdash_r \rho_i' \leadsto E_i'\quad\quad (\forall \rho_i' \in \bar{\rho}')
% \end{equation*}
% 
% we have that
% \begin{equation*}
%   \bar{\alpha}; \env'; \rho \leadsto x \vdash_\downarrow \tau \leadsto E\eta
% \end{equation*}
% where $\eta = [\bar{E}'/\bar{\omega}]$.
% \end{lemma}
% \end{minipage}
% }}
% 
% \begin{proof}
% The proof proceeds by induction on the derivation.
% \begin{description}
% %===============================================================================
% \item[(\texttt{M1-Head})]\quad$\bar{\alpha}; \env,\rho \leadsto x \vdash_{\mathit{match1st}} \tau \hookrightarrow \bar{\rho};\bar{\omega};E$ \ \\
% %===============================================================================
%   From rule \mylabel{M1-Head} and the previous lemma, we then have that
% \begin{equation*}
%   \rho \lhd \tau 
% \end{equation*}
%   
%   Using rule \mylabel{L-Head} we can conclude that
% \begin{equation*}
%   \elookup{(\env,\rho\leadsto x)}{\tau} = \rho \leadsto x
% \end{equation*}
% 
%   Similarly, we have that 
% \begin{equation*}
%   \bar{\alpha}; \env'; \rho \leadsto x\eta \vdash_\downarrow \tau \leadsto E\eta
% \end{equation*}
% 
%   which simplifies to
% \begin{equation*}
%   \bar{\alpha}; \env'; \rho \leadsto x \vdash_\downarrow \tau \leadsto E\eta
% \end{equation*}
% 
% %===============================================================================
% \item[(\texttt{M1-Tail})]\quad$\bar{\alpha}; \env,\rho \leadsto x \vdash_{\mathit{match1st}} \tau \hookrightarrow \bar{\rho};\bar{\omega};E$ \ \\
% %===============================================================================
%   From rule \mylabel{M1-Tail} and the induction hypothesis, we then have that there
%   exist $\rho'$ and $x'$ such that
% \begin{equation*}
%   \elookup{\env}{\tau} = \rho' \leadsto x'
% \end{equation*}
% 
%   Also from rule \mylabel{M1-Tail} and a previous lemma, we have that
% \begin{equation*}
%   \rho \not\!\!\lhd \tau
% \end{equation*}
% 
%   Hence, using rule \mylabel{L-Tail} we can conclude that 
% \begin{equation*}
%   \elookup{(\env,\rho\leadsto x)}{\tau} = \rho' \leadsto x'
% \end{equation*}
% 
%   From rule \mylabel{M1-Tail} and the induction hypothesis, we also have that
% \begin{equation*}
%   \bar{\alpha}; \env'; \rho' \leadsto x' \vdash_\downarrow \tau \leadsto E\eta
% \end{equation*}
%   for $\env' \supseteq (\env,\rho \leadsto x)$.
% 
% \end{description}
% \end{proof}
% 
% %###############################################################################
% {\centering
% \fbox{
% \begin{minipage}{0.95\columnwidth}
% \begin{lemma}\label{lemma:sa:3}
%   If 
% \begin{equation*}
%    \bar{\alpha}; \env \vdash_{\mathit{alg}} \rho \leadsto E
% \end{equation*}
% 
%   then
% \begin{equation*}
%   \bar{\alpha}; \env \vdash_r \rho \leadsto E
% \end{equation*}
% \end{lemma}
% \end{minipage}
% }}
% 
% \begin{proof}
% The proof proceeds by induction on the derivation.
% \begin{description}
% %===============================================================================
% \item[(\texttt{Alg-TAbs})]\quad$\bar{\alpha}; \env \vdash_{\mathit{alg}} \forall \alpha.\rho \leadsto \Lambda \alpha.E$ \ \\
% %===============================================================================
%   From rule \mylabel{Alg-TAbs} and the induction hypothesis, it follows that
% \begin{equation*}
%   \bar{\alpha}; \env \vdash_{\mathit{r}} \rho \leadsto E
% \end{equation*}
% 
%   Using rule \mylabel{R-TAbs}, we then get
% \begin{equation*}
%   \bar{\alpha}; \env \vdash_{\mathit{r}} \forall \alpha.\rho \leadsto \Lambda \alpha.E
% \end{equation*}
% 
% %===============================================================================
% \item[(\texttt{Alg-IAbs})]\quad$\bar{\alpha}; \env \vdash_{\mathit{alg}} \rho_1 \iarrow \rho_2 \leadsto \lambda (x:|\rho_1|).E$ \ \\
% %===============================================================================
%   From rule \mylabel{Alg-IAbs} and the induction hypothesis, it follows that
% \begin{equation*}
%   \bar{\alpha}; \env, \rho_1 \leadsto x \vdash_{\mathit{r}} \rho_2 \leadsto E
% \end{equation*}
% 
%   Using rule \mylabel{R-IAbs}, we then get
% \begin{equation*}
%   \bar{\alpha}; \env \vdash_{\mathit{r}} \rho_1 \iarrow \rho_2 \leadsto \lambda (x:|\rho_1|).E
% \end{equation*}
% 
% %===============================================================================
% \item[(\texttt{Alg-Simp})]\quad$\bar{\alpha}; \env \vdash_{\mathit{alg}} \tau \leadsto E[\bar{\omega}/\bar{E}]$ \ \\
% %===============================================================================
%   From rule \mylabel{Alg-Simp} and the previous lemma it follows that
% \begin{equation*}
%   \elookup{\env}{\tau} = \rho \leadsto x
% \end{equation*}
% 
%   and
% \begin{equation*}
%   \bar{\alpha}; \env; \rho \leadsto x \vdash_\downarrow \tau \leadsto E[\bar{\omega}/\bar{E}]
% \end{equation*}
% 
%   Hence, using rule \mylabel{R-Simp}, we conclude
% \begin{equation*}
%   \bar{\alpha}; \env \vdash_r \tau \leadsto E[\bar{\omega}/\bar{E}]
% \end{equation*}
% 
% \end{description}
% \end{proof}
% 
% %-------------------------------------------------------------------------------
% % \subsection{Completeness of the Resolution Algorithm}
% % 
% % %###############################################################################
% % {\centering
% % \fbox{
% % \begin{minipage}{0.95\columnwidth}
% % \begin{lemma}\label{lemma:sa:2}
% %   If 
% % \begin{equation*}
% %    \rho \lhd \tau
% % \end{equation*}
% % 
% %   then for all $\rho^*, \bar{\rho}, \bar{\alpha}, \bar{\omega}, E, \theta$ such that
% % \begin{equation*}
% %   \rho = \rho^*\theta
% % \end{equation*}
% % 
% %   where $\mathit{dom}(\theta) \subseteq \bar{\alpha}$, we have that
% %   there exist $\bar{\rho}', \bar{\omega}', E'$ such that
% % 
% % \begin{equation*}
% %   \rho^*; \bar{\rho}; \bar{\alpha}; \bar{\omega}; E \vdash_\mathit{match} \tau \hookrightarrow \bar{\rho}'; \bar{\omega}'; E'
% % \end{equation*}
% % \end{lemma}
% % \end{minipage}
% % }}
% % 
% % \begin{proof}
% % The proof proceeds by induction on the derivation.
% % \begin{description}
% % %===============================================================================
% % \item[(\texttt{M-Simp})]\quad$\tau \lhd \tau$ \ \\
% % %===============================================================================
% %   From the assumption and the definition of $\mathit{mgu}_{\bar{\alpha}}$, it follows that
% % \begin{equation*}
% %   \theta = \mathit{mgu}_{\bar{\alpha}}(\tau^*,\tau)
% % \end{equation*}
% % 
% %   Hence, by rule \mylabel{MTC-Simp}, we conclude
% % \begin{equation*}
% %   \tau^*; \bar{\rho}; \bar{\alpha}; \bar{\omega}; E \vdash_\mathit{match} \tau \hookrightarrow \bar{\rho}\theta; \bar{\omega}; E\theta
% % \end{equation*}
% % 
% % %===============================================================================
% % \item[(\texttt{M-IApp})]\quad$\rho_1 \iarrow \rho_2 \lhd \tau$ \ \\
% % %===============================================================================
% %   From rule \mylabel{MTC-IAbs} and the induction hypothesis, we have that
% % \begin{equation*}
% %   \rho^*_2; \bar{\rho},\rho^*_1; \bar{\alpha}; \bar{\omega},\omega; E \vdash_\mathit{match} \tau \hookrightarrow \bar{\rho}'; \bar{\omega}'; E'
% % \end{equation*}
% % 
% %   Hence, by rule \mylabel{MTC-IAbs}, we conclude
% % \begin{equation*}
% %   \rho^*_1 \iarrow \rho^*_2; \bar{\rho}; \bar{\alpha}; \bar{\omega}; E \vdash_\mathit{match} \tau \hookrightarrow \bar{\rho}'; \bar{\omega}'; E'
% % \end{equation*}
% % \end{description}
% % \end{proof}
% 
%-------------------------------------------------------------------------------
\subsection{Deterministic Resolution is Deterministic}\label{proof:determinism}

\newcommand{\unamb}{\vdash_{\mathit{unamb}}}

%###############################################################################
{\centering
\fbox{
\begin{minipage}{0.95\columnwidth}
\begin{lemma}\label{lemma:determinism:0}
  If 
\begin{equation*}
  \unamb \tenv
\end{equation*}
  and
\begin{equation*}
  \unamb \rulet
\end{equation*}
  and
\begin{equation*}
  \tenv \ivturns \rulet \leadsto E_1
\end{equation*}
and
\begin{equation*}
  \tenv \ivturns \rulet \leadsto E_2
\end{equation*}
then
\begin{equation*}
  E_1 = E_2
\end{equation*}
\end{lemma}
\end{minipage}
}}

\begin{proof}
From the third and fourth hypotheses of the lemma, the hypothesis
of rule \mylabel{R-Main} and Lemma~\ref{lemma:determinism:1} the desired
result follows
\begin{equation*}
  E_1 = E_2
\end{equation*}
\end{proof}

%###############################################################################
{\centering
\fbox{
\begin{minipage}{0.95\columnwidth}
\begin{lemma}\label{lemma:determinism:1}
  If 
\begin{equation*}
  \unamb \tenv
\end{equation*}
  and
\begin{equation*}
  \unamb \rulet
\end{equation*}
  and
\begin{equation*}
  \bar{\alpha};\tenv \ivturns \rulet \leadsto E_1
\end{equation*}
and
\begin{equation*}
  \bar{\alpha};\tenv \ivturns \rulet \leadsto E_2
\end{equation*}
then
\begin{equation*}
  E_1 = E_2
\end{equation*}
\end{lemma}
\end{minipage}
}}

\begin{proof}
The proof proceeds by induction on the derivation of the third hypothesis.
\begin{description}
\setlength{\itemsep}{1em}
%===============================================================================
\item[\fbox{\texttt{(R-IAbs)}}]\quad$\bar{\alpha};\tenv \ivturns \rulet_1 \iarrow \rulet_2 \leadsto \lambda\relation{x}{||\rulet_1||}.E_1$ \ \\
%===============================================================================

It follows that the lemma's fourth hypothesis is also derived by rule
\mylabel{R-IAbs}. It follows from the lemma's second hypothesis that 
\begin{equation*}
  \unamb \rulet_1 \quad\wedge\quad \unamb \rulet_2
\end{equation*}
From this and lemma's first hypothesis, it follows that
\begin{equation*}
  \unamb \tenv,\rulet_1 \leadsto x
\end{equation*}
From the rule's hypothesis and the induction hypothesis, it follows that
\begin{equation*}
  E_1 = E_2
\end{equation*}
Hence, we may conclude
\begin{equation*}
  \lambda x:|\rulet_1|.E_1 = \lambda x:|\rulet_1|.E_2
\end{equation*}

%===============================================================================
\item[\fbox{\texttt{(R-TAbs)}}]\quad$\bar{\alpha};\tenv \ivturns \forall \alpha. \rho \leadsto \Lambda\alpha.E_1$ \ \\
%===============================================================================

It follows that the lemma's fourth hypothesis is also derived by rule
\mylabel{R-TAbs}. It follows from the lemma's second hypothesis that 
\begin{equation*}
  \unamb \rulet
\end{equation*}
From the lemma's first hypothesis, it follows that
\begin{equation*}
  \unamb \tenv, \alpha
\end{equation*}
From the rule's hypothesis and the induction hypothesis, it follows that
\begin{equation*}
  E_1 = E_2
\end{equation*}
Hence, we may conclude
\begin{equation*}
  \Lambda \alpha.E_1 = \Lambda \alpha.E_2
\end{equation*}

%===============================================================================
\item[\fbox{\texttt{(R-Simp)}}]\quad$\bar{\alpha};\tenv \ivturns \type \leadsto E_1$ \ \\
%===============================================================================

It follows that the lemma's fourth hypothesis is also derived by rule
\mylabel{R-Simp}. We obtain the desired result from Lemma~\ref{lemma:determinism:2}
\begin{equation*}
  E_1 = E_2
\end{equation*}

   
\end{description}
\end{proof}

%###############################################################################
{\centering
\fbox{
\begin{minipage}{0.95\columnwidth}
\begin{lemma}\label{lemma:determinism:2}
  If 
\begin{equation*}
  \unamb \tenv
\end{equation*}
  and
\begin{equation*}
  \unamb \tenv'
\end{equation*}
  and
\begin{equation*}
  \bar{\alpha};\tenv;\tenv' \ivturns \type \leadsto E_1
\end{equation*}
and
\begin{equation*}
  \bar{\alpha};\tenv;\tenv' \ivturns \type \leadsto E_2
\end{equation*}
then
\begin{equation*}
  E_1 = E_2
\end{equation*}
\end{lemma}
\end{minipage}
}}

\begin{proof}
The proof proceeds by induction on the derivation of the third hypothesis.
\begin{description}
\setlength{\itemsep}{1em}
%===============================================================================
\item[\fbox{\texttt{(L-RuleMatch)}}]\quad$\bar{\alpha};\tenv;\tenv',\rulet \leadsto x \ivturns \type \leadsto E_1[\bar{E}_1/\bar{x}]$ \ \\
%===============================================================================
  Then the fourth hypothesis was either derived from rule \mylabel{L-RuleMatch},
  or from rule \mylabel{L-RuleNoMatch}. However, the hypothesis of the latter is
  not satisfied: $\epsilon;E_1;\Sigma_1$ forms a counter-example. Hence, the fourth
  hypothesis is also formed by rule \mylabel{L-RuleMatch}.

  Then it follows from the first hypothesis of the rule and Lemma~\ref{lemma:determinism:3} that
\begin{equation*}
  E_1 = E_2 \quad\wedge\quad \Sigma_1 = \Sigma_2
\end{equation*}
  From the second hypothesis of the rule and Lemma~\ref{lemma:determinism:1} it also follows that
\begin{equation*}
  \bar{E}_1 = \bar{E_2}
\end{equation*}
  Hence, we may conclude
\begin{equation*}
  E_1[\bar{E}_1/\bar{x}] = E_2[\bar{E_2}/\bar{x}]
\end{equation*}


%===============================================================================
\item[\fbox{\texttt{(L-RuleNoMatch)}}]\quad$\bar{\alpha};\tenv;\tenv',\rulet \leadsto x \ivturns \type \leadsto E_1'$ \ \\
%===============================================================================
  Then the fourth hypothesis was either derived from rule \mylabel{L-RuleMatch},
  or from rule \mylabel{L-RuleNoMatch}. However, the hypothesis of the former is
  not satisfied, as it would be a counter-example for the first hypothesis of
  the assumed rule of the third hypothesis. Hence, the fourth hypothesis is also
  formed by rule \mylabel{L-RuleNoMatch}.

  From the second hypothesis of the lemma we derive $\unamb \tenv'$.
  Then from the second hypothesis of the rule and the induction hypothesis we conclude
  the desired result
\begin{equation*}
  E_1' = E_2'
\end{equation*}

%===============================================================================
\item[\fbox{\texttt{(L-Var)}}]\quad$\bar{\alpha};\tenv;\tenv',x:\rulet \ivturns \type \leadsto E_1$ \ \\
%===============================================================================
  Clearly the fourth hypothesis is also derived by rule \mylabel{L-Var}.
  Moreover, from the second hypothesis it follows that $\unamb \tenv'$.
  Hence, from the induction hypothesis we conclude that
\begin{equation*}
  E_1 = E_2
\end{equation*}

%===============================================================================
\item[\fbox{\texttt{(L-TyVar)}}]\quad$\bar{\alpha};\tenv;\tenv',\alpha \ivturns \type \leadsto E_1$ \ \\
%===============================================================================
  Clearly the fourth hypothesis is also derived by rule \mylabel{L-TyVar}.
  Moreover, from the second hypothesis it follows that $\unamb \tenv'$.
  Hence, from the induction hypothesis we conclude that
\begin{equation*}
  E_1 = E_2
\end{equation*}
\end{description}
\end{proof}

We annotated the judgement with the sequence of substitution types $\bar{\suty}$
used to instantiate the universal quantifiers.

{\centering
\fbox{
\begin{minipage}{0.95\columnwidth}
\begin{center}
$\ba{c}
\myruleform{\bar{\suty};\tenv; \rulet~\gbox{\leadsto E} \ivturns \type~\gbox{\leadsto E'}; \Sigma}\\ \\
\mylabel{M-Simp} \quad
          {\epsilon;\tenv; \type~\gbox{\leadsto E} \ivturns \type~\gbox{\leadsto E}; \epsilon} \\ \\
\mylabel{M-IApp} \quad
  \myirule{\bar{\suty};\tenv, \rulet_1 \gbox{\leadsto x}; \rulet_2 ~\gbox{\leadsto E\,x} \ivturns \type~\gbox{\leadsto E'}; \Sigma 
           \quad\quad\quad \gbox{x~\mathit{fresh}}
          }
          {\bar{\suty};\tenv; \rulet_1 \iarrow \rulet_2 ~\gbox{\leadsto E} \ivturns \type~\gbox{\leadsto E'}; \Sigma, \rulet_1~\gbox{\leadsto x}} \\ \\ 
\mylabel{M-TApp} \quad
  \myirule{\bar{\suty};\tenv; \rulet[\suty/\alpha] ~\gbox{\leadsto E\,|\suty|} \ivturns \type~\gbox{\leadsto E'; \Sigma}
           \quad\quad\quad
           \tenv \turns \suty
          }
          {\bar{\suty},\suty;\tenv; \forall \alpha. \rulet ~\gbox{\leadsto E} \ivturns \type~\gbox{\leadsto E'}; \Sigma} \\
\ea
$
\end{center}
\end{minipage}
}}

It is not difficult to see that any derivation of the annotated judgement
is in one to one correspondence with a derivation of the unannotated 
judgement.

The judgement is deterministic.

%###############################################################################
{\centering
\fbox{
\begin{minipage}{0.95\columnwidth}
\begin{lemma}\label{lemma:determinism:3}
  If 
\begin{equation*}
  \bar{\alpha} \unamb \rulet
\end{equation*}
  and
\begin{equation*}
  \bar{\suty}_1 ; \tenv; \rulet[\bar{\suty}_2/\bar{\alpha}]\leadsto E[|\bar{\suty}_2|/\bar{\alpha}] \ivturns \type \leadsto E_1; \Sigma_1
\end{equation*}
and
\begin{equation*}
  \bar{\suty}_1' ; \tenv; \rulet[\bar{\suty}_2'/\bar{\alpha}]\leadsto E[|\bar{\suty}_2'|/\bar{\alpha}] \ivturns \type \leadsto E_2; \Sigma_2
\end{equation*}
then
\begin{equation*}
  \bar{\suty}_1 = \bar{\suty}_1' \quad\wedge\quad \bar{\suty}_2 = \bar{\suty}_2'
  \quad\wedge\quad E_1 = E_2 \quad\wedge\quad \Sigma_1 = \Sigma_2
  \quad\wedge\quad \unamb \Sigma_1
\end{equation*}
% where
% \begin{equation*}
% \begin{array}{c}
%   \myruleform{\bar{\suty}; \rulet \vdash \type} \\ \\
%   \mylabel{P-1} \quad \myirule{}{\epsilon; \type \vdash \type} \\ \\
%   \mylabel{P-2} \quad \myirule{\bar{\suty}; \rulet_2 \vdash \type}{\bar{\suty}; \rulet_1 \iarrow \rulet_2 \vdash \type}
%   \mylabel{P-3} \quad \myirule{\bar{\suty}; \rulet[\suty/\alpha] \vdash \type}{\suty,\bar{\suty}; \forall\alpha.\rulet \vdash \type}
% \end{array}
% \end{equation*}
\end{lemma}
\end{minipage}
}}

\begin{proof}
The proof proceeds by induction on the derivation of the first hypothesis.
\begin{description}
\setlength{\itemsep}{1em}
%===============================================================================
\item[\fbox{\texttt{(UA-Simp)}}]\quad$\bar{\alpha} \unamb \type'$ \ \\
%===============================================================================
  Then the second and third hypothesis of the lemma must have been formed by rule \mylabel{M-Simp}
  and hence 
\begin{equation*}
  \bar{\suty}_1 = \epsilon = \bar{\suty}_1'
\end{equation*}
  
  For the same reason we have that $\type'[\bar{\suty}_2/\bar{\alpha}] = \type = \type'[\bar{\suty}_2'/\bar{\alpha}]$. Since we know that $\bar{\alpha} \subseteq \mathit{ftv}(\type)$, it must follow also that
\begin{equation*}
  \bar{\suty}_2 = \bar{\suty}_2'
\end{equation*}

  As a consequence, we also have that
\begin{equation*}
  E_1 = E[|\bar{\sigma}_2|/\bar{\alpha}] = E[|\bar{\sigma}_2'|/\bar{\alpha}] = E_2
\end{equation*}

  Finally, it also follows from rule \mylabel{M-Simp} that
\begin{equation*}
  \Sigma_1 = \epsilon = \Sigma_2
\end{equation*}
  and trivially
\begin{equation*}
  \unamb \epsilon
\end{equation*}

%===============================================================================
\item[\fbox{\texttt{(UA-IAbs)}}]\quad$\bar{\alpha} \unamb \rulet_1 \iarrow \rulet_2$ \ \\
%===============================================================================
  Then the second and third hypothesis of the lemma must have been formed by rule \mylabel{M-IApp}.
  From their two hypotheses and from the hypothesis of the rule and the induction hypothesis, we obtain
  the desired results
\begin{equation*}
  \bar{\suty}_1 = \bar{\suty}_1' \quad\wedge\quad \bar{\suty}_2 = \bar{\suty}_2'
  \quad\wedge\quad 
  E_1 = E_2
  \quad\wedge\quad
  \Sigma_1, \rulet_1[|\bar{\sigma}_2|/\bar{\alpha}]\leadsto x
  = 
  \Sigma_2, \rulet_1[|\bar{\sigma}_2'|/\bar{\alpha}]\leadsto x
\end{equation*}

  We also derive from the induction hypothesis that $\unamb \Sigma_1$. 
  Since $\bar{\alpha} \unamb \rulet_1 \iarrow \rulet_2$, we also have $\unamb \rulet_1$.
  Hence we also conclude
\begin{equation*}
  \unamb \Sigma_1, \rulet_1 \leadsto x
\end{equation*}

%===============================================================================

\item[\fbox{\texttt{(UA-TAbs)}}]\quad$\bar{\alpha} \unamb \forall \alpha.\rulet$ \ \\
%===============================================================================
  Then the second and third hypothesis of the lemma must have been formed by rule \mylabel{M-TApp},
  with $\bar{\suty}_1 = \bar{\suty}_{1,1},\suty_{1,2}$ and $\bar{\suty}_1' = \bar{\suty}_{1,1}',\suty_{2,2}'$.
  From their two hypotheses and from the hypothesis of the rule and the induction hypothesis, we obtain
\begin{equation*}
  \bar{\suty}_2,\suty_{1,2} = \bar{\suty}_2',\suty_{1,2}' 
  \quad\wedge\quad 
  \bar{\suty}_{1,1} = \bar{\suty}_{1,1}'
  \quad\wedge\quad 
  E_1 = E_2
  \quad\wedge\quad 
  \Sigma_1 = \Sigma_2
  \quad\wedge\quad 
  \unamb \Sigma_1
\end{equation*}
  
  From this we conclude the desired result 
\begin{equation*}
  \bar{\suty}_1 = \bar{\suty}_1' \quad\wedge\quad \bar{\suty}_2 = \bar{\suty}_2'
  \quad\wedge\quad 
  E_1 = E_2
  \quad\wedge\quad 
  \Sigma_1 = \Sigma_2
  \quad\wedge\quad 
  \unamb \Sigma_1
\end{equation*}
   
\end{description}
\end{proof}

%-------------------------------------------------------------------------------
\subsection{Resolution Coherence}\label{proof:coherence}

Deterministic resolution is stable under substitution.

%###############################################################################
{\centering
\fbox{
\begin{minipage}{0.95\columnwidth}
\begin{lemma}\label{lemma:coherence:0}
  If 
\begin{equation*}
   \tenv,\alpha,\tenv' \ivturns \rulet\leadsto E
\end{equation*}
  and
\begin{equation*}
  \tenv \turns \suty
\end{equation*}
  then
\begin{equation*}
  \tenv,\tenv'[\suty/\alpha] \ivturns \rulet[\suty/\alpha] \leadsto E[|\suty|/\alpha]
\end{equation*}
\end{lemma}
\end{minipage}
}}

\begin{proof}
  The hypothesis of rule \label{R-Main} then is
\begin{equation*}
  \mathit{ftv}(\tenv),\alpha,\mathit{ftv}(\tenv'); \tenv,\alpha,\tenv' \ivturns \rulet \leadsto E
\end{equation*}

  From Lemma~\ref{lemma:coherence:1} it follows that
\begin{equation*}
  \mathit{ftv}(\tenv),\mathit{ftv}(\tenv'); \tenv,\tenv'[\suty/\alpha] \ivturns \rulet[\suty/\alpha] \leadsto E[|\suty|/\alpha]
\end{equation*}

  As $\mathit{ftv}(\tenv') = \mathit{ftv}(\tenv'[\suty/\alpha])$, the desired result
  follows from
  rule \mylabel{R-Main}
\begin{equation*}
  \tenv,\tenv'[\suty/\alpha] \ivturns \rulet[\suty/\alpha] \leadsto E[|\suty|/\alpha]
\end{equation*}
\end{proof}

%###############################################################################
{\centering
\fbox{
\begin{minipage}{0.95\columnwidth}
\begin{lemma}\label{lemma:coherence:1}
  If 
\begin{equation*}
   \bar{\alpha},\alpha,\bar{\alpha}'; \tenv,\alpha,\tenv' \ivturns \rulet\leadsto E
\end{equation*}
  and
\begin{equation*}
  \tenv \turns \suty
\end{equation*}
  then
\begin{equation*}
  \bar{\alpha},\bar{\alpha}'; \tenv,\tenv'[\suty/\alpha] \ivturns \rulet[\suty/\alpha] \leadsto E[|\suty|/\alpha]
\end{equation*}
\end{lemma}
\end{minipage}
}}

\begin{proof}
\begin{description}
\setlength{\itemsep}{1em}
%===============================================================================
\item[\fbox{\texttt{(R-IAbs)}}]\quad$\bar{\alpha},\alpha,\bar{\alpha}';\tenv,\alpha,\tenv'
\ivturns \rulet_1 \iarrow \rulet_2 \leadsto \lambda\relation{x}{|\rulet_1|}.E$ \ \\
%===============================================================================
  From the rule's hypothesis and the induction hypothesis we have
\begin{equation*}
\bar{\alpha},\bar{\alpha}';\tenv,(\tenv',\rulet_1\leadsto x)[\suty/\alpha] \ivturns \rulet_2[\suty/\alpha] \leadsto E[|\suty|/\alpha]
\end{equation*}

  From the definition of substitution and rule \mylabel{R-IAbs} we then conclude
\begin{equation*}
\bar{\alpha},\bar{\alpha}';\tenv,\tenv'[\suty/\alpha] \ivturns (\rulet_1 \iarrow \rulet_2)[\suty/\alpha] \leadsto (\lambda x:|\rulet_1|.E)[|\suty|/\alpha]
\end{equation*}

%===============================================================================
\item[\fbox{\texttt{(R-TAbs)}}]\quad$\bar{\alpha},\alpha,\bar{\alpha}';\tenv,\alpha,\tenv'
\ivturns \forall\beta.\rulet \leadsto \Lambda\beta.E$ \ \\
%===============================================================================
  From the rule's hypothesis and the induction hypothesis we have
\begin{equation*}
\bar{\alpha},\bar{\alpha}';\tenv,(\tenv',\beta)[\suty/\alpha] \ivturns \rulet[\suty/\alpha] \leadsto E[|\suty|/\alpha]
\end{equation*}

  From the definition of substitution and rule \mylabel{R-TAbs} we then conclude
\begin{equation*}
\bar{\alpha},\bar{\alpha}';\tenv,\tenv'[\suty/\alpha] \ivturns (\forall\beta.\rulet)[\suty/\alpha] \leadsto (\Lambda\beta.E)[|\suty|/\alpha]
\end{equation*}

%===============================================================================
\item[\fbox{\texttt{(R-Simp)}}]\quad$\bar{\alpha},\alpha,\bar{\alpha}';\tenv,\alpha,\tenv'
\ivturns \type \leadsto E$ \ \\
%===============================================================================
  From the rule's hypothesis and Lemma~\ref{lemma:coherence:2} we conclude
\begin{equation*}
  \bar{\alpha},\bar{\alpha}'; \tenv,\tenv'[\suty/\alpha];\tenv''' \ivturns \type[\suty/\alpha] \leadsto E[|\suty|/\alpha]
\end{equation*}
  and
\begin{equation*}
  R(\tenv;\alpha;\tenv';\tenv,\alpha,\tenv';\tenv''';\suty)
\end{equation*}
  The latter could only have been obtained by ryle \mylabel{R-2}. 
  Hence, we know that $\tenv''' = \tenv,\tenv'[\suty/\alpha]$ and the former is equivalent to
\begin{equation*}
  \bar{\alpha},\bar{\alpha}'; \tenv,\tenv'[\suty/\alpha];\tenv,\tenv'[\suty/\alpha] \ivturns \type[\suty/\alpha] \leadsto E[|\suty|/\alpha]
\end{equation*}

  With this fact we can conclude by rule \mylabel{R-Simp}
\begin{equation*}
\bar{\alpha},\bar{\alpha}';\tenv,\tenv'[\suty/\alpha] \ivturns \type[\suty/\alpha] \leadsto E[|\suty|/\alpha]
\end{equation*}
\end{description}
\end{proof}

%###############################################################################
{\centering
\fbox{
\begin{minipage}{0.95\columnwidth}
\begin{lemma}\label{lemma:coherence:2}
  If 
\begin{equation*}
   \bar{\alpha},\alpha,\bar{\alpha}'; \tenv,\alpha,\tenv';\tenv'' \ivturns \type \leadsto E
\end{equation*}
  and
\begin{equation*}
  \tenv'' \subseteq \tenv,\alpha,\tenv'
\end{equation*}
  and
\begin{equation*}
  \tenv \turns \suty
\end{equation*}
  then
\begin{equation*}
  \bar{\alpha},\bar{\alpha}'; \tenv,\tenv'[\suty/\alpha];\tenv''' \ivturns \type[\suty/\alpha] \leadsto E[|\suty|/\alpha]
\end{equation*}
  and
\begin{equation*}
  R(\tenv;\alpha;\tenv';\tenv'';\tenv''';\suty)
\end{equation*}
\end{lemma}
where
\begin{equation*}
\begin{array}{c}
  \myruleform{R(\tenv;\alpha;\tenv';\tenv'';\tenv''';\suty)} \\ \\
  \mylabel{R-1} \quad \myirule{}{R(\tenv_1,\tenv_2;\alpha;\tenv';\tenv_1;\tenv_1;\suty)} \\ \\
  \mylabel{R-2} \quad \myirule{}{R(\tenv;\alpha;\tenv_1',\tenv_2';\tenv,\alpha,\tenv_1';\tenv,\tenv_1'[\suty/\alpha];\suty)}
\end{array}
\end{equation*}
  
\end{minipage}
}}

\begin{proof}
\begin{description}
\setlength{\itemsep}{1em}
%===============================================================================
\item[\fbox{\texttt{(L-RuleMatch)}}]\quad$
  \bar{\alpha},\alpha,\bar{\alpha}'; \tenv,\alpha,\tenv'; \tenv'',\rulet \leadsto x \ivturns \type \leadsto E
$ \ \\
%===============================================================================
  Then it follows from the first hypothesis of the rule and of Lemma~\ref{lemma:coherence:3}
  that
\begin{equation*}
\tenv,\tenv'[\suty/\alpha]; \rulet[\suty/\alpha] \leadsto E[|\suty|/\alpha] \ivturns \type[\suty/\alpha] \leadsto E'[|\suty|/\alpha]; \bar{\rulet}[\suty/\alpha] \leadsto \bar{x}
\end{equation*}

  Also it follows from the second hypothesis of the rule and of Lemma~\ref{lemma:coherence:1}
  that
\begin{equation*}
  \bar{\alpha},\bar{\alpha}'; \tenv,\tenv'[\suty/\alpha] \ivturns \bar{\rulet}[\suty/\alpha] \leadsto \bar{E}[|\suty|/\alpha]
\end{equation*}

  By combining these two observations with rule \mylabel{L-RuleMatch} we obtain the first desired result
\begin{equation*}
  \bar{\alpha},\bar{\alpha}'; \tenv,\tenv'[\suty/\alpha]; \tenv''',\rulet[\suty/\alpha] \leadsto x \ivturns \type[\suty/\alpha] \leadsto E[|\suty|/\alpha]
\end{equation*}
  
  We obtain the second desired result by case analysis on $\tenv'' \subseteq \tenv,\alpha,\tenv'$:
  \begin{enumerate}
  \item $\tenv = \tenv_1,\rulet \leadsto x,\tenv_2 \quad\wedge\quad \tenv'' = \tenv_1$: \\
  In this case we can use rule \mylabel{R-1} to establish:
\begin{equation*}
  R((\tenv_1,\rulet \leadsto x),\tenv_2;\alpha;\tenv';\tenv_1,\rulet \leadsto x;\tenv_1,\rulet \leadsto x;\sigma)
\end{equation*}
  which is equivalent to
\begin{equation*}
  R(\tenv;\alpha;\tenv';\tenv'',\rulet \leadsto x;\tenv'',\rulet \leadsto x;\sigma)
\end{equation*}

  \item $\tenv' = \tenv_1',\rulet \leadsto x, \tenv_2' \quad\wedge\quad \tenv'' = \tenv,\alpha,\tenv_1'$: \\
  In this case we can use rule \mylabel{R-2} to establish:
\begin{equation*}
  R(\tenv;\alpha;(\tenv_1',\rulet \leadsto x),\tenv_2';\tenv,\alpha,(\tenv_1',\rulet \leadsto x);
  \tenv,(\tenv_1',\rulet \leadsto x)[\suty/\alpha];\sigma)
\end{equation*}
  which is equivalent to
\begin{equation*}
  R(\tenv;\alpha;\tenv';\tenv'',\rulet \leadsto x;\tenv,(\tenv_1',\rulet \leadsto x)[\suty/\alpha];\sigma)
\end{equation*}
  \end{enumerate}

%===============================================================================
\item[\fbox{\texttt{(L-RuleNoMatch)}}]\quad$\bar{\alpha},\alpha,\bar{\alpha}'; \tenv,\alpha,\tenv'; \tenv'',\rulet \leadsto x \ivturns \type \leadsto E'$ \ \\
%===============================================================================
  The rule's first hypothesis states that
\begin{equation*}
  \not\exists \theta, E, \Sigma, \mathit{dom}(\theta) \subseteq (\bar{\alpha},\alpha,\bar{\alpha}'): \theta(\tenv,\alpha,\tenv'); \theta(\rulet)\leadsto x \ivturns \theta(\tau)\leadsto E; \Sigma
\end{equation*}
  Hence, the above also holds when we restrict $\theta$ to be of the form
$\theta' \cdot [\suty/\alpha]$. In this case, the above simplifies to
\begin{equation*}
  \not\exists \theta', E, \Sigma, \mathit{dom}(\theta) \subseteq (\bar{\alpha},\bar{\alpha}'): \theta'(\tenv,\tenv'[\suty/\alpha]); \theta'(\rulet[\suty/\alpha])\leadsto x \ivturns \theta'(\tau[\suty/\alpha])\leadsto E; \Sigma
\end{equation*}

  From the rule's second hypothesis and the induction hypothesis we have
\begin{equation*}
  \bar{\alpha},\bar{\alpha}';\tenv,\tenv'[\suty/\alpha];\tenv''' \ivturns \type[\suty/\alpha] \leadsto E'[|\suty|/\alpha]
\end{equation*}

  With rule \mylabel{L-RuleNoMatch} we combine these two observations into the desired first result
\begin{equation*}
  \bar{\alpha},\bar{\alpha}';\tenv,(\tenv',\rulet \leadsto x)[\suty/\alpha];\tenv''' \ivturns \type[\suty/\alpha] \leadsto E'[|\suty|/\alpha]
\end{equation*}

  Similarly, following the rule's second hypothesis and the induction hypothesis we have:
\begin{equation*}
  R(\tenv;\alpha;\tenv';\tenv'';\tenv''';\suty)
\end{equation*}

  We do a case analysis on the derivation of this judgement.
  \begin{enumerate}
  \item \mylabel{R-1}: \\ Then we have
\begin{equation*}
  \tenv = \tenv_1,x:\rulet,\tenv_2  \quad\wedge\quad \tenv'' = \tenv_1 \quad\wedge\quad \tenv''' = \tenv_1
\end{equation*}
       By rule \mylabel{R-2} we then have
\begin{equation*}
R((\tenv_1,x:\rulet),\tenv_2;\alpha;\tenv';\tenv_1,\rulet \leadsto x;\tenv_1,\rulet\leadsto x;\suty)
\end{equation*}
        which, given all the equations we have, is equivalent to
\begin{equation*}
R(\tenv;\alpha;\tenv';\tenv'',\rulet\leadsto x;\tenv'',\rulet\leadsto x;\suty)
\end{equation*}

  \item \mylabel{R-2}: \\
   Then we have
\begin{equation*}
  \tenv' = \tenv_1',\tenv_2'  \quad\wedge\quad \tenv'' = \tenv,\alpha,\tenv_1' \quad\wedge\quad 
      \tenv''' = \tenv,\tenv_1'[\suty/\alpha]
\end{equation*}

  Since $\tenv'',\rulet \leadsto x \subseteq \tenv,\alpha,\tenv'$, it follows that $\tenv_2' = \rulet\leadsto x,\tenv_{2,2}'$.
  Hence, by rule $\mylabel{R-2}$ we can establish
\begin{equation*}
R(\tenv;\alpha;(\tenv_1',\rulet\leadsto x),\tenv_2';\tenv,\alpha,(\tenv_1',\rulet \leadsto x);\tenv,(\tenv_1',\rulet\leadsto x)[\suty/\alpha];\suty)
\end{equation*}
        which, given all the equations we have, is equivalent to
\begin{equation*}
R(\tenv;\alpha;\tenv';\tenv'',\rulet\leadsto x;\tenv,(\tenv_1',\rulet \leadsto x)[\suty/\alpha];\suty)
\end{equation*}
  \end{enumerate}
%===============================================================================
\item[\fbox{\texttt{(L-Var)}}]\quad$\bar{\alpha},\alpha,\bar{\alpha}';\tenv,\alpha,\tenv';\tenv'', x : \rulet  \ivturns \type \leadsto E$ \ \\
%===============================================================================
  Then following the rule's hypothesis and the induction hypothesis we have:
\begin{equation*}
  \bar{\alpha},\bar{\alpha}';\tenv,\tenv'[\suty/\alpha];\tenv''' \ivturns \type[\suty/\alpha] \leadsto E[|\suty|/\alpha]
\end{equation*}
  By rule \mylabel{L-Var} and the definition of substitution we then have
\begin{equation*}
  \bar{\alpha},\bar{\alpha}';\tenv,\tenv'[\suty/\alpha];\tenv''',x:\rulet[\suty/\alpha] \ivturns \type[\suty/\alpha] \leadsto E[|\suty|/\alpha]
\end{equation*}

  Similarly, following the rule's hypothesis and the induction hypothesis we have:
\begin{equation*}
  R(\tenv;\alpha;\tenv';\tenv'';\tenv''';\suty)
\end{equation*}

  We do a case analysis on the derivation of this judgement.
  \begin{enumerate}
  \item \mylabel{R-1}: \\ Then we have
\begin{equation*}
  \tenv = \tenv_1,x:\rulet,\tenv_2  \quad\wedge\quad \tenv'' = \tenv_1 \quad\wedge\quad \tenv''' = \tenv_1
\end{equation*}
       By rule \mylabel{R-2} we then have
\begin{equation*}
R((\tenv_1,x:\rulet),\tenv_2;\alpha;\tenv';\tenv_1,x:\rulet;\tenv_1,x:\rulet;\suty)
\end{equation*}
        which, given all the equations we have, is equivalent to
\begin{equation*}
R(\tenv;\alpha;\tenv';\tenv'',x:\rulet;\tenv'',x:\rulet;\suty)
\end{equation*}

  \item \mylabel{R-2}: \\
   Then we have
\begin{equation*}
  \tenv' = \tenv_1',\tenv_2'  \quad\wedge\quad \tenv'' = \tenv,\alpha,\tenv_1' \quad\wedge\quad 
      \tenv''' = \tenv,\tenv_1'[\suty/\alpha]
\end{equation*}

  Since $\tenv'',x:\rulet \subseteq \tenv,\alpha,\tenv'$, it follows that $\tenv_2' = x:\rulet,\tenv_{2,2}'$.
  Hence, by rule $\mylabel{R-2}$ we can establish
\begin{equation*}
R(\tenv;\alpha;(\tenv_1',x:\rulet),\tenv_2';\tenv,\alpha,(\tenv_1',x:\rulet);\tenv,(\tenv_1',x:\rulet)[\suty/\alpha];\suty)
\end{equation*}
        which, given all the equations we have, is equivalent to
\begin{equation*}
R(\tenv;\alpha;\tenv';\tenv'',x:\rulet;\tenv,(\tenv_1',x:\rulet)[\suty/\alpha];\suty)
\end{equation*}
  \end{enumerate}


%===============================================================================
\item[\fbox{\texttt{(L-TyVar)}}]\quad$\bar{\alpha},\alpha,\bar{\alpha}';\tenv,\alpha,\tenv';\tenv'',\beta \ivturns \type \leadsto E$ \ \\
%===============================================================================
  Then following the rule's hypothesis and the induction hypothesis we have:
\begin{equation*}
  \bar{\alpha},\bar{\alpha}';\tenv,\tenv'[\suty/\alpha];\tenv''' \ivturns \type[\suty/\alpha] \leadsto E[|\suty|/\alpha]
\end{equation*}
  By rule \mylabel{L-TyVar} and the definition of substitution we then have
\begin{equation*}
  \bar{\alpha},\bar{\alpha}';\tenv,\tenv'[\suty/\alpha];\tenv''',\beta \ivturns \type[\suty/\alpha] \leadsto E[|\suty|/\alpha]
\end{equation*}

  Similarly, following the rule's hypothesis and the induction hypothesis we have:
\begin{equation*}
  R(\tenv;\alpha;\tenv';\tenv'';\tenv''';\suty)
\end{equation*}

  We do a case analysis on the derivation of this judgement.
  \begin{enumerate}
  \item \mylabel{R-1}: \\ Then we have
\begin{equation*}
  \tenv = \tenv_1,\tenv_2  \quad\wedge\quad \tenv'' = \tenv_1 \quad\wedge\quad \tenv''' = \tenv_1
\end{equation*}
  We further distinguish between two mutually exclusive cases:
  \begin{enumerate}
  \item $\tenv_2 = \epsilon$ \\
        It follows that $\alpha = \beta$ and we can establish by means of \mylabel{R-2} that
\begin{equation*}
R(\tenv_1,\tenv_2;\alpha;\epsilon,\tenv';\tenv_1,\tenv_2,\alpha;\tenv_1,\tenv_2,\epsilon[\suty/\alpha];\suty)
\end{equation*}
        which, given all the equations we have, is equivalent to
\begin{equation*}
R(\tenv;\alpha;\tenv';\tenv'',\beta;\tenv;\suty)
\end{equation*}

  \item $\tenv_2 \neq \epsilon$ \\
       Then it follows that $\tenv_2 = \beta,\tenv_{2,2}$ and by rule \mylabel{R-2} we have
\begin{equation*}
R((\tenv_1,\beta),\tenv_{2,2};\alpha;\tenv';\tenv_1,\beta;\tenv_1,\beta;\suty)
\end{equation*}
        which, given all the equations we have, is equivalent to
\begin{equation*}
R(\tenv;\alpha;\tenv';\tenv'',\beta;\tenv_1,\beta;\suty)
\end{equation*}
  \end{enumerate}

  \item \mylabel{R-2}: \\
   Then we have
\begin{equation*}
  \tenv' = \tenv_1',\tenv_2'  \quad\wedge\quad \tenv'' = \tenv,\alpha,\tenv_1' \quad\wedge\quad 
      \tenv''' = \tenv,\tenv_1'[\suty/\alpha]
\end{equation*}

  Since $\tenv'',\beta \subseteq \tenv,\alpha,\tenv'$, it follows that $\tenv_2' = \beta,\tenv_{2,2}'$.
  Hence, by rule $\mylabel{R-2}$ we can establish
\begin{equation*}
R(\tenv;\alpha;(\tenv_1',\beta),\tenv_2';\tenv,\alpha,(\tenv_1',\beta);\tenv,(\tenv_1',\beta)[\suty/\alpha];\suty)
\end{equation*}
        which, given all the equations we have, is equivalent to
\begin{equation*}
R(\tenv;\alpha;\tenv';\tenv'',\beta;\tenv,(\tenv_1',\beta)[\suty/\alpha];\suty)
\end{equation*}
  \end{enumerate}

\end{description}
\end{proof}

%###############################################################################
{\centering
\fbox{
\begin{minipage}{0.95\columnwidth}
\begin{lemma}\label{lemma:coherence:3}
  If 
\begin{equation*}
   \tenv,\alpha,\tenv'; \rulet \leadsto E \ivturns \type \leadsto E'; \bar{\rulet} \leadsto \bar{x}
\end{equation*}
  and
\begin{equation*}
  \tenv \turns \suty
\end{equation*}
  then
\begin{equation*}
   \tenv,\tenv'[\suty/\alpha]; \rulet[\suty/\alpha] \leadsto E[|\suty|/\alpha] \ivturns \type[\suty/\alpha] \leadsto E'[|\suty|/\alpha]; \bar{\rulet}[\suty/\alpha] \leadsto \bar{x}
\end{equation*}
\end{lemma}
\end{minipage}
}}

\begin{proof}
\begin{description}
\setlength{\itemsep}{1em}
%===============================================================================
\item[\fbox{\texttt{(M-Simp)}}]\quad$\tenv,\alpha,\tenv'; \type \leadsto E \ivturns \type \leadsto E; \epsilon$ \ \\
%===============================================================================
  The desired conclusion follows directly from rule \mylabel{M-Simp}
\begin{equation*}
  \tenv,\tenv'[\suty/\alpha]; \type[\suty/\alpha] \leadsto E[|\suty/\alpha|] \ivturns \type[\suty/\alpha] \leadsto E[|\suty/\alpha|]; \epsilon
\end{equation*}

%===============================================================================
\item[\fbox{\texttt{(M-IApp)}}]\quad$\tenv,\alpha,\tenv'; \rulet_1 \iarrow \rulet_2 \leadsto E \ivturns \type \leadsto E'; \Sigma, \rulet_1 \leadsto x$ \ \\
%===============================================================================
  From the rule's hypothesis and the induction hypothesis we have 
\begin{equation*}
  \tenv,(\tenv',\rulet_1 \leadsto x)[\suty/\alpha]; \rulet_2[\suty/\alpha] \leadsto (E x)[|\suty|/\alpha] \ivturns \type[\suty/\alpha] \leadsto E'[|\suty|/\alpha]; \Sigma[\suty/\alpha]
\end{equation*}

  Then from the definition of substutition and rule \mylabel{M-IApp} we conclude
\begin{equation*}
  \tenv,\tenv'[\suty/\alpha]; (\rulet_1 \iarrow \rulet_2)[\suty/\alpha] \leadsto E[|\suty|/\alpha] \ivturns \type[\suty/\alpha] \leadsto E'[|\suty|/\alpha]; (\Sigma,\rulet_1 \leadsto x)[\suty/\alpha]
\end{equation*}

%===============================================================================
\item[\fbox{\texttt{(M-TApp)}}]\quad$\tenv,\alpha,\tenv'; \forall\beta.\rulet \leadsto E \ivturns \type \leadsto E'; \Sigma$ \ \\
%===============================================================================
  From the rule's first hypothesis and the induction hypothesis we have 
\begin{equation*}
  \tenv,\tenv'[\suty/\alpha]; \rulet[\suty'/\beta][\suty/\alpha] \leadsto (E |\suty'|)[|\suty|/\alpha] \ivturns \type[\suty/\alpha] \leadsto E'[|\suty|/\alpha]; \Sigma[\suty/\alpha]
\end{equation*}

  From the rule's second hypothesis (and the preservation of well-typing under type-susbstitution) we have
\begin{equation*}
  \tenv,\tenv'[\suty/\alpha] \vdash \suty'[\suty'/\alpha]
\end{equation*}

  From these two facts we conclude by rule \mylabel{M-TApp}, reasoning modulo
  the definition of substitution
\begin{equation*}
  \tenv,\tenv'[\suty/\alpha]; (\forall\beta.\rulet)[\suty/\alpha] \leadsto E[|\suty|/\alpha] \ivturns \type[\suty/\alpha] \leadsto E'[|\suty|/\alpha]; \Sigma[\suty/\alpha]
\end{equation*}

\end{description}
\end{proof}

%-------------------------------------------------------------------------------
\subsection{Soundness of the Algorithm wrt Deterministic Resolution}

%###############################################################################
{\centering
\fbox{
\begin{minipage}{0.95\columnwidth}
\begin{lemma}\label{lemma:asoundness:0}
  If 
\begin{equation*}
  \tenv \alg \rulet \leadsto E
\end{equation*}
  then
\begin{equation*}
  \tenv \ivturns \rulet \leadsto E
\end{equation*}
\end{lemma}
\end{minipage}
}}

\begin{proof}
  From the hypothesis it follows that 
\begin{equation*}
  \mathit{tyvars}(\tenv);\tenv \alg \rulet \leadsto E
\end{equation*}
  Hence, by Lemma~\ref{lemma:asoundness:1} and rule \mylabel{R-Main} the desired conclusion follows
\begin{equation*}
  \tenv \ivturns \rulet \leadsto E
\end{equation*}
\end{proof}

%###############################################################################
{\centering
\fbox{
\begin{minipage}{0.95\columnwidth}
\begin{lemma}\label{lemma:asoundness:1}
  If 
\begin{equation*}
  \bar{\alpha};\tenv \alg \rulet \leadsto E
\end{equation*}
  then
\begin{equation*}
  \bar{\alpha};\tenv \ivturns \rulet \leadsto E
\end{equation*}
\end{lemma}
\end{minipage}
}}

\begin{proof}
The lemma follows from the isomorphism between the 
rule sets of the two judgements and from Lemma~\ref{lemma:asoundness:2}.
\end{proof}

%###############################################################################
{\centering
\fbox{
\begin{minipage}{0.95\columnwidth}
\begin{lemma}\label{lemma:asoundness:2}
  If 
\begin{equation*}
  \bar{\alpha};\tenv;\tenv' \alg \rulet \leadsto E
\end{equation*}
  then
\begin{equation*}
  \bar{\alpha};\tenv;\tenv' \ivturns \rulet \leadsto E
\end{equation*}
\end{lemma}
\end{minipage}
}}

\begin{proof}
The proof proceeds by induction on the derivation of the hypothesis.
\begin{description}
\setlength{\itemsep}{1em}
%===============================================================================
\item[\fbox{\texttt{(AL-RuleMatch)}}]\quad$\bar{\alpha};\tenv; \tenv', \rulet~{\leadsto x} \alg \type~{\leadsto E[\bar{E}/\bar{x}]}$ \ \\
%===============================================================================
From the rule's first hypothesis and Lemma~\ref{lemma:asoundness:3} we have
\begin{equation*}
\tenv; \rulet \leadsto E \ivturns \type \leadsto E'; \bar{\rulet} \leadsto \bar{x}
\end{equation*}
Then, using Lemma~\ref{lemma:asoundness:1} and rule \mylabel{L-RuleMatch} we conclude
\begin{equation*}
\bar{\alpha};\tenv; \tenv', \rulet~{\leadsto x} \ivturns \type~{\leadsto E[\bar{E}/\bar{x}]}
\end{equation*}

%===============================================================================
\item[\fbox{\texttt{(AL-RuleNoMatch)}}]\quad$\bar{\alpha};\tenv;\tenv', \rulet~{\leadsto x}\alg \type~{\leadsto E'}$ \ \\
%===============================================================================
  From the rule's second hypothesis and the induction hypothesis we have
\begin{equation*}
  \bar{\alpha};\tenv;\tenv' \ivturns \type~{\leadsto E'}
\end{equation*}

  Then from the rule's first hypothesis and the negation of Lemma~\ref{lemma:asoundness:5}, we have:
\begin{equation*}
\not\exists E, \Sigma: \bar{\alpha};\tenv;\rulet \leadsto x; \epsilon \alg \type \leadsto E; \Sigma
\end{equation*}
  By Lemma~\ref{lemma:asoundness:3} we thus have
\begin{equation*}
\not\exists \theta, E, \Sigma, \mathit{dom}(\theta) \subseteq \bar{\alpha}: \theta(\tenv);\theta(\rulet) \leadsto x \ivturns \type \leadsto E; \Sigma
\end{equation*}
  Hence with rule \mylabel{L-RuleNoMatch} we conclude
\begin{equation*}
  \bar{\alpha};\tenv;\tenv',\rulet \leadsto x \ivturns \type~{\leadsto E'}
\end{equation*}

%===============================================================================
\item[\fbox{\texttt{(AL-Var)}}]\quad$\bar{\alpha};\tenv;\tenv',x:\rulet \alg \type~{\leadsto E}$ \ \\
%===============================================================================
From the rule's hypothesis and the induction hypothesis we obtain
\begin{equation*}
  \bar{\alpha};\tenv;\tenv' \ivturns \type~{\leadsto E}
\end{equation*}
By rule \mylabel{L-Var} we conclude
\begin{equation*}
  \bar{\alpha};\tenv;\tenv',x:\rulet \ivturns \type~{\leadsto E}
\end{equation*}

%===============================================================================
\item[\fbox{\texttt{(AL-TyVar)}}]\quad$\bar{\alpha};\tenv;\tenv',\alpha \alg \type~{\leadsto E}$ \ \\
%===============================================================================
From the rule's hypothesis and the induction hypothesis we obtain
\begin{equation*}
  \bar{\alpha};\tenv;\tenv' \ivturns \type~{\leadsto E}
\end{equation*}
By rule \mylabel{L-TyVar} we conclude
\begin{equation*}
  \bar{\alpha};\tenv;\tenv',\alpha \ivturns \type~{\leadsto E}
\end{equation*}

\end{description}
\end{proof}

We assume that the judgement is decorated with an additional argument, the substitution
for the $\bar{\alpha}$ type variables.
%###############################################################################
{\centering
\fbox{
\begin{minipage}{0.95\columnwidth}
\begin{lemma}\label{lemma:asoundness:3}
  If 
\begin{equation*}
  \bar{\alpha};\tenv;\rulet \leadsto E; \Sigma \alg \type \leadsto E'; \Sigma',\theta(\Sigma); \theta
\end{equation*}
  and
\begin{equation*}
  \mathit{dom}(\theta) \subseteq \bar{\alpha}
\end{equation*}
  then
\begin{equation*}
  \theta(\tenv); \theta(\rulet) \leadsto |\theta|(E) \ivturns \theta(\type) \leadsto E'; \Sigma'
\end{equation*}
\end{lemma}
\end{minipage}
}}

\begin{proof}
The proof proceeds by induction on the derivation of the first hypothesis.
\begin{description}
\setlength{\itemsep}{1em}
%===============================================================================
\item[\fbox{\texttt{(AM-Simp)}}]\quad$\bar{\alpha}; \tenv; \type' \leadsto E; \Sigma \alg \type \leadsto |\theta|(E); \epsilon,\theta(\Sigma);\theta$ \ \\
%===============================================================================
From the hypothesis of the rule and Lemma~\ref{lemma:asoundness:6} it follows that
$\theta(\type') = \theta(\type)$. Hence, the target judgement can be rewritten as
\begin{equation*}
  \theta(\tenv); \theta(\type') \leadsto |\theta|(E) \ivturns \theta(\type') \leadsto |\theta|(E); \epsilon
\end{equation*}
This follows from rule \mylabel{M-Simp}.

%===============================================================================
\item[\fbox{\texttt{(AM-IApp)}}]\quad$\bar{\alpha}; \tenv; \rulet_1 \iarrow \rulet_2~{\leadsto E}; \Sigma \alg \type~{\leadsto E'}; \Sigma',\theta(\rulet_1) \leadsto x, \theta(\Sigma);\theta$ \ \\
%===============================================================================
  From the hypothesis of the rule and the induction hypothesis, we have that
\begin{equation*}
 \theta(\tenv,\rulet_1 \leadsto x); \theta(\rulet_2) \leadsto |\theta|(E\,x) \ivturns \theta(\type) \leadsto E'; \Sigma' 
\end{equation*}
  By rule \mylabel{M-IApp} we may then conclude
\begin{equation*}
 \theta(\tenv); \theta(\rulet_1 \iarrow \rulet_2) \leadsto |\theta|(E) \ivturns \theta(\type) \leadsto E'; \Sigma',\theta(\rulet_1) \leadsto x
\end{equation*} 

%===============================================================================
\item[\fbox{\texttt{(AM-TApp)}}]\quad$\bar{\alpha}; \tenv; \forall \alpha. \rulet~{\leadsto E}; \Sigma \alg \type~{\leadsto E'}; \Sigma',\theta(\Sigma);\theta$ \ \\
%===============================================================================
  Then it follows from the rule's hypothesis and from the induction hypothesis that
\begin{equation*}
  \theta(\tenv); \theta(\rulet) \leadsto |\theta|(E\,\alpha) \ivturns \theta(\type) \leadsto E'; \Sigma'
\end{equation*}
  Hence, it follows from rule \mylabel{M-TApp} that
\begin{equation*}
  \theta(\tenv); \theta(\forall \alpha.\rulet) \leadsto |\theta|(E) \ivturns \theta(\type) \leadsto E'; \Sigma'
\end{equation*}

\end{description}
\end{proof}


%###############################################################################
{\centering
\fbox{
\begin{minipage}{0.95\columnwidth}
\begin{lemma}\label{lemma:asoundness:5}
  If 
\begin{equation*}
  \bar{\alpha};\tenv;\rulet \leadsto E; \Sigma \alg \type \leadsto E'; \Sigma'
\end{equation*}
  then
\begin{equation*}
  \bar{\alpha}; \rulet \coh \type
\end{equation*}
\end{lemma}
\end{minipage}
}}

\begin{proof}
  The derivation of the conclusion is obtained by erasing the irrelevant arguments
  from the derivation of the hypothesis.
\end{proof}


%###############################################################################
{\centering
\fbox{
\begin{minipage}{0.95\columnwidth}
\begin{lemma}\label{lemma:asoundness:6}
  If 
\begin{equation*}
  \theta = \mgu{\type}{\type'}
\end{equation*}
  then
\begin{equation*}
  \theta(\type) = \theta(\type')
\end{equation*}
  and
\begin{equation*}
  \mathit{dom}(\theta) \subseteq \bar{\alpha}
\end{equation*}

\end{lemma}
\end{minipage}
}}

\begin{proof}
Straightforward induction on the derivation.
\end{proof}

%-------------------------------------------------------------------------------
\subsection{Completeness of the Algorithm wrt Deterministic Resolution}

%###############################################################################
{\centering
\fbox{
\begin{minipage}{0.95\columnwidth}
\begin{lemma}\label{lemma:acompleteness:0}
  If 
\begin{equation*}
  \tenv \ivturns \rulet \leadsto E
\end{equation*}
  then
\begin{equation*}
  \tenv \alg \rulet \leadsto E
\end{equation*}
\end{lemma}
\end{minipage}
}}

\begin{proof}
  From the hypothesis it follows that 
\begin{equation*}
  \mathit{tyvars}(\tenv);\tenv \ivturns \rulet \leadsto E
\end{equation*}
  Hence, by Lemma~\ref{lemma:acompleteness:1} and rule \mylabel{AR-Main} the desired conclusion follows
\begin{equation*}
  \tenv \alg \rulet \leadsto E
\end{equation*}
\end{proof}

%###############################################################################
{\centering
\fbox{
\begin{minipage}{0.95\columnwidth}
\begin{lemma}\label{lemma:acompleteness:1}
  If 
\begin{equation*}
  \bar{\alpha};\tenv \ivturns \rulet \leadsto E
\end{equation*}
  then
\begin{equation*}
  \bar{\alpha};\tenv \alg \rulet \leadsto E
\end{equation*}
\end{lemma}
\end{minipage}
}}

\begin{proof}
The lemma follows from the isomorphism between the 
rule sets of the two judgements and from Lemma~\ref{lemma:acompleteness:2}.
\end{proof}

%###############################################################################
{\centering
\fbox{
\begin{minipage}{0.95\columnwidth}
\begin{lemma}\label{lemma:acompleteness:2}
  If 
\begin{equation*}
  \bar{\alpha};\tenv;\tenv' \ivturns \rulet \leadsto E
\end{equation*}
  then
\begin{equation*}
  \bar{\alpha};\tenv;\tenv' \alg \rulet \leadsto E
\end{equation*}
\end{lemma}
\end{minipage}
}}

\begin{proof}
The proof proceeds by induction on the derivation of the hypothesis.
\begin{description}
\setlength{\itemsep}{1em}
%===============================================================================
\item[\fbox{\texttt{(L-RuleMatch)}}]\quad$\bar{\alpha};\tenv; \tenv', \rulet~{\leadsto x} \ivturns \type~{\leadsto E[\bar{E}/\bar{x}]}$ \ \\
%===============================================================================
From the rule's first hypothesis and Lemma~\ref{lemma:acompleteness:3} we have
\begin{equation*}
\epsilon; \tenv; \rulet \leadsto x; \epsilon \alg \type \leadsto E'; \bar{\rulet} \leadsto \bar{x}
\end{equation*}
Then, using Lemma~\ref{lemma:acompleteness:1} and rule \mylabel{AL-RuleMatch} we conclude
\begin{equation*}
\bar{\alpha};\tenv; \tenv', \rulet~{\leadsto x} \alg \type~{\leadsto E[\bar{E}/\bar{x}]}
\end{equation*}

%===============================================================================
\item[\fbox{\texttt{(L-RuleNoMatch)}}]\quad$\bar{\alpha};\tenv;\tenv', \rulet~{\leadsto x}\ivturns \type~{\leadsto E'}$ \ \\
%===============================================================================
  From the rule's second hypothesis and the induction hypothesis we have
\begin{equation*}
  \bar{\alpha};\tenv;\tenv' \alg \type~{\leadsto E'}
\end{equation*}

  From the rule's first hypothesis and the negation of Lemma~\ref{lemma:acompleteness:4}, we have:
\begin{equation*}
\bar{\alpha}; \rulet \not\coh \type
\end{equation*}
  Hence with rule \mylabel{AL-RuleNoMatch} we conclude
\begin{equation*}
  \bar{\alpha};\tenv;\tenv',\rulet \leadsto x \alg \type~{\leadsto E'}
\end{equation*}

%===============================================================================
\item[\fbox{\texttt{(L-Var)}}]\quad$\bar{\alpha};\tenv;\tenv',x:\rulet \ivturns \type~{\leadsto E}$ \ \\
%===============================================================================
From the rule's hypothesis and the induction hypothesis we obtain
\begin{equation*}
  \bar{\alpha};\tenv;\tenv' \alg \type~{\leadsto E}
\end{equation*}
By rule \mylabel{AL-Var} we conclude
\begin{equation*}
  \bar{\alpha};\tenv;\tenv',x:\rulet \alg \type~{\leadsto E}
\end{equation*}

%===============================================================================
\item[\fbox{\texttt{(L-TyVar)}}]\quad$\bar{\alpha};\tenv;\tenv',\alpha \ivturns \type~{\leadsto E}$ \ \\
%===============================================================================
From the rule's hypothesis and the induction hypothesis we obtain
\begin{equation*}
  \bar{\alpha};\tenv;\tenv' \alg \type~{\leadsto E}
\end{equation*}
By rule \mylabel{AL-TyVar} we conclude
\begin{equation*}
  \bar{\alpha};\tenv;\tenv',\alpha \alg \type~{\leadsto E}
\end{equation*}

\end{description}
\end{proof}

%###############################################################################
{\centering
\fbox{
\begin{minipage}{0.95\columnwidth}
\begin{lemma}\label{lemma:acompleteness:3}
  If 
\begin{equation*}
  \theta_1(\tenv); \theta_1(\rulet) \leadsto |\theta_1|(E) \ivturns \theta_1(\type) \leadsto |\theta_1|(E'); \theta_1(\Sigma')
\end{equation*}
  and
\begin{equation*}
  \mathit{dom}(\theta_1) \subseteq \bar{\alpha}
\end{equation*}
  then
\begin{equation*}
  \bar{\alpha};\tenv;\rulet \leadsto E; \Sigma \alg \type \leadsto |\theta_2|(E'); \theta_2(\Sigma',\Sigma)
\end{equation*}
  and
\begin{equation*}
  \mathit{dom}(\theta_2) \subseteq \bar{\alpha}
\end{equation*}
  and
\begin{equation*}
  \theta_1 \sqsubseteq \theta_2
\end{equation*}
\end{lemma}
\end{minipage}
}}

\begin{proof}
The proof proceeds by induction on the derivation of the first hypothesis.
\begin{description}
\setlength{\itemsep}{1em}
%===============================================================================
\item[\fbox{\texttt{(M-Simp)}}]\quad$\theta_1(\tenv); \theta_1(\type') \leadsto |\theta_1|(E) \ivturns \theta_1(\type) \leadsto |\theta_1|(E); \theta_1(\epsilon)$ \hfill where $\theta_1(\tenv) = \theta_1(\type')$. \\
%===============================================================================

From Lemma~\ref{lemma:acompleteness:5} and rule \mylabel{AM-Simp} we then have
\begin{equation*}
  \bar{\alpha}; \tenv; \type' \leadsto E; \Sigma \alg \type \leadsto \theta_2(E); \theta_2(\Sigma)
\end{equation*}

%===============================================================================
\item[\fbox{\texttt{(M-IApp)}}]\quad$\theta_1(\tenv); \theta_1(\rulet_1 \iarrow \rulet_2)~{\leadsto |\theta_1|(E)}\ivturns \theta_1(\type)~{\leadsto |\theta_1|(E')}; \theta_1(\Sigma',\rulet_1 \leadsto x)$ \ \\
%===============================================================================
  From the hypothesis of the rule and the induction hypothesis, we have that
\begin{equation*}
 \bar{\alpha};\tenv,\rulet_1 \leadsto x;\rulet_2 \leadsto E\,x; \rulet_1 \leadsto x, \Sigma \alg \type \leadsto |\theta_2|(E'); \theta_2(\Sigma',\rulet_1 \leadsto x, \Sigma) 
\end{equation*}
  By rule \mylabel{AM-IApp} we may then conclude
\begin{equation*}
 \bar{\alpha};\tenv;\rulet_1 \iarrow \rulet_2 \leadsto E; \Sigma \alg \type \leadsto |\theta_2|(E'); \theta_2(\Sigma',\rulet_1 \leadsto x, \Sigma) 
\end{equation*} 

%===============================================================================
\item[\fbox{\texttt{(M-TApp)}}]\quad$\theta_1(\tenv); \theta_1(\forall\alpha.\rulet) \leadsto |\theta_1|(E) \ivturns \theta_1(\type) \leadsto |\theta_1|(E'); \theta_1(\Sigma')$ \ \\
%===============================================================================
  From the hypothesis of the rule and the induction hypothesis, we have that
\begin{equation*}
 \bar{\alpha},\alpha;\tenv;\rulet \leadsto E\,\alpha; \Sigma \alg \type \leadsto |\theta_2|(E'); \theta_2(\Sigma',\Sigma) 
\end{equation*}
  Hence, it follows from rule \mylabel{AM-TApp} that
\begin{equation*}
 \bar{\alpha};\tenv;\forall \alpha.\rulet \leadsto E; \Sigma \alg \type \leadsto |\theta_2|(E'); \theta_2(\Sigma',\Sigma) 
\end{equation*}

\end{description}
\end{proof}

%###############################################################################
{\centering
\fbox{
\begin{minipage}{0.95\columnwidth}
\begin{lemma}\label{lemma:acompleteness:4}
  If 
\begin{equation*}
  \bar{\alpha}; \rulet \coh \type
\end{equation*}
  then for all $E, \tenv, \Sigma$ there exist $E', \Sigma'$ such that
\begin{equation*}
  \bar{\alpha};\tenv;\rulet \leadsto E; \Sigma \alg \type \leadsto E'; \Sigma'
\end{equation*}
\end{lemma}
\end{minipage}
}}

\begin{proof}
  The proof is straightforward induction on the derivation. The conclusion's judgement
  is an annotated version of the hypothesis' judgement.
\end{proof}

%###############################################################################
{\centering
\fbox{
\begin{minipage}{0.95\columnwidth}
\begin{lemma}\label{lemma:acompleteness:5}
  If 
\begin{equation*}
  \theta(\type) = \theta(\type')
\end{equation*}
  and
\begin{equation*}
  \mathit{dom}(\theta) \subseteq \bar{\alpha}
\end{equation*}
  then
\begin{equation*}
  \theta' = \mgu{\type}{\type'}
\end{equation*}
  and
\begin{equation*}
  \mathit{dom}(\theta') \subseteq \bar{\alpha}
\end{equation*}
  and
\begin{equation*}
  \theta \sqsubseteq \theta'
\end{equation*}
\end{lemma}
\end{minipage}
}}

% \newcommand{\vdashr}[4]{#1; #2 \vdash_r #3 \leadsto #4}
% 
% \begin{proof}
% The proof proceeds by induction on the $\bar{\alpha};\env \vdash_r \rho \leadsto E$ derivation.
% \begin{description}
% \renewcommand{\itemsep}{10mm}
% %= = = = = = = = = = = = = = = = = = = = = = = = = = = = = = = = = = = = = = = = 
% \item[\texttt{(R-IAbs)}] \quad
%     $\vdashr{\bar{\alpha}}{\env}{\rho_1 \iarrow \rho_2}{\lambda\relation{x}{||\rho_1||}.E}$ \ \\
% %= = = = = = = = = = = = = = = = = = = = = = = = = = = = = = = = = = = = = = = = 
% From rule \mylabel{R-IAbs} and the induction hypothesis, it follows that
% \begin{equation*}
% \vdashr{\bar{\alpha}}{\theta(\env,\rho_1\leadsto x)}{\theta(\rho_2)}{\theta(E)}
% \end{equation*}
% Hence, using rule \mylabel{R-IAbs}, we conclude
% \begin{equation*}
% \vdashr{\bar{\alpha}}{\theta(\env)}{\theta(\rho_1\iarrow\rho_2)}{\theta(\lambda\relation{x}{||\rho_1||}.E)}
% \end{equation*}
% %= = = = = = = = = = = = = = = = = = = = = = = = = = = = = = = = = = = = = = = = 
% \item[\texttt{(R-TAbs)}] \quad
%     $\vdashr{\bar{\alpha}}{\env}{\forall \alpha. \rho}{\Lambda\alpha.E}$ \ \\
% %= = = = = = = = = = = = = = = = = = = = = = = = = = = = = = = = = = = = = = = = 
% From rule \mylabel{R-TAbs} and the induction hypothesis, it follows that
% \begin{equation*}
% \vdashr{\bar{\alpha},\alpha}{\theta(\env)}{\theta(\rho)}{\theta(E)}
% \end{equation*}
% Hence, using rule \mylabel{R-TAbs}, we conclude
% \begin{equation*}
% \vdashr{\bar{\alpha}}{\theta(\env)}{\theta(\forall \alpha. \rho)}{\theta(\Lambda\alpha.E)}
% \end{equation*}
% %= = = = = = = = = = = = = = = = = = = = = = = = = = = = = = = = = = = = = = = = 
% \item[\texttt{(R-Simp)}] \quad
%     $\vdashr{\bar{\alpha}}{\env}{\tau}{E}$ \ \\
% %= = = = = = = = = = = = = = = = = = = = = = = = = = = = = = = = = = = = = = = = 
% We distinguish two cases. Firstly, we assume
% that $\theta(\tau)$ yields the simple type $\tau'$.
% Then, it follows from rule \mylabel{R-Simp} and Lemma~\ref{} (TODO) that
% \begin{equation*}
% \elookup{\theta(\env)}{\theta(\tau)} = \theta(\rho)~\leadsto x
% \end{equation*}
% Similarly, it follows from rule \mylabel{R-Simp} and Lemma~\ref{} (TODO) that
% \begin{equation*}
% \bar{\alpha};\theta(\env); \theta(\rho)~\leadsto x
%    \turns_\downarrow \theta(\type)~\leadsto \theta(E)
% \end{equation*}
% Hence, using rule \mylabel{R-Simp} we conclude
% \begin{equation*}
% \vdashr{\bar{\alpha}}{\theta(\env)}{\theta(\tau)}{\theta(E)}
% \end{equation*}
% 
% Secondly, we assume that $\theta(\tau) = \forall \bar{\beta}.\tau'$.
% This happens only if $\tau = \alpha$.
% In this case, we can show:
% \begin{equation*}
% \vdashr{\bar{\alpha}}{\theta(\env)}{\theta(\tau)}{E'} \quad \wedge \quad E' \equiv_{\alpha,H} \theta(E)
% \end{equation*}
% if we take $E' = \Lambda\bar{\beta}.\theta(E)\,\bar{\beta}$
% by repeated application of rule \mylabel{R-TAbs} and showing that
% \begin{equation*}
% \vdashr{\bar{\alpha},\bar{\beta}}{\theta(\env)}{\tau'}{\theta(E)\,\bar{\beta}}
% \end{equation*}
% 
% %   \item $\theta(\tau) = \tau'$
% %   \item $\theta(\tau) = \forall \bar{\alpha'}.\tau'$
% \end{description}
% \end{proof}


% \begin{lemma}\label{lemma:substitution:lhd}
% The $\rho \lhd \tau$ judgement is stable under substitution.
% \[\forall \rho, \tau, \alpha, \suty: 
%     \rho \lhd \tau \quad \rightarrow \quad \theta(\rho) \lhd \head{\theta(\tau)}
% \]
% where $\theta = [\suty/\alpha]$.
% \end{lemma}
% \begin{proof}
% The proof proceeds by induction on the $\rho \lhd \tau$ derivation.
% \begin{description}
% \renewcommand{\itemsep}{10mm}
% %= = = = = = = = = = = = = = = = = = = = = = = = = = = = = = = = = = = = = = = = 
% \item[\texttt{(M-IApp)}] \quad
%     $\rho_1 \iarrow \rho_2 \lhd \tau$ \ \\
% %= = = = = = = = = = = = = = = = = = = = = = = = = = = = = = = = = = = = = = = = 
% From rule \mylabel{M-IApp} and the induction hypothesis it follows that:
% \begin{equation*}
% \theta(\rho_2) \lhd \head{\theta(\tau)}
% \end{equation*}
% Then it follows from rule \mylabel{M-IApp} that:
% \begin{equation*}
% \theta(\rho_1) \iarrow \theta(\rho_2) \lhd \head{\theta(\tau)}
% \end{equation*}
% Hence, because $\theta(\rho_1) \iarrow \theta(\rho_2) = \theta(\rho_1 \iarrow \rho_2)$.
% we conclude:
% \begin{equation*}
% \theta(\rho_1 \iarrow \rho_2) \lhd \head{\theta(\tau)}
% \end{equation*}
% 
% %= = = = = = = = = = = = = = = = = = = = = = = = = = = = = = = = = = = = = = = = 
% \item[\texttt{(M-TApp)}] \quad
%     $\forall \alpha. \rho \lhd \tau$ \ \\
% %= = = = = = = = = = = = = = = = = = = = = = = = = = = = = = = = = = = = = = = = 
% From rule \mylabel{M-TApp} and the induction hypothesis it follows that:
% \begin{equation*}
% \theta(\rho[\suty'/\alpha']) \lhd \head{\theta(\tau)}
% \end{equation*}
% We can commute the two substitutions as follows because their
% domains are disjoint:
% \begin{equation*}
% \theta(\rho)[\theta(\suty')/\alpha'] \lhd \head{\theta(\tau)}
% \end{equation*}
% Then, by rule \mylabel{M-TApp}, we get:
% \begin{equation*}
% \forall \alpha. \theta(\rho) \lhd \head{\theta(\tau)}
% \end{equation*}
% Finally, because $\forall \alpha. \theta(\rho) = \theta(\forall \alpha. \rho)$,
% we conclude:
% \begin{equation*}
% \theta(\forall \alpha. \rho) \lhd \head{\theta(\tau)}
% \end{equation*}
% 
% %= = = = = = = = = = = = = = = = = = = = = = = = = = = = = = = = = = = = = = = = 
% \item[\texttt{(M-Simp)}] \quad
%     $\tau \lhd \tau$ \ \\
% %= = = = = = = = = = = = = = = = = = = = = = = = = = = = = = = = = = = = = = = = 
% We distinguish three cases:
% Firstly, we consider the case where $\tau = \rho_1 \arrow \rho_2$.
% Then, $\theta(\tau) = \head{\theta(\tau)}$ by rule \mylabel{M-Simp}
% we trivially have:
% \begin{equation*}
% \theta(\tau) \lhd \head{\theta(\tau)}
% \end{equation*}
% 
% Secondly, we consider the case where $\tau = \alpha' \neq \alpha$.
% Then, $\head(\theta(\tau)) = \head{\tau} = \tau$ and by rule 
% \mylabel{M-Simp} we trivally have:
% \begin{equation*}
% \theta(\tau) \lhd \head{\theta(\tau)}
% \end{equation*}
% 
% Thirdly, we consider the case where $\tau = \alpha$. Then $\theta(\tau) =
% \suty$.  We have that $\suty$ is of the general form $\forall
% \bar{\beta}.\tau'$. Hence, $\head{\theta(\tau)} = \tau'$.
% Then we can establish the desired goal:
% \begin{equation*}
% \forall \bar{\beta}.\tau' \lhd \tau'
% \end{equation*}
% by repeated application of rule \mylabel{M-TApp}
% choosing substitutions of the form $[\beta/\beta]$
% and the trivial base case $\tau' \lhd \tau'$ shown by rule
% \mylabel{M-Simp}.
% 
% \end{description}
% \end{proof}
% 
% \begin{lemma}\label{lemma:substitution:lookup}
% The $\elookup{\env}{\type} = \rho \leadsto x$ judgement is stable under substitution.
% \[\forall \env, \bar{\alpha}, \type, \rho, \alpha, \suty: 
%     \elookup{\env}{\type} = \rho \leadsto x
%     \quad\rightarrow\quad
%     \elookup{\theta(\env)}{\head{\theta(\type)}} = \theta(\rho) \leadsto x
% \]
% where $\theta = [\suty/\alpha]$.
% \end{lemma}
% \begin{proof}
% The proof proceeds by induction on the $\elookup{\env}{\type} = \rho \leadsto x$ derivation.
% \begin{description}
% \renewcommand{\itemsep}{10mm}
% %= = = = = = = = = = = = = = = = = = = = = = = = = = = = = = = = = = = = = = = = 
% \item[\texttt{(L-Head)}] \quad
%     $\elookup{(\env,\rho \leadsto x)}{\type} = \rho \leadsto x$ \ \\
% %= = = = = = = = = = = = = = = = = = = = = = = = = = = = = = = = = = = = = = = = 
% From rule \mylabel{L-Head} it follows that
% \begin{equation*}
% \rho \lhd \tau
% \end{equation*}
% Hence, from Lemma~\ref{lemma:substitution:lhd}, we know that:
% \begin{equation*}
% \theta(\rho) \lhd \head{\theta(\type)}
% \end{equation*}
% Thus we obtain with rule \mylabel{L-Head}:
% \begin{equation*}
% \elookup{(\theta(\env),\theta(\rho) \leadsto x)}{\head{\theta(\type)}} = \theta(\rho) \leadsto x
% \end{equation*}
% As $(\theta(\env),\theta(\rho) \leadsto x) = \theta(\env,\rho \leadsto x)$,
% this yields the desired result:
% \begin{equation*}
% \elookup{\theta(\env,\rho \leadsto x)}{\head{\theta(\type)}} = \theta(\rho) \leadsto x
% \end{equation*}
% 
% %= = = = = = = = = = = = = = = = = = = = = = = = = = = = = = = = = = = = = = = = 
% \item[\texttt{(L-Tail)}] \quad
%     $\elookup{(\env,\rho_1 \leadsto x)}{\type} = \rho_2 \leadsto y$ \ \\
% %= = = = = = = = = = = = = = = = = = = = = = = = = = = = = = = = = = = = = = = = 
% From rule \mylabel{L-Tail} and the induction hypothesis it follows that
% \begin{equation*}
% \elookup{\theta(\env)}{\head{\theta(\type)}} = \theta(\rho_2) \leadsto y
% \end{equation*}
% 
% Also from rule \mylabel{L-Tail} we have
% \begin{equation*}
% \forall \theta': 
% 	\theta'(\rulet_1) \mathop{\not\!\!\lhd} \head{\theta'(\type)}
% \end{equation*}
% As a consequence this also holds for every choice $\theta' = \theta'' \cdot \theta$.
% \begin{equation*}
% \forall \theta'': 
% 	\theta''(\theta(\rulet_1)) \mathop{\not\!\!\lhd} \head{\theta''(\theta(\type))}
% \end{equation*}
% Now we apply the property that $\mathrm{hd} \cdot \theta'' = \mathrm{hd} \cdot \theta'' \cdot \mathrm{hd}$:
% \begin{equation*}
% \forall \theta'': 
% 	\theta''(\theta(\rulet_1)) \mathop{\not\!\!\lhd} \head{\theta''(\head{\theta(\type)})}
% \end{equation*}
% We now have all the necessary ingredients to apply rule \mylabel{L-Tail}
% and to conclude:
% \begin{equation*}
% \elookup{(\theta(\env),\theta(\rho_1) \leadsto x)}{\head{\theta(\type)}} = \theta(\rho_2) \leadsto y
% \end{equation*}
% Finally, as $(\theta(\env),\theta(\rho_1) \leadsto x) = \theta(\env, \rho_1
% \leadsto x)$, we obtain the desired result:
% \begin{equation*}
% \elookup{\theta(\env,\rho_1 \leadsto x)}{\head{\theta(\type)}} = \theta(\rho_2) \leadsto y
% \end{equation*}
% 
% \end{description}
% \end{proof}


% History dates
%\received{February 2007}{March 2009}{June 2009}

\end{document}
