\section{Conclusion}
\label{sec:conclusion}

This paper presented $\ourlang$, the Calculus Of CoHerent ImplicitS, a new
calculus for implicit programming that improves upon the implicit calculus and
strikes a good balance between flexibility and coherence. In particular,
$\ourlang$ supports local scoping, overlapping rules, first-class
rules, and higher-order rules, while remaining type safe, coherent and
unambiguous.
Interesting future work includes integrating Genus's solution for the
instance coherence problem~\cite{Zhang15LFO} in $\ourlang$; and 
adding more features that show up 
in various IP mechanisms, such as \emph{associated types}~\cite{assoctypes,assoctypes2} and \emph{type
  families}~\cite{typefunc}.

%%In particular, 
%%resolution for any types local and nested scoping for implicits 
 
%%features for
%%generic programming, nested scoping and resolution of any types.

%%We present a formal type system for the calculus and provide semantics via a
%%translation to System F. We prove that the translation preserves types and thus
%%establish type soundness for $\ourlang$.

%%The calculus provides a formal platform for the development of
%%a realistic source language. Our small source language already shows 
%%how to add implicit instantiation on top of the calculus. In further
%%work we intend to develop this into a full-fledged language.



% integrates key features in GP, while at the same
% time being minimalistic. The calculus supports nested scoping,
% overlapping rules, higher-order rules and rule resolutions. 
%
% and encodings of existing GP practice. 
% 
% In the calculus, two key features -- scoping and resolution -- of many
% GP mechanisms were put in the spotlight, and we
% offered answers to two challenging, and essentially unsolved issues in
% the literature: how to support coherence for overlapping rules in
% nested scoping; and how to support resolution with higher-order
% rules. Such properties of scoping and resolution have not received
% appropriate attention in the past. 
% 
% of GP, which has been an
% increasingly popular trend in programming languages such as Haskell, Scala,
% Java, C++ and Agda. 


%%Both of these shortcommings hinder the development of applications 
%%that rely on sophisticated implicit programming mechanisms, as
%%document previously in the literature~\cite{derivable,}.

%Future work includes the design of a source language with a powerful 
%modularity construct (which could be a form of modules, OO classes 
%or records with associated types) to model ``type-class'' style
%interfaces. Type inference and additional source-level constructs on top of
%$\ourlang$ also deserve further attention.

%%% Local Variables: 
%%% mode: latex
%%% TeX-master: "../Main"
%%% End: 


