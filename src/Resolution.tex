
\section{Resolution with Overlapping Rules}
\label{sec:resolution}

An essential property of resolution is \textit{coherence}, which ensures that
the result of resolution is predictable and consistent. Incoherent programs,
where resolution cannot be resolved unambiguously, should be rejected.

%- - - - - - - - - - - - - - - - - - - - - - - - - - - - - - - - - - - - - - - - 
\subsection{Overlap}
Overlapping rules are a well-known source of incoherence. Consider
\begin{equation*}
\left\{ \forall \alpha. \alpha \rightarrow \tyint,  
        \forall \alpha. \tyint \rightarrow \alpha \right\} \vturns \tyint \rightarrow \tyint
\end{equation*}
Both the first and the second rule can be used to discharge this judgement.
In general, we have incoherence when there are multiple possible resolution derivations.

In order to avoid incoherence, existing approaches in the literature take a the
Draconian measure of forbidding overlap altogether. However, overlap
is a highly useful features, as illustrated by the many practical
applications in GHC/Haskell.\footnote{despite the fact that Haskell
implementations do not truly rule out incoherence}
Hence, we adopt a less conservative approach.

%- - - - - - - - - - - - - - - - - - - - - - - - - - - - - - - - - - - - - - - - 
\subsection{Overlap Resolution}
To avoid the potential incoherence, we provide a two-level priority scheme to
specify which one of several overlapping rules is selected during resolution. 

The first priority is the natural priotirization in the implicit environment.
This environment behaves as a stack of sets. A rule in a set closer to the top
of the stack has priority over one further down in the stack. For instance,
\begin{equation*}
\left\{ \forall \alpha. \alpha \rightarrow \tyint \right\},  
\left\{ \forall \alpha. \tyint \rightarrow \alpha \right\} \vturns \tyint \rightarrow \tyint
\end{equation*}
is uniquely resolved using the second rule, which is at the top of the stack.

The second level, in case of
overlapping rules within the same set in the implicit environment, is to
select the most specific matching rule. For instance,
\begin{equation*}
\left\{ \forall \alpha. \alpha \rightarrow \alpha,  
        \forall \alpha. \alpha \rightarrow \tyint \right\} \vturns \tyint \rightarrow \tyint
\end{equation*}
is uniquely resolved using the second rule, which is more specific than the first.

While this approach allows us to make effective use of overlapping rules in many
cases, we must still avoid situations where no most specific rule can be determined,
such as the first example.
This overlap resolution also raises concerns with respect to type safety. Consider:
\begin{equation*}
\left\{ \forall \alpha. \alpha \rightarrow \alpha,  
        \{\textit{Char}\} \To  \tyint \rightarrow \tyint \right\} \vturns \beta \rightarrow \beta
\end{equation*}
Regardless of whether $\beta$ is \tyint or not, there is a rule to select
in the implicit environment. However, when $\beta$ is \tyint, resolution
gets stuck, because there is no rule to produce a value of type \texttt{Char}.
In contrast, resolution does not get stuck for other instantiations.
As we wish to claim that well-typed programs do not get stuck, we need to make
sure that situations like this do not happen. 

%- - - - - - - - - - - - - - - - - - - - - - - - - - - - - - - - - - - - - - - -
\subsection{Condition Specifications} 

In order to avoid unsafe and incoherent overlap in a ruleset, we impose three
conditions.

%~~~
\paragraph{Most Specific Rule}
Firstly, there must always be a most specific rule.
\begin{definition}[Existence of Most Specific Rule Condition]
We say that a ruleset $\pi$ always has a most specific rule iff
\[
\begin{array}{l}
  \forall \theta.\forall \tau.\rho_1,\rho_2\in\pi.
  \theta|\rho_1| \succeq \theta\tau \wedge \theta|\rho_2| \succeq \theta\tau \\
  \implies \exists \rho \in \pi. \theta|\rho| = \theta(|\rho_1| \sqcap |\rho_2|)
\end{array}
\]
We use $\mathit{exists}(\pi)$ to denote such ruleset.
\end{definition}

Note that the order between types can be affected by type type substitution.
A pair of types on which no order relationship exists may have an order after 
applying substiution. So we have to consider every possible substitutions.
See following example.
\[
\{
  \forall \alpha. \alpha \rightarrow {\tt int} \rightarrow a,
  \forall \alpha. {\tt int} \rightarrow \alpha  \rightarrow {\tt int} 
\} 
\vturns {\tt int} \rightarrow {\tt int} \rightarrow {\tt int}
\]

The resolution will be ambiguous if type variable $a$ is substituted with ${\tt int}$.

%~~~
\paragraph{Uniqueness}
Secondly, the most specific instance must be unique. This avoids situations
where multiple rules \textit{statically} have the same result type.
\begin{equation*}
\left\{ \texttt{int}, \texttt{char} \Rightarrow \texttt{int}, \texttt{char} \right\} 
\vturns \texttt{int}
\end{equation*}
However, we should also consider the effect of a type substitution, and make
sure that uniqueness remains true \textit{dynamically}. Consider this case:
\begin{equation*}
\left\{ \alpha, \tyint \right\} \vturns \tyint
\end{equation*}
It statically appears that the second is the only appropriate one to resolve
\tyint. However, at runtime, the type variable $\alpha$ may well be
instantiated to \tyint, which would mean that there are actually two
implicit \tyint values.

The following condition makes sure of both static and dynamic uniqueness.
\begin{definition}[Uniqueness of Instances]
We say that the ruleset $\pi$ is consistent with unique instances iff 
\[
\forall \rho_1,\rho_2 \in \pi, \theta. 
  \rho_1 \neq \rho_2
  \implies \theta |\rho_1| \neq \theta |\rho_2|
\]
We use $\mathit{unique}(\pi)$ to denote such ruleset.
\end{definition}

%~~~
\paragraph{Type Safety}
We must ensure that every resolution that succeeds for a general type also
succeeds for a more specific type.
\begin{definition}[Type Safety Condition]
An implicit environment $\env$ is type safe iff
\begin{equation*}
\forall \rulet, \theta  . \env \vturns \rulet \wedge \mathit{dom}(\theta) = \mathit{fv}(\rulet) \cup \mathit{fv}(\env)
~~\Longrightarrow~~ 
\theta\env \vturns \theta\rulet
\end{equation*}
\end{definition}

We may need restricted algorithmic check. Naively restricting specific instances to
have less constraint then general instances will not work. Assume we have the most
general pretty printer that prints $``\dots''$ for any input and polymorphic pretty
printer for lists which takes a pretty printer for an elemetns type implicitly.
A program having such pretty printers is natural but it will be rejected by naive
restriction 

\tom{Algorithm}


%~~~
\paragraph{Coherent Environment} We now combine the above conditions
on rulesets and extend them to environments.
\begin{theorem}[Coherent Environment]
An environment $\env$ has a unique most specific instance for any query if
every stack element of the environment has unique most specific instances.
\tom{formal statement}
\end{theorem}
\wt{PROOF}

%-------------------------------------------------------------------------------
\subsection{Static Condition Checking}

We have formulated coherence conditions on the implicit environment, but not
yet stated where in the type system these conditions are imposed.

One approach is to impose the conditions throughout, on every 
environment and ruleset in all type system rules. This approach not only burdens the
formulation of the type system, it also unnecessarily rejects many useful
programs, like the following rule\footnote{similar to 
an \texttt{instance (C a, C b) => C (a,b)} of a type class \texttt{C} in Haskell}
\begin{equation*}
f \stackrel{\mathit{def}}{\equiv}\ruleabs{(\forall a,b.\{a,b\} \Rightarrow a \times b)}{(?a,?b)}
\end{equation*}
where $\{a,b\}$ violates the uniqueness condition (as $\theta a = \theta b$ for $\theta = [a \mapsto \type, b \mapsto \type]$). It would be a shame if we could
not use this rule in safe situations, like
\begin{equation*}
\ruleapp{f[\tyint,\texttt{Bool}]}{\{1:\tyint,\texttt{true}:\texttt{Bool}\}}
\end{equation*}
In the same vain, we would like to accept
\begin{equation*}
\ruleapp{f[\tyint,\tyint]}{\{1:\tyint\}}
\end{equation*}
where the substitution collapses the set of required implicit rules to a
singleton.
In order to avoid rejecting these programs, we must not check for uniqueness
at rule abstraction sites (\textbf{rule}), but defer them to a \textit{later} point.
In particular, we do want to reject programs like
\begin{equation*}
g \stackrel{\mathit{def}}{\equiv}
\ruleabs{(\forall a.\{a\} \Rightarrow a \times \tyint)}
      {\ruleapp
        {f[a,\tyint]}
        {\{?a:a,3:\tyint\}}}
\end{equation*}
which dynamically does not have a unique most general rule in a context like 
$\ruleapp{ g[\tyint]}{\{ 4 : \tyint\}}$. Deferring the check beyond
$g$'s definition leaves us with too little information. Hence, we must perform
the existence check at rule application (\textbf{with}).

Finally, we need another existence check at resolution ($?\rulet$). Consider
the program
\[\begin{array}{l}
\tstate \vturns 
\ruleapp{(?(\forall a.\{a\}\Rightarrow a \times \texttt{int}))[\texttt{int}]}
  {\{4:\texttt{int}\}}
\end{array}\]
where $\tstate = \{f:\forall a,b.\{a,b\}\Rightarrow (a, b) ,3:\texttt{int}\}$.
In this program, the two \texttt{int} values are supplied by different means to $f$,
$4$ through the \texttt{with} construct, but $3$ through recursive resolution. Hence,
the latter also requires a check.

\paragraph{Summary}
Our type system checks:  
\textit{existence} of most specific instances at rule definitions (\texttt{rule}), and
\textit{uniqueness} of instances at resolution ($?\rulet$) and rule application (\texttt{with}).
% If the ruleset in rule signatures statically ensures the existence of most specific instances, 
% then the runtime environment will always have most specific instances.
% Furthermore, if a static check prevents every attempt to make an environment with duplicate
% entries at rule application and sites, then we will always have unambiguous environments at runtime.
% There are only two such positions : rule applications and resolutions.

\begin{definition}[Coherent Program]
Program $e$ with type $\tau$ is coherent iff the following conditions hold throughout
the derivation tree of $\epsilon \turns e:\tau$.
\begin{itemize}
\item 
  For all subexpression $\ruleabs{\forall \overline{\alpha}.\pi\Rightarrow \tau}{e}$ :
    $\texttt{exists}(\pi)$
\item
  For all subexpression $\ruleapp{e}{\overline{e}:\overline{\rho}}$ :
    \mbox{$\texttt{unique}$}$(\overline{\rho})$
\item
  For all sub-derivations 
  $\Gamma \turns \forall \overline{\alpha}. \pi_1 \Rightarrow \tau$ :
    Assume $\Gamma(\forall \overline{\alpha}.\tau) 
    = \forall \overline{\beta}.\pi_2 \Rightarrow \tau_2$
    and $\theta$ is a substitution such that $\theta \tau_2 = \tau_1$.
    The the followings should hold : $\texttt{unique}(\theta\pi_2 - \pi_1)$
    and $\forall \rho_1 \in \pi_1.\rho_2 \in \pi_2. \theta\rho_2 \nsucceq \rho_1$.
\end{itemize}
\end{definition}


% Statically checking an argument at rule application sites is enough 
% to reject following ambiguous program. 
% \[\begin{array}{l}
% \texttt{let } f=\ruleabs{(\forall a,b.\{a,b\} \Rightarrow a \times b)}{(?a,?b)} \\
% \texttt{in let } g=
%   {\ruleabs{(\forall a.\{a\} \Rightarrow a \times \texttt{int})}
%     {\ruleapp
%       {f[a,\texttt{int}]}
%       {\{?a,3\}}}} \\
% \texttt{in } \ruleapp{g[\texttt{int}]}{\{4\}}
% \end{array}\]

% Static check at a resolution site can also prevent following ambiguous program.
% Let $I = \{(?a,?b):\forall a,b.\{a,b\}\Rightarrow a \times b  ,3:\texttt{int}\}$
% \[\begin{array}{l}
% I \vturns 
% \ruleapp{(\forall a.\{a\}\Rightarrow a \times \texttt{int})[\texttt{int}]}
%   {\{4:\texttt{int}\}}
% \end{array}\]

% Our checking algorithm can accept following case since we delay unambiguouty check to
% the use point.
% \[
% \ruleapp
%   {(\ruleabs{\forall a,b.\{a,b\} \Rightarrow (a,b)}{(?a,?b)})
%     [\texttt{int},\texttt{int}]}
%   {\{3:\texttt{int}\}}
% \]

\wt{THEOREM : Type Safe Program Never Generates Ambiguous Environment}
\tom{The theorem should just be type saftey.}
\wt{Mentioning other possible options}

%\begin{definition}[Most Specific Rule Condition]
%We say that an environment $\env$ always has a most specific rule iff
%\begin{multline*}
%\forall \tau.
%\forall \rho_1, 
%        \rho_2 \in \env^*(\tau).
%        \exists (\rulesch{\alpha_3}{\ruleset_3}{\tau_3}) \in \env^{*}(\tau).\\
%        \{\rho_1,\rho_2\} \subseteq \env^*(\tau_3)
%\end{multline*}
%where
%\begin{equation*}
%\left\{\begin{array}{rcl}
%(\env,\bar{\rho})^*(\tau) & = & \{(\rulesch{\alpha'}{\ruleset'}{\tau'}) \in \bar{\rho} \mid \tau \preceq_{\bar{\alpha}'} \tau' \} \lhd \env^*(\tau)\\
%\epsilon^*(\tau)          & = & \emptyset 
%\end{array}\right.
%\end{equation*}
%with
%\begin{equation*}
%\left\{\begin{array}{rcl}
%\emptyset \lhd \bar{\rho}_2      & = & \bar{\rho}_2\\
%\bar{\rho}_1 \lhd \bar{\rho}_2   & = & \bar{\rho}_1 \hspace{1cm}, \rho_1 \neq \emptyset
%\end{array}\right.
%\end{equation*}
%and
%begin{equation*}
%
%\tau_1 \preceq_{\bar{\alpha}} \tau_2 \stackrel{\mathit{def}}{\Longleftrightarrow}  
%\exists \bar{\tau}. [\bar{\alpha} \mapsto \bar{\tau}]\tau_2 = \tau_1
%\end{equation*}
%\end{definition}
%
%\tom{define lookup}
%
%
%The above condition only makes sure that every most step in the recursive
%resolution process either fails or succeeds deterministically.
%
%% and
%% \begin{multline*}
%% \forall (\rulesch{\alpha_1}{\kappa_1}{\ruleset_1}{\tau_1}), 
%%         (\rulesch{\alpha_2}{\kappa_2}{\ruleset_2}{\tau_2}) \in \ruleset.\\
%%         \tau_2 \preceq \tau_1 \wedge
%%         \tau_1 \preceq \tau_2 \Longrightarrow \\
%%         \rulesch{\alpha_1}{\kappa_1}{\ruleset_1}{\tau_1} = 
%%         \rulesch{\alpha_2}{\kappa_2}{\ruleset_2}{\tau_2}
%% \end{multline*}
%
%

%- - - - - - - - - - - - - - - - - - - - - - - - - - - - - - - - - - - - - - - - 
\subsection{Operational versus Elaboration Semantics}
Our informal notion of coherent overlap always allows us to select
the appropriate implicit rule \textit{at runtime}, which is perfect
for the operational semantics.
Not so for the elaboration semantics, which (by definition) has to select the
appropriate rule statically. For instance,
\begin{equation*}
\left\{ \forall \alpha. \alpha \rightarrow \alpha,  
        \tyint \rightarrow \tyint \right\} \vturns \beta \rightarrow \beta
\end{equation*}
Statically, we do not know whether $\beta$ equals \tyint or not.  For the
operational semantics this is fine: statically we only check that there is a
unique resolution in both cases. At runtime we will know which of these to
chose, as the type variable $\beta$ will be instantiated.
However, the elaboration semantics must chose at compile time, and thus cannot
resolve this situation.  

As a consequence, the elaboration semantics is forced to accept fewer programs
than the operational semantics.  In particular, if we insist that the implicit
environment is able to resolve every possible query coherently, then the
elaboration semantics cannot tolerate any overlap at all. A more liberal
approach only considers whether the queries $\env \vturns \rulet$ that actually
appear in the program (originating from the $?\rulet$ construct) are resolved
coherently. GHC takes a similar approach for overlapping type class instances,
yet coherence is not guaranteed across multiple modules. Note that only
checking coherence for program queries also makes sense for the operational
semantics.



%%% Local Variables: 
%%% mode: latex
%%% TeX-master: "../Main"
%%% End: 
