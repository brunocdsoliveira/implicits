%% ODER: format ==         = "\mathrel{==}"
%% ODER: format /=         = "\neq "
%
%
\makeatletter
\@ifundefined{lhs2tex.lhs2tex.sty.read}%
  {\@namedef{lhs2tex.lhs2tex.sty.read}{}%
   \newcommand\SkipToFmtEnd{}%
   \newcommand\EndFmtInput{}%
   \long\def\SkipToFmtEnd#1\EndFmtInput{}%
  }\SkipToFmtEnd

\newcommand\ReadOnlyOnce[1]{\@ifundefined{#1}{\@namedef{#1}{}}\SkipToFmtEnd}
\usepackage{amstext}
\usepackage{amssymb}
\usepackage{stmaryrd}
\DeclareFontFamily{OT1}{cmtex}{}
\DeclareFontShape{OT1}{cmtex}{m}{n}
  {<5><6><7><8>cmtex8
   <9>cmtex9
   <10><10.95><12><14.4><17.28><20.74><24.88>cmtex10}{}
\DeclareFontShape{OT1}{cmtex}{m}{it}
  {<-> ssub * cmtt/m/it}{}
\newcommand{\texfamily}{\fontfamily{cmtex}\selectfont}
\DeclareFontShape{OT1}{cmtt}{bx}{n}
  {<5><6><7><8>cmtt8
   <9>cmbtt9
   <10><10.95><12><14.4><17.28><20.74><24.88>cmbtt10}{}
\DeclareFontShape{OT1}{cmtex}{bx}{n}
  {<-> ssub * cmtt/bx/n}{}
\newcommand{\tex}[1]{\text{\texfamily#1}}	% NEU

\newcommand{\Sp}{\hskip.33334em\relax}


\newcommand{\Conid}[1]{\mathit{#1}}
\newcommand{\Varid}[1]{\mathit{#1}}
\newcommand{\anonymous}{\kern0.06em \vbox{\hrule\@width.5em}}
\newcommand{\plus}{\mathbin{+\!\!\!+}}
\newcommand{\bind}{\mathbin{>\!\!\!>\mkern-6.7mu=}}
\newcommand{\rbind}{\mathbin{=\mkern-6.7mu<\!\!\!<}}% suggested by Neil Mitchell
\newcommand{\sequ}{\mathbin{>\!\!\!>}}
\renewcommand{\leq}{\leqslant}
\renewcommand{\geq}{\geqslant}
\usepackage{polytable}

%mathindent has to be defined
\@ifundefined{mathindent}%
  {\newdimen\mathindent\mathindent\leftmargini}%
  {}%

\def\resethooks{%
  \global\let\SaveRestoreHook\empty
  \global\let\ColumnHook\empty}
\newcommand*{\savecolumns}[1][default]%
  {\g@addto@macro\SaveRestoreHook{\savecolumns[#1]}}
\newcommand*{\restorecolumns}[1][default]%
  {\g@addto@macro\SaveRestoreHook{\restorecolumns[#1]}}
\newcommand*{\aligncolumn}[2]%
  {\g@addto@macro\ColumnHook{\column{#1}{#2}}}

\resethooks

\newcommand{\onelinecommentchars}{\quad-{}- }
\newcommand{\commentbeginchars}{\enskip\{-}
\newcommand{\commentendchars}{-\}\enskip}

\newcommand{\visiblecomments}{%
  \let\onelinecomment=\onelinecommentchars
  \let\commentbegin=\commentbeginchars
  \let\commentend=\commentendchars}

\newcommand{\invisiblecomments}{%
  \let\onelinecomment=\empty
  \let\commentbegin=\empty
  \let\commentend=\empty}

\visiblecomments

\newlength{\blanklineskip}
\setlength{\blanklineskip}{0.66084ex}

\newcommand{\hsindent}[1]{\quad}% default is fixed indentation
\let\hspre\empty
\let\hspost\empty
\newcommand{\NB}{\textbf{NB}}
\newcommand{\Todo}[1]{$\langle$\textbf{To do:}~#1$\rangle$}

\EndFmtInput
\makeatother
%

\section{Overview} %%of Rule Calculus $\ourlang$}
\label{sec:overview}

In this section, we motivate and give an overview of the rule calculus and the
source language described in Section~\cite{}. 

%%The rule calculus is
%%presented as a simple logic programming language of Horn clauses
%%and the relation to existing implicit programming mechanisms is
%%discussed. Then it is shown how a source language built on top 
%%of the rule calculus can model 

\subsection{Rule Calculus $\ourlang$ Syntax}

\begin{figure}

\begin{center}
  
  \[ \begin{array}{llrl}
    \text{Types} & \type  & ::=  & \alpha \mid \rulet \\
    \text{Rule Types} & \rulet  & ::= & 
    \rulesch{\alpha}{\rulesetvar}{\type} \\
    \text{Context} & \rulesetvar  & ::= & \bar{\rulet} \\
    \text{Expressions} & e & ::=  & 
    ?\rulet \mid 
    \ruleabs{\rulet}{e} \mid
    e[\overline{\type}] \mid
    \ruleapp{e}{\rulesetexp} \\
  \end{array} \]
\end{center}

\caption{Rule calculus syntax}

\end{figure}

\subsection{Rules and Rule Types}

The key notion of the rule calculus is that of a rule and its
corresponding rule type. Rule types are 

Relation to type classes; well-known relation between 
type classes and logic programming.

\paragraph{Horn clauses as rule types}

\Ex{
  1) \quad \{\} \To \tyint \\
  2) \quad \{\tybool\} \To \tyint \\
  3) \quad \{a \to a \to \tybool \} \To [a] \to [a] \to [a]  \\
  4) \quad \{\Eq~a\} \To [a] \to [a] \to [a] \\
  5) \quad \{\Eq~a, \Eq~b\} \To \Eq~(a,b)   
}

\paragraph{Polymorphic and higher-order rules}

\subsection{Resolution and Queries}

\paragraph{Higher-order queries}

\subsection{Scoping}

\paragraph{Local Scoping}

\paragraph{Nested Scoping}

\paragraph{Overlapping}

\subsection{Source Language and Type-Inference}

present $\ourlang$ and its core features using
some examples. The formal system of the calculus is presented in
Section~\ref{sec:calculus}. In the rest of this section, we assume
that the calculus is conservatively extended to have constructs of
$\lambda$-calculus and several types.

$\ourlang$ is a simple logic programming language of Horn clauses. We
can compose Horn clauses and resolve a clause under the given set of
clauses. Consider the following example:
\\
\Ex{
  \relation{f}{\myset{\tyint} \To \tyint} \\
  f = \ruleabs{(\myset{\tyint} \To \tyint)}{(?\tyint + 1)} \\
}
\\
In the example, $f$ is a {\it rule} whose type corresponds to Horn
clause $\tyint \To \tyint$. The clause, which we call {\it rule type},
means that $f$ will give us an integer value if there is an integer
clause in the assumptions. We call the current set of assumptions as
{\it rule environment}. In the body, the rule tries to resolve query
$?\tyint$ which essentially asks whether the rule type $\tyint$ can be
deduced from the given assumptions. Since $f$ assumes that there must
be a rule type $\tyint$, the query can be resolved. If we apply this
rule to a rule of type $\tyint$, we can get the final value like this:
\\
\Ex{ \ruleapp{f}{\myset{\relation{1}{\tyint}}} }
\\
This program will give us 2 as a result.

The rule type can be polymorphic in $\ourlang$. Consider the following
example:
\\
\Ex{
  \relation{g}{\myset{\forall \alpha. \alpha \to \alpha} \To 
    \tyint \to \tyint} \\
  g = \ruleabs{(\myset{\forall \alpha. \alpha \to \alpha} \To 
    \tyint \to \tyint)}{(?\tyint \to \tyint)} \\
}
\\
The query $?\tyint \to \tyint$ can be resolved by instantiating the
assumption $\forall \alpha. \alpha \to \alpha$ for the $\tyint$
type. 

In the rest of the section, we explain the key features that
differentiate our calculus from the others. Note that some of these
features have already existed in the previous systems but none of them
provides the combination of all.

\subsection{Flexible Implicit Values}
\label{subsec:flexible}

We allow arbitrary types in rule types. This is different from type
class-like systems. The type class-like systems can be modeled as
simple logic language with Horn clauses of predicates on
types. Rather, our system is closer to Scala implicits~\cite{scala}.

Using arbitrary types for the assumptions has more flexibility than
using only predicates on types. Because of this limitation, some
Haskell programs try to encode natural number using type-level Church
numeral to do the very same thing like this:

\[  
\begin{array}{l}
  \relation{f}{\myset{\tyint} \To \tyint} \\ 
  f = \ruleabs{(\myset{\tyint} \To \tyint)}{(?\tyint + 1)}
\end{array}
\]
\[  
\begin{array}{l}
  \ruleapp{f}{\myset{\relation{\ruleabs{(\varnothing \To
          \tyint)}{1}}{\varnothing \To \tyint}}}
\end{array}
\]

\subsection{Lexical Scoping}
\label{subsec:lexical}

We support lexical scoping of rule environments. Consider the
following example:
\\
\[  
\begin{array}{l}
  \ruleapp
  {\ruleapp{f}{\myset{\relation{2}{\tyint}}}}
  {\myset{\relation{1}{\tyint}}}
\end{array}
\]
\\
The rule $f$ is now put under two environments. 

Under the nesting, we choose the closest one from the given rule.

\subsection{Overlapping}
\label{subsec:overlapping}

We allow overlapping of rules. 

\[
\begin{array}{l}
  \relation{f}{\forall \alpha. \alpha \to \alpha} \\
  f = 
  \ruleapp
  {(\ruleabs
    {(\forall \alpha.
      \myset{\forall \beta. \beta \to \beta, \tyint \to \tyint}
      \To \alpha \to \alpha)
    }
    {(?\alpha \to \alpha)})\\\quad\quad}
  {\myset{
      \relation{\lambda x. x + 1}{\tyint \to \tyint},
      \relation{\lambda x. x}{\forall \gamma. \gamma \to \gamma}
    }}
\end{array}
\]

%%% Local Variables: 
%%% mode: latex
%%% TeX-master: "../Main"
%%% End: 
