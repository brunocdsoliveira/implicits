\figtwocol{fig:ftype}{System F Type System}{
\begin{center}
\framebox{
\begin{minipage}{\textwidth}
\bda{lc}
\multicolumn{2}{c}{\myruleform{\Gamma \turns T}} \\ \\
  (\texttt{F-WF-VarTy}) & 
\myirule{
           \alpha \in \Gamma
 }{
            \Gamma \turns \alpha
} \\ \\
  (\texttt{F-WF-FunTy}) & 
\myirule{
            \Gamma \turns T_1 \quad\quad \Gamma \turns T_2
 }{
            \Gamma \turns T_1 \arrow T_2
} \\ \\
  (\texttt{F-WF-AbsTy}) & 
\myirule{
            \Gamma, \alpha \turns T
 }{
            \Gamma \turns \forall \alpha.T
} \\ \\
\multicolumn{2}{c}{\myruleform{\Gamma \turns E : T}} \\ \\
  (\texttt{F-Var}) & 
\myirule{
           (x : T) \in \Gamma
 }{
            \Gamma \turns x : T
} \\ \\

  (\texttt{F-Abs}) & 
\myirule{
           \Gamma, x : T_1 \turns E : T_2 \quad\quad \Gamma \turns T_1
 }{
           \Gamma \turns \lambda x:T_1.E : T_1 \rightarrow T_2
} \\ \\

  (\texttt{F-App}) & 
\myirule{
  \Gamma \turns E_1 : T_2 \rightarrow T_1 \quad\quad
           \Gamma \turns E_2 : T_2
          }{
           \Gamma \turns E_1 \, E_2 : T_1
} \\ \\

  (\texttt{F-TApp}) & 
\myirule{
  \Gamma \turns E : \forall \alpha. T_2 \quad\quad \Gamma \turns T_1
           }{
            \Gamma \turns E \, T_1 : T_2[T_1/\alpha]
} \\ \\

  (\texttt{F-TAbs}) & 
\myirule{
   \Gamma, \alpha \turns E : T
            }{
             \Gamma \turns \Lambda \alpha.E : \forall \alpha. T 
} \\ \\
\eda
\end{minipage}
}
\end{center}
}

%###############################################################################
\section{Proofs}

Throughout the proofs we refer to the type system rules of System F listed
in Figure~\ref{fig:ftype}.

%-------------------------------------------------------------------------------
\subsection{Type Preservation}\label{proof:preservation}

Lemma~\ref{lemma:tp:0} states that the translation preserves the well-formedness of types. 

{\centering
\fbox{
\begin{minipage}{0.95\columnwidth}
\begin{lemma}\label{lemma:tp:0}
  If 
\begin{equation*}
    \tenv \turns \rulet
\end{equation*}
  then
\begin{equation*}
    |\tenv| \turns |\rulet|
\end{equation*}
\end{lemma}
\end{minipage}
}}

\begin{proof}
By structural induction on the expression and corresponding inference rule.
\begin{description}
\renewcommand{\itemsep}{10mm}
%===============================================================================
\item[\fbox{\texttt{(WF-VarTy)}}\quad$\tenv \turns \alpha$] \ \\
%===============================================================================
  It follows from the rule that $\alpha \in \tenv$. Hence, obviously $\alpha
  \in |\tenv|$. Finally, by rule \mylabel{F-WF-VarTy}, and taking into account
  that $|\alpha| = \alpha$, we conclude
\begin{equation*}
  |\tenv| \turns \alpha
\end{equation*}

%===============================================================================
\item[\fbox{\texttt{(WF-FunTy)}}\quad$\tenv \turns \rulet_1 \arrow \rulet_2$] \ \\
%===============================================================================
  It follows from the induction hypotheses and the hypotheses of the rule that
\begin{equation*}
  |\tenv| \turns |\rulet_1| \quad \wedge \quad |\tenv| \turns |\rulet_2|
\end{equation*}

  Hence, by rule \mylabel{F-WF-FunTy}, and taking into account that
  $|\rulet_1| \arrow |\rulet_2| = |\rulet_1 \arrow \rulet_2|$, we conclude that
\begin{equation*}
  |\tenv| \turns |\rulet_1 \arrow \rulet_2|
\end{equation*}

%===============================================================================
\item[\fbox{\texttt{(WF-RulTy)}}\quad$\tenv \turns \rulet_1 \iarrow \rulet_2$] \ \\
%===============================================================================
  It follows from the induction hypotheses and the hypotheses of the rule that
\begin{equation*}
  |\tenv| \turns |\rulet_1| \quad \wedge \quad |\tenv| \turns |\rulet_2|
\end{equation*}

  Hence, by rule \mylabel{F-WF-FunTy}, and taking into account that
  $|\rulet_1| \arrow |\rulet_2| = |\rulet_1 \iarrow \rulet_2|$, we conclude that
\begin{equation*}
  |\tenv| \turns |\rulet_1 \iarrow \rulet_2|
\end{equation*}

%===============================================================================
\item[\fbox{\texttt{(WF-AbsTy)}}\quad$\tenv \turns \forall\alpha.\rulet$] \ \\
%===============================================================================
  It follows from the induction hypothesis and the hypothesis of the rule that
\begin{equation*}
  |\tenv,\alpha| \turns |\rulet|
\end{equation*}
  As $|\tenv,\alpha| = |\tenv|,\alpha$, we can simplify this to
\begin{equation*}
  |\tenv|,\alpha \turns |\rulet|
\end{equation*}

  Hence, by rule \mylabel{F-WF-AbsTy}, and taking into account that
  $\forall\alpha.|\rulet| = |\forall\alpha.\rulet|$, we conclude that
\begin{equation*}
  |\tenv| \turns |\forall\alpha.\rulet|
\end{equation*}

\end{description}
\end{proof}


Lemma~\ref{lemma:tp:1} states that the translation of expressions to System F preserves
types. Its proof relies on Lemma~\ref{lemma:tp:2}, which states that the translation
of resolution preserves types.

{\centering
\fbox{
\begin{minipage}{0.95\columnwidth}
\begin{lemma}\label{lemma:tp:1}
  If 
\begin{equation*}
    \tenv \turns e : \rho \leadsto E
\end{equation*}
  then
\begin{equation*}
    |\tenv| \turns E : |\rho|
\end{equation*}
\end{lemma}
\end{minipage}
}}

\begin{proof}
By structural induction on the expression and corresponding inference rule.
\begin{description}
\renewcommand{\itemsep}{10mm}
%===============================================================================
\item[\fbox{\texttt{(Ty-Var)}}\quad$\tenv \turns x : \rulet \leadsto x$] \ \\
%===============================================================================

 It follows from \mylabel{Ty-Var} that 
\begin{equation*} 
    (x : \rulet) \in \tenv
\end{equation*} 

Based on the definition of $|\cdot|$  it follows 
\begin{equation*} 
   (x : |\rulet|) \in |\tenv| 
\end{equation*} 

Thus we have by (\texttt{F-Var}) that
\begin{equation*} 
   |\tenv| \turns x : |\rulet|
\end{equation*} 

%===============================================================================
\item[\fbox{\texttt{(Ty-Abs)}}\quad$\tenv \turns \lambda x:\rho_1.e : \rho_1 \rightarrow \rho_2 \leadsto \lambda x:|\rho_1|.E$] \ \\
%===============================================================================

  The first hypothesis of (\texttt{Ty-Abs}) is that
\begin{equation*} 
    \tenv, x : \rho_1 \turns e : \rho_2 \leadsto E
\end{equation*} 
  and thus by the induction hypothesis we have that
\begin{equation*} 
    |\tenv|, x : |\rho_1| \turns E : |\rho_2|
\end{equation*} 

  The second hypothesis of (\texttt{Ty-Abs}) is that
\begin{equation*} 
    \tenv \turns |\rho_1|
\end{equation*} 
  and thus by Lemma~\ref{lemma:tp:0} we have that
\begin{equation*} 
    |\tenv| \turns |\rho_1|
\end{equation*} 

  Hence, by \mylabel{F-Abs} we conclude 
\begin{equation*} 
    |\tenv| \turns \lambda x:|\rho_1|.E : |\rho_1 \rightarrow \rho_2|
\end{equation*} 

%===============================================================================
\item[\fbox{\texttt{(Ty-App)}}\quad$\tenv \turns e_1\,e_2 : \rulet_1 \leadsto E_1\,E_2$] \ \\
%===============================================================================

  By the induction hypothesis, we have:
\begin{equation*} 
   |\tenv| \turns E_1 : |\rulet_2 \rightarrow \rulet_1| \quad\wedge\quad |\tenv| \turns E_2 : |\rulet_2|
\end{equation*} 
  and, because $|\rulet_2 \arrow \rulet_1| = |\rulet_2| \arrow |\rulet_1|$, we can write the former as
\begin{equation*} 
   |\tenv| \turns E_1 : |\rulet_2| \arrow |\rulet_1|
\end{equation*} 

  Then it follows by (\texttt{F-App}) that
\begin{equation*} 
   |\tenv| \turns E_1\, E_2 : |\rulet_1|
\end{equation*} 

%===============================================================================
\item[\fbox{\texttt{(Ty-TAbs)}}\quad$\tenv \turns \Lambda \alpha.e : \forall \alpha. \rulet \leadsto \Lambda \alpha.E$]\ \\
%===============================================================================

  Based on (\texttt{Ty-TAbs}) and the induction hypothesis, we have
\begin{equation*} 
    |\tenv, \alpha| \turns E : |\rulet|
\end{equation*}

  Thus, based on (\texttt{F-TAbs}) and because $|\tenv,\alpha| = |\tenv|,\alpha$, we have
\begin{equation*} 
    |\tenv| \turns \Lambda \alpha. E : \forall \alpha.|\rulet|
\end{equation*}
  or, because $|\forall\alpha.\rulet|=\forall\alpha.|\rulet|$, we conclude
\begin{equation*} 
    |\tenv| \turns \Lambda \alpha. E : |\forall \alpha.\rulet|
\end{equation*}

%===============================================================================
\item[\fbox{\texttt{(Ty-TApp)}}]\quad$\tenv \turns e\,\rulet_1 : \rulet_2[\rulet_1/\alpha] \leadsto E\,|\rulet_1|$\ \\
%===============================================================================
 
  By the first hypothesis of the rule and the induction hypothesis of the
  lemma, it follows that
\begin{equation*} 
    |\tenv| \turns E : |\forall \alpha.\rulet_2|
\end{equation*} 
  From this we have by definition of $|\cdot|$
\begin{equation*} 
    |\tenv| \turns E : \forall{\alpha}.|\rulet_2|
\end{equation*} 

  By the second hypothesis of the rule and Lemma~\ref{lemma:tp:0} we also have
\begin{equation*} 
    |\tenv| \turns |\rulet_1|
\end{equation*} 

  It then follows from (\texttt{F-TApp}) that
\begin{equation*} 
    |\tenv| \turns E\,|\rulet_1| : |\rulet_2|[|\rulet_1|/\alpha]
\end{equation*} 
  This is easily seen to be equivalent to 
\begin{equation*} 
    |\tenv| \turns E\,|\rulet_1| : |\rulet_2[\rulet_1/\alpha]|
\end{equation*} 

%===============================================================================
\item[\fbox{\texttt{(Ty-IAbs)}}\quad$\tenv \turns \ilambda \rulet_1.e : \rulet_1 \iarrow \rulet_2 \leadsto \lambda x:|\rulet_1|.E$]\ \\
%===============================================================================

  Based on the first hypothesis of the rule and the induction hypothesis, we have
\begin{equation*} 
    |\tenv, \rulet_1 \leadsto x| \turns E : |\rulet_2|
\end{equation*}
  or, using the definition of $|\cdot|$,
\begin{equation*} 
    |\tenv|, x : |\rulet_1| \turns E : |\rulet_2|
\end{equation*}

  Based on the second hypothesis of the rule and Lemma~\ref{lemma:tp:0} we have
\begin{equation*} 
    |\tenv| \turns |\rulet_1|
\end{equation*}

  Thus, based on (\texttt{F-Abs}) we have
\begin{equation*} 
    |\tenv| \turns \lambda x : |\rulet_1|.E : |\rulet_1| \arrow |\rulet_2|
\end{equation*}
  or, using the definition of $|\cdot|$ again,
\begin{equation*} 
    |\tenv| \turns \lambda x : |\rulet_1|.E : |\rulet_1 \iarrow \rulet_2|
\end{equation*}
 
%===============================================================================
\item[\fbox{\texttt{(Ty-IApp)}}\quad$\tenv \turns e_1 \with e_2 : \rulet_1 \leadsto E_1\,E_2$] \ \\
%===============================================================================

  From the hypotheses of the rule and the induction hypothesis we have:
\begin{equation*}
    |\tenv| \turns E_1 : |\rulet_2 \iarrow \rulet_1| \quad\wedge\quad |\tenv| \turns E_2 : |\rulet_2|
\end{equation*}

  Based on the definition of $|\cdot|$, the first of these means 
\begin{equation*}
    |\tenv| \turns E_1 : |\rulet_2| \arrow |\rulet_1|
\end{equation*}

  Finally, based on (\texttt{F-App}), we know
\begin{equation*}
    |\tenv| \turns E_1\,E_2 : |\rulet_1|
\end{equation*}

%===============================================================================
\item[\fbox{\texttt{(Ty-Query)}}\quad$\tenv \turns ?\rulet : \rulet \leadsto E$] \ \\
%===============================================================================

  Based on the first hypothesis of the rule and Lemma~\ref{lemma:resolution} we know
\begin{equation*} 
    |\tenv| \turns E : |\rulet|
\end{equation*} 

\end{description}
\end{proof}

%###############################################################################
{\centering
\fbox{
\begin{minipage}{0.95\columnwidth}
\begin{lemma}\label{lemma:resolution}\label{lemma:tp:2}
  If 
\begin{equation*}
    \tenv \vdash_r \rulet \leadsto E
\end{equation*}
  and
\begin{equation*}
    \tenv \turns \rulet 
\end{equation*}
  then
\begin{equation*}
    |\tenv| \turns E : |\rulet|
\end{equation*}
\end{lemma}
\end{minipage}
}}

\begin{proof}

  By induction on the derivation.
\begin{description}
%===============================================================================
\item[\fbox{\texttt{(R-TAbs)}}\quad$\tenv \vdash_r \forall \alpha.\rulet \leadsto \Lambda \alpha.E$] \ \\
%===============================================================================

  From the hypothesis of the rule and the induction hypothesis, we have
\begin{equation*}
  |\tenv, \alpha| \vdash E : |\rulet|
\end{equation*}
  or alternatively, based on the definition of $|\cdot|$,
\begin{equation*}
  |\tenv|, \alpha \vdash E : |\rulet|
\end{equation*}
 
  Then, rule \mylabel{F-TAbs} allows us to conclude
\begin{equation*}
  |\tenv| \vdash \Lambda \alpha. E : \forall \alpha.|\rulet|
\end{equation*}
  or, again based on the definition of $|\cdot|$,
\begin{equation*}
  |\tenv| \vdash \Lambda \alpha. E : |\forall \alpha.\rulet|
\end{equation*}

%===============================================================================
\item[\fbox{\texttt{(R-TApp)}}]\quad$\tenv \vdash_r \rulet [\suty/\alpha] \leadsto E\,|\suty|$ \ \\
%===============================================================================

  From the first hypothesis of the rule and the induction hypothesis, we have
\begin{equation*}
  |\tenv| \vdash E : |\forall \alpha. \rulet|
\end{equation*}
  or alternatively, based on the definition of $|\cdot|$,
\begin{equation*}
  |\tenv| \vdash E : \forall \alpha. |\rulet|
\end{equation*}

  From the second hypothesis of the rule and Lemma~\ref{lemma:tp:0}, we have
\begin{equation*}
  |\tenv| \vdash |\suty|
\end{equation*}

  Then, rule \mylabel{F-TApp} allows us to conclude
\begin{equation*}
  |\tenv| \vdash E\,|\suty| : |\rulet|[|\suty|/\alpha]
\end{equation*}
  or, again based on the definition of $|\cdot|$,
\begin{equation*}
  |\tenv| \vdash E\,|\suty| : |\rulet[\suty/\alpha]|
\end{equation*}

%===============================================================================
\item[\fbox{\texttt{(R-IVar)}}]\quad$\tenv \vdash_r \rulet \leadsto x$ \ \\
%===============================================================================

  From the hypothesis of the rule and the definition of $|\cdot|$, we have
\begin{equation*}
  (x : |\rulet|) \in |\tenv|
\end{equation*}

  Thus, using rule \mylabel{F-Var}, we can conclude
\begin{equation*}
  |\tenv| \vdash x : |\rulet|
\end{equation*}

%===============================================================================
\item[\fbox{\texttt{(R-IAbs)}}]\quad$\tenv \vdash_r \rulet_1 \iarrow \rulet_2 \leadsto \lambda x:|\rulet_1|.E$ \ \\
%===============================================================================

  From the first hypothesis of the rule and the induction hypothesis, we have
\begin{equation*}
  |\tenv, \rulet_1 \leadsto x| \vdash E : |\rulet_2|
\end{equation*}
  or alternatively, based on the definition of $|\cdot|$,
\begin{equation*}
  |\tenv|, x : |\rulet_1| \vdash E : |\rulet_2|
\end{equation*}

  Then, rule \mylabel{F-Abs} allows us to conclude
\begin{equation*}
  |\tenv| \vdash \lambda x:|\rulet_1|.E : |\rulet_1| \arrow |\rulet_2|
\end{equation*}

  or, again based on the definition of $|\cdot|$,
\begin{equation*}
  |\tenv| \vdash \lambda x:|\rulet_1|.E : |\rulet_1 \iarrow \rulet_2|
\end{equation*}

%===============================================================================
\item[\fbox{\texttt{(R-IApp)}}]\quad$\tenv \vdash_r \rulet_2 \leadsto E_2\,E_1$ \ \\
%===============================================================================

  From the hypotheses of the rule and the induction hypothesis, we have
\begin{equation*}
  |\tenv| \vdash E_1 : |\rulet_1| \quad\wedge\quad
  |\tenv| \vdash E_2 : |\rulet_1 \iarrow \rulet_2|
\end{equation*}

  The second conjunct can be reformulated, based on the definition of $|\cdot|$, to
\begin{equation*}
  |\tenv| \vdash E_2 : |\rulet_1| \arrow |\rulet_2|
\end{equation*}

  Then, rule \mylabel{F-App} allows us to conclude
\begin{equation*}
  |\tenv| \vdash E_2\,E_1 : |\rulet_2|
\end{equation*}

\end{description}
\end{proof}


%-------------------------------------------------------------------------------
\subsection{Auxiliary Lemmas About Non-Determistic Resolution}

The non-deterministic resolution judgement enjoys a number of
typical binder-related properties.

The first lemma is the weakening lemma: that states that an extended context
preserves all the derivations of the original context.
%###############################################################################
{\centering
\fbox{
\begin{minipage}{0.95\columnwidth}
\begin{lemma}[Weakening]\label{lemma:weakening}
  If 
\begin{equation*}
  \tenv, \tenv' \vturns \rulet \leadsto E 
\end{equation*}
  then
\begin{equation*}
    \tenv, \tenv'', \tenv' \vturns \rulet \leadsto E
\end{equation*}
\end{lemma}
\end{minipage}
}}

\begin{proof}
The proof proceeds by straightfoward induction on the derivation of the hypothesis.
\end{proof}

The second lemma is the substitution lemma which states that
we can drop an axiom from the context if it is already implied
by the remainder of the context.

%###############################################################################
{\centering
\fbox{
\begin{minipage}{0.95\columnwidth}
\begin{lemma}[Substitution]\label{lemma:substitution}
  If 
\begin{equation*}
  \tenv, \rulet \leadsto x, \tenv' \vturns \rulet' \leadsto E'
\end{equation*}
  and
\begin{equation*}
  \tenv \vturns \rulet \leadsto E 
\end{equation*}
  then
\begin{equation*}
    \tenv, \tenv' \vturns \rulet' \leadsto E'[E/x]
\end{equation*}
\end{lemma}
\end{minipage}
}}

\begin{proof}
The proof proceeds by straightfoward induction on the derivation of the first
hypothesis.

The key case is the one for rule \mylabel{R-IVar} where $\rulet' = \rulet$ and
$E' = x$. In this case the second hypothesis gives us 
\begin{equation*}
\tenv \vturns \rulet \leadsto E
\end{equation*}
As $E = x[E/x]$, this also means
\begin{equation*}
\tenv \vturns \rulet \leadsto x[E/x]
\end{equation*}
Finally, we can apply the Weakening Lemma~\ref{lemma:weakening} to obtain the
desired result.
\begin{equation*}
\tenv, \tenv' \vturns \rulet \leadsto x[E/x]
\end{equation*}

All other cases are straightforward.
\end{proof}

%-------------------------------------------------------------------------------
\subsection{Soundness of Deterministic Resolution}

Lemma~\ref{lemma:s23:4} states that deterministic resolution is
sound with respect to non-deterministic resolution. 

%###############################################################################
{\centering
\fbox{
\begin{minipage}{0.95\columnwidth}
\begin{lemma}\label{lemma:s23:4}
  If 
\begin{equation*}
  \tenv \ivturns \rulet \leadsto E 
\end{equation*}
  then
\begin{equation*}
    \tenv \vturns \rulet \leadsto E
\end{equation*}
\end{lemma}
\end{minipage}
}}

\begin{proof}
The lemma immediately follows from Lemma~\ref{lemma:s23:3}.
\end{proof}

%###############################################################################
{\centering
\fbox{
\begin{minipage}{0.95\columnwidth}
\begin{lemma}\label{lemma:s23:3}
  If 
\begin{equation*}
  \bar{\alpha}; \tenv \ivturns \rulet \leadsto E 
\end{equation*}
  then
\begin{equation*}
    \tenv \vturns \rulet \leadsto E
\end{equation*}
\end{lemma}
\end{minipage}
}}

\begin{proof}
The proof proceeds by induction on the derivation.
\begin{description}
\setlength{\itemsep}{1em}
%===============================================================================
\item[\fbox{\texttt{(R-IAbs)}$^3$}]\quad$\bar{\alpha};\tenv \ivturns \rulet_1 \iarrow \rulet_2 \leadsto \lambda x:|\rulet_1|.E$ \ \\
%===============================================================================
  From the first assumption of rule \mylabel{R-IAbs}$^3$ and the induction hypothesis, we have
\begin{equation*}
  \tenv, x : \rulet_1 \vturns \rulet_2 \leadsto E
\end{equation*}

  Hence, from rule \mylabel{R-IAbs}$^2$ and the freshness condition on $x$ in
rule \mylabel{R-IAbs}$^3$ it follows that
\begin{equation*}
  \tenv \vturns \rulet_1 \iarrow \rulet_2 \leadsto \lambda x:|\rulet_1|.E
\end{equation*}

%===============================================================================
\item[\fbox{\texttt{(R-TAbs)}$^3$}]\quad$\bar{\alpha};\tenv \ivturns \forall \alpha.\rulet \leadsto \Lambda \alpha.E$ \ \\
%===============================================================================
  From the precondition of rule \mylabel{R-TAbs}$^3$ and the induction hypothesis, we have
\begin{equation*}
  \tenv,\alpha \vturns \rulet \leadsto E
\end{equation*}

  Hence, from rule \mylabel{R-TAbs}$^2$ it follows that
\begin{equation*}
  \tenv \vturns \forall \alpha.\rulet \leadsto \Lambda \alpha.E
\end{equation*}

%===============================================================================
\item[\fbox{\texttt{(R-Simp)}$^3$}]\quad$\bar{\alpha};\tenv \ivturns \tau \leadsto E$ \ \\
%===============================================================================

  From the precondition of the rule, Lemma~\ref{lemma:s23:2} and the simple fact that
  $\tenv \subseteq \tenv$, it follows that
\begin{equation*}
  \tenv \vturns \tau \leadsto E
\end{equation*}
\end{description}
\end{proof}


The above proof relies on the following auxiliary lemma for the 
resolution of simple types. The proof of this auxiliary lemma proceeds
by mutual induction with the proof of the main lemma.

%###############################################################################
{\centering
\fbox{
\begin{minipage}{0.95\columnwidth}
\begin{lemma}\label{lemma:s23:2}
  If 
\begin{equation*}
   \bar{\alpha}; \tenv; \tenv' \ivturns \tau \leadsto E
\end{equation*}
  and
\begin{equation*}
  \tenv' \subseteq \tenv 
\end{equation*}
  then
\begin{equation*}
  \tenv \vturns \tau \leadsto E
\end{equation*}
\end{lemma}
\end{minipage}
}}

\begin{proof}
The proof proceeds by induction on the derivation, mutually 
with the previous proof.
\begin{description}
\setlength{\itemsep}{1em}
%===============================================================================
\item[\fbox{\texttt{(L-RuleMatch)}}]\quad$\bar{\alpha}; \tenv; \tenv', \rulet \leadsto x \ivturns \type \leadsto E[\bar{E}/\bar{x}]$ \ \\
%===============================================================================
  The first assumption of the rule is
\begin{equation*}
  \tenv; \rulet \leadsto x \ivturns \type \leadsto E; \bar{E} \leadsto \bar{x}
\end{equation*}

  From the lemma's assumption $(\tenv', \rulet \leadsto x) \subseteq \tenv$ we conclude
  $(\rulet \leadsto x) \in \tenv$. Hence, by rule \mylabel{R-Simp}$^2$ we have
\begin{equation*}
  \tenv \vturns \rulet \leadsto x
\end{equation*}

  From the second precondition of the rule and the (mutual) induction hypothesis, we also have
\begin{equation*}
  \tenv \vturns \bar{\rulet} \leadsto \bar{E}
\end{equation*}

  The above three observations allow us to invoke the auxiliary Lemma~\ref{lemma:s23:1}
  and conclude
\begin{equation*}
  \tenv \vturns \tau \leadsto E[\bar{E}/\bar{x}]
\end{equation*}

%===============================================================================
\item[\fbox{\texttt{(L-Var), (L-TyVar), (L-RuleNoMatch)}}] \ \\
%===============================================================================
  Trivially by applying the induction hypothesis on the precondition of the rule.
\end{description}
\end{proof}

The above proof relies on the following auxiliary lemma.

%###############################################################################
{\centering
\fbox{
\begin{minipage}{0.95\columnwidth}
\begin{lemma}\label{lemma:s23:1}
  If 
\begin{equation*}
   \tenv; \rulet \leadsto E \ivturns \type \leadsto E'; \bar{\rulet} \leadsto \bar{x}
\end{equation*}
  and
\begin{equation*}
  \tenv \vturns \rulet \leadsto E
\end{equation*}
  and
\begin{equation*}
  \tenv \vturns \bar{\rulet} \leadsto \bar{E}
\end{equation*}
  then
\begin{equation*}
  \tenv \vturns \tau \leadsto E'[\bar{E}/\bar{x}]
\end{equation*}
\end{lemma}
\end{minipage}
}}

\begin{proof}
The proof proceeds by induction on the derivation of the first assumption.
\begin{description}
\setlength{\itemsep}{1em}
%===============================================================================
\item[\fbox{\texttt{(M-Simp)}}]\quad$\tenv;\type\leadsto E \ivturns \type \leadsto E; \epsilon$ \ \\
%===============================================================================
  The first assumption of the lemma is the desired conclusion
\begin{equation*}
  \tenv \vturns \type \leadsto E
\end{equation*}

%===============================================================================
\item[\fbox{\texttt{(M-IApp)}}]\quad$\tenv; \rulet_1 \iarrow \rulet_2 \leadsto E \ivturns \type \leadsto E'; \bar{\rulet} \leadsto \bar{x}, \rulet_1 \leadsto x$ \ \\
%===============================================================================
  The third hypothesis of the lemma then is
\begin{equation*}
  \tenv \vturns \bar{\rulet} \leadsto \bar{E} \quad\wedge\quad \tenv \vturns \rulet_1 \leadsto E_1
\end{equation*}
  The Weakening Lemma~\ref{lemma:weakening} turns the first conjunct into
\begin{equation*}
  \tenv, \rulet_1 \leadsto x \vturns \bar{\rulet} \leadsto \bar{E}
\end{equation*}

  The second hypothesis of the lemma then is
\begin{equation*}
  \tenv \vturns \rulet_1 \iarrow \rulet_2 \leadsto E
\end{equation*}
  By applying the Weakening Lemma~\ref{lemma:weakening} we get
\begin{equation*}
  \tenv, \rulet_1 \leadsto x \vturns \rulet_1 \iarrow \rulet_2 \leadsto E
\end{equation*}
  From rule \mylabel{R-IVar}$^2$ we can also conclude
\begin{equation*}
  \tenv, \rulet_1 \leadsto x \vturns \rulet_1 \leadsto x
\end{equation*}
  These two facts allow us to derive from rule \mylabel{R-IApp}
\begin{equation*}
  \tenv, \rulet_1 \leadsto x \vturns \rulet_2 \leadsto E\,x 
\end{equation*}

  We now have the necessary ingredients to invoke the induction hypothesis
  on the hypothesis of the rule and obtain
\begin{equation*}
  \tenv, \rulet_1 \leadsto x \vturns \type \leadsto E'[\bar{E}/\bar{x}]
\end{equation*}

  Finally, we use the second conjunct of the third hypothesis to invoke
  the Substitution Lemma~\ref{lemma:substitution} on the above and
  reach our desired conclusion
\begin{equation*}
  \tenv \vturns \type \leadsto E'[\bar{E}/\bar{x}][E_1/x]
\end{equation*}

%===============================================================================
\item[\fbox{\texttt{(M-TApp)}}]\quad$\tenv; \forall\alpha.\rulet \leadsto E \ivturns \type \leadsto E'; \bar{\rulet} \leadsto \bar{x}$ \ \\
%===============================================================================
  Then the second hypothesis of the lemma is
\begin{equation*}
  \tenv \vturns \forall\alpha.\rulet \leadsto E
\end{equation*}
  This allows us to conclude by rule \mylabel{R-TApp} that
\begin{equation*}
  \tenv \vturns \rulet[\suty/\alpha] \leadsto E\,|\suty|
\end{equation*}

  The third hypothesis of the lemma is
\begin{equation*}
  \tenv \vturns \bar{\rulet} \leadsto \bar{E}
\end{equation*}

  We now have the necessary ingredients to invoke the induction hypothesis
  on the hypothesis of the rule and obtain the desired conclusion
\begin{equation*}
  \tenv \vturns \type \leadsto E'[\bar{E}/\bar{x}]
\end{equation*}

\end{description}
\end{proof}

% %-------------------------------------------------------------------------------
% \subsection{Correctness of the Resolution Algorithm}
% 
% Lemma~\ref{lemma:sa:3} states that the resolution algorithm
% $\vdash_{\mathit{alg}}$ is sound with respect to the deterministic resolution
% specification $\vdash_r$.
% 
% Its proof relies on Lemma~\ref{lemma:sa:2}, which states that
% the auxiliary relation $\vdash_{\mathit{match1st}}$ is sound, whose
% proof in turn relies on Lemma~\ref{lemma:sa:1} which states that
% the auxiliary relation $\vdash_{\mathit{match}}$ is sound.
% 
% The completeness proof proceeds in a similar fashion.
% 
% %###############################################################################
% {\centering
% \fbox{
% \begin{minipage}{0.95\columnwidth}
% \begin{lemma}\label{lemma:sa:1}
%   If 
% \begin{equation*}
%     \rho;\bar{\rho};\bar{\alpha};\bar{\omega};E \vdash_{\mathit{match}} \tau \hookrightarrow \bar{\rho}';\bar{\omega'};E'
% \end{equation*}
%   then there exist $\bar{\rho}''$ with 
% \begin{equation*}
%   \bar{\rho}\theta \subseteq \bar{\rho'}
% \end{equation*}
%   where $\theta = [\bar{\rho''}/\bar{\alpha}]$,
% 
%   and 
% \begin{equation*}
%   \bar{\omega} \subseteq \bar{\omega}'
% \end{equation*}
% 
%   such that forall $\env$ and $\bar{E}''$: 
% 
%   if
% \begin{equation*}
%   \bar{\alpha}; \env \vdash_r \rho_i' \leadsto E_i''\quad\quad (\forall \rho_i' \in \bar{\rho}')
% \end{equation*}
% 
%   then 
% \begin{equation*}
%   \rho\theta \lhd \tau
% \end{equation*}
%   and
% \begin{equation*}
%   \bar{\alpha}; \env; \rho\theta \leadsto E|\theta|\eta \vdash_\downarrow \tau \leadsto E'\eta
% \end{equation*}
%   where $\eta = [\bar{E}''/\bar{\omega}']$
% \end{lemma}
% \end{minipage}
% }}
% 
% \begin{proof}
% The proof proceeds by induction on the derivation.
% \begin{description}
% %===============================================================================
% \item[(\texttt{MTC-Simp})]\quad$\tau';\bar{\rho};\bar{\alpha};\bar{\omega};E \vdash_{\mathit{match}} \tau \hookrightarrow \bar{\rho}\theta;\bar{\omega};E|\theta|$ \ \\
% %===============================================================================
%   Obviously, we have that 
% \begin{equation*}
%   \bar{\omega} \subseteq \bar{\omega}
% \end{equation*}
% 
%   and
% \begin{equation*}
%   \bar{\rho}\theta \subseteq \bar{\rho}\theta
% \end{equation*}
% 
%   From rule \mylabel{MTC-Simp} we have
% \begin{equation*}
%   \theta = \mathit{mgu}_{\bar{\alpha}}(\tau,\tau')
% \end{equation*}
% 
%   This means 
% \begin{equation*}
%   \tau'\theta = \tau
% \end{equation*}
% 
%   Hence, from rule \mylabel{M-Simp} it follows that
% \begin{equation*}
%   \tau'\theta \lhd \tau
% \end{equation*}
% 
%   Also, from rule \mylabel{I-Simp} it follows that
% \begin{equation*}
%   \bar{\alpha}; \env; \tau \leadsto E|\theta|\eta  \vdash_\downarrow \tau \leadsto E|\theta|\eta
% \end{equation*}
% 
%   or, equivalently,
% \begin{equation*}
%   \bar{\alpha}; \env; \tau'\theta \leadsto E|\theta|\eta \vdash_\downarrow \tau \leadsto E|\theta|\eta
% \end{equation*}
% 
% %===============================================================================
% \item[(\texttt{MTC-IAbs})]\quad$\rho_1 \iarrow \rho_2;\bar{\rho};\bar{\alpha};\bar{\omega};E \vdash_{\mathit{match}} \tau \hookrightarrow \bar{\rho}';\bar{\omega};E'$ \ \\
% %===============================================================================
% 
%   From the rule \mylabel{MTC-IAbs} and the induction hypothesis, it follows that
% \begin{equation*}
%   \bar{\omega}, \omega \subseteq \bar{\omega'}
% \end{equation*}
% 
%   Hence,
% \begin{equation*}
%   \bar{\omega} \subseteq \bar{\omega'}
% \end{equation*}
% 
%   Similarly, it follows that
% \begin{equation*}
%   (\bar{\rho},\rho_1)\theta \subseteq \bar{\rho}'
% \end{equation*}
% 
%   Hence,
% \begin{equation*}
%   \bar{\rho}\theta \subseteq \bar{\rho}'
% \end{equation*}
% 
%   Also, from the rule \mylabel{MTC-IAbs} and the induction hypothesis, it follows that
% \begin{equation*}
%   \rho_2\theta \lhd \tau
% \end{equation*}
% 
%   and, by rule \mylabel{M-IApp}, we hence have
% \begin{equation*}
%   \rho_1\theta \iarrow \rho_2\theta \lhd \tau
% \end{equation*}
% 
%   or, more succinctly,
% \begin{equation*}
%   (\rho_1 \iarrow \rho_2)\theta \lhd \tau
% \end{equation*}
% 
%   Finally, from the rule \mylabel{MTC-IAbs} and the induction hypothesis, it follows that
% \begin{equation*}
%  \bar{\alpha}; \env; \rho_2\theta \leadsto (E\,\omega)|\theta|\eta \vdash_\downarrow \tau \leadsto E'\eta 
% \end{equation*}
% 
%   or
% \begin{equation*}
%  \bar{\alpha}; \env; \rho_2\theta \leadsto (E\theta\eta)\,(\omega\eta) \vdash_\downarrow \tau \leadsto E'\eta 
% \end{equation*}
% 
%   Using rule \mylabel{I-IAbs} we may then conclude
% \begin{equation*}
%  \bar{\alpha}; \env; \rho_1\theta \iarrow \rho_2\theta \leadsto E\theta\eta \vdash_\downarrow \tau \leadsto E'\eta 
% \end{equation*}
% 
%   or, equivalently,
% \begin{equation*}
%  \bar{\alpha}; \env; (\rho_1 \iarrow \rho_2)\theta \leadsto E\theta\eta \vdash_\downarrow \tau \leadsto E'\eta 
% \end{equation*}
% 
% %===============================================================================
% \item[(\texttt{MTC-TAbs})]\quad$\forall \alpha.\rho;\bar{\rho};\bar{\alpha};\bar{\omega};E \vdash_{\mathit{match}} \tau \hookrightarrow \bar{\rho}';\bar{\omega};E'$ \ \\
% %===============================================================================
% 
%   From the rule \mylabel{MTC-TAbs} and the induction hypothesis, it follows that
% \begin{equation*}
%   \bar{\omega} \subseteq \bar{\omega'}
% \end{equation*}
% 
%   Similarly, it follows that
% \begin{equation*}
%   \bar{\rho}\theta \subseteq \bar{\rho}'
% \end{equation*}
% 
%   Also it follows that
% \begin{equation*}
%   \rho\theta \lhd \tau
% \end{equation*}
% 
%   or, equivalently,
% \begin{equation*}
%   \rho[\bar{\alpha}\theta/\bar{\alpha}][\alpha\theta/\alpha] \lhd \tau
% \end{equation*}
% 
%   Hence, following rule \mylabel{MTC-TAbs}, we have that
% \begin{equation*}
%   \forall \alpha.(\rho[\bar{\alpha}\theta/\bar{\alpha}]) \lhd \tau
% \end{equation*}
% 
%   or, equivalently,
% \begin{equation*}
%   (\forall \alpha.\rho)[\bar{\alpha}\theta/\bar{\alpha}] \lhd \tau
% \end{equation*}
% 
%   Finally, from the rule \mylabel{MTC-TAbs} and the induction hypothesis, it follows that
% \begin{equation*}
%   \bar{\alpha}; \env; \rho\theta \leadsto (E\,\alpha)|\theta|\eta \vdash_\downarrow \tau \leadsto E'\eta 
% \end{equation*}
% 
%   or, equivalently,
% \begin{equation*}
%   \bar{\alpha}; \env; \rho\theta \leadsto (E|\theta|[\alpha|\theta|/\alpha]\eta)\,(\alpha|\theta|) \vdash_\downarrow \tau \leadsto E'\eta 
% \end{equation*}
% 
%   Hence, following rule \mylabel{I-TAbs}, we get
% \begin{equation*}
%   \bar{\alpha}; \env; (\forall \alpha.\rho)\theta \leadsto E|\theta|\eta \vdash_\downarrow \tau \leadsto E'\eta 
% \end{equation*}
% \end{description}
% \end{proof}
% 
% %###############################################################################
% {\centering
% \fbox{
% \begin{minipage}{0.95\columnwidth}
% \begin{lemma}\label{lemma:sa:2}
%   If 
% \begin{equation*}
%    \bar{\alpha}; \env \vdash_{\mathit{match1st}} \tau \hookrightarrow \bar{\rho}';\bar{\omega};E
% \end{equation*}
%   then there exist $\rho$ and $x$ such that
% 
% \begin{equation*}
%   \elookup{\env}{\tau} = \rho \leadsto x
% \end{equation*}
% 
% and for all $\env' \supseteq \env$
% and for all $\bar{E}'$ such that
% \begin{equation*}
%   \bar{\alpha}; \env' \vdash_r \rho_i' \leadsto E_i'\quad\quad (\forall \rho_i' \in \bar{\rho}')
% \end{equation*}
% 
% we have that
% \begin{equation*}
%   \bar{\alpha}; \env'; \rho \leadsto x \vdash_\downarrow \tau \leadsto E\eta
% \end{equation*}
% where $\eta = [\bar{E}'/\bar{\omega}]$.
% \end{lemma}
% \end{minipage}
% }}
% 
% \begin{proof}
% The proof proceeds by induction on the derivation.
% \begin{description}
% %===============================================================================
% \item[(\texttt{M1-Head})]\quad$\bar{\alpha}; \env,\rho \leadsto x \vdash_{\mathit{match1st}} \tau \hookrightarrow \bar{\rho};\bar{\omega};E$ \ \\
% %===============================================================================
%   From rule \mylabel{M1-Head} and the previous lemma, we then have that
% \begin{equation*}
%   \rho \lhd \tau 
% \end{equation*}
%   
%   Using rule \mylabel{L-Head} we can conclude that
% \begin{equation*}
%   \elookup{(\env,\rho\leadsto x)}{\tau} = \rho \leadsto x
% \end{equation*}
% 
%   Similarly, we have that 
% \begin{equation*}
%   \bar{\alpha}; \env'; \rho \leadsto x\eta \vdash_\downarrow \tau \leadsto E\eta
% \end{equation*}
% 
%   which simplifies to
% \begin{equation*}
%   \bar{\alpha}; \env'; \rho \leadsto x \vdash_\downarrow \tau \leadsto E\eta
% \end{equation*}
% 
% %===============================================================================
% \item[(\texttt{M1-Tail})]\quad$\bar{\alpha}; \env,\rho \leadsto x \vdash_{\mathit{match1st}} \tau \hookrightarrow \bar{\rho};\bar{\omega};E$ \ \\
% %===============================================================================
%   From rule \mylabel{M1-Tail} and the induction hypothesis, we then have that there
%   exist $\rho'$ and $x'$ such that
% \begin{equation*}
%   \elookup{\env}{\tau} = \rho' \leadsto x'
% \end{equation*}
% 
%   Also from rule \mylabel{M1-Tail} and a previous lemma, we have that
% \begin{equation*}
%   \rho \not\!\!\lhd \tau
% \end{equation*}
% 
%   Hence, using rule \mylabel{L-Tail} we can conclude that 
% \begin{equation*}
%   \elookup{(\env,\rho\leadsto x)}{\tau} = \rho' \leadsto x'
% \end{equation*}
% 
%   From rule \mylabel{M1-Tail} and the induction hypothesis, we also have that
% \begin{equation*}
%   \bar{\alpha}; \env'; \rho' \leadsto x' \vdash_\downarrow \tau \leadsto E\eta
% \end{equation*}
%   for $\env' \supseteq (\env,\rho \leadsto x)$.
% 
% \end{description}
% \end{proof}
% 
% %###############################################################################
% {\centering
% \fbox{
% \begin{minipage}{0.95\columnwidth}
% \begin{lemma}\label{lemma:sa:3}
%   If 
% \begin{equation*}
%    \bar{\alpha}; \env \vdash_{\mathit{alg}} \rho \leadsto E
% \end{equation*}
% 
%   then
% \begin{equation*}
%   \bar{\alpha}; \env \vdash_r \rho \leadsto E
% \end{equation*}
% \end{lemma}
% \end{minipage}
% }}
% 
% \begin{proof}
% The proof proceeds by induction on the derivation.
% \begin{description}
% %===============================================================================
% \item[(\texttt{Alg-TAbs})]\quad$\bar{\alpha}; \env \vdash_{\mathit{alg}} \forall \alpha.\rho \leadsto \Lambda \alpha.E$ \ \\
% %===============================================================================
%   From rule \mylabel{Alg-TAbs} and the induction hypothesis, it follows that
% \begin{equation*}
%   \bar{\alpha}; \env \vdash_{\mathit{r}} \rho \leadsto E
% \end{equation*}
% 
%   Using rule \mylabel{R-TAbs}, we then get
% \begin{equation*}
%   \bar{\alpha}; \env \vdash_{\mathit{r}} \forall \alpha.\rho \leadsto \Lambda \alpha.E
% \end{equation*}
% 
% %===============================================================================
% \item[(\texttt{Alg-IAbs})]\quad$\bar{\alpha}; \env \vdash_{\mathit{alg}} \rho_1 \iarrow \rho_2 \leadsto \lambda (x:|\rho_1|).E$ \ \\
% %===============================================================================
%   From rule \mylabel{Alg-IAbs} and the induction hypothesis, it follows that
% \begin{equation*}
%   \bar{\alpha}; \env, \rho_1 \leadsto x \vdash_{\mathit{r}} \rho_2 \leadsto E
% \end{equation*}
% 
%   Using rule \mylabel{R-IAbs}, we then get
% \begin{equation*}
%   \bar{\alpha}; \env \vdash_{\mathit{r}} \rho_1 \iarrow \rho_2 \leadsto \lambda (x:|\rho_1|).E
% \end{equation*}
% 
% %===============================================================================
% \item[(\texttt{Alg-Simp})]\quad$\bar{\alpha}; \env \vdash_{\mathit{alg}} \tau \leadsto E[\bar{\omega}/\bar{E}]$ \ \\
% %===============================================================================
%   From rule \mylabel{Alg-Simp} and the previous lemma it follows that
% \begin{equation*}
%   \elookup{\env}{\tau} = \rho \leadsto x
% \end{equation*}
% 
%   and
% \begin{equation*}
%   \bar{\alpha}; \env; \rho \leadsto x \vdash_\downarrow \tau \leadsto E[\bar{\omega}/\bar{E}]
% \end{equation*}
% 
%   Hence, using rule \mylabel{R-Simp}, we conclude
% \begin{equation*}
%   \bar{\alpha}; \env \vdash_r \tau \leadsto E[\bar{\omega}/\bar{E}]
% \end{equation*}
% 
% \end{description}
% \end{proof}
% 
% %-------------------------------------------------------------------------------
% % \subsection{Completeness of the Resolution Algorithm}
% % 
% % %###############################################################################
% % {\centering
% % \fbox{
% % \begin{minipage}{0.95\columnwidth}
% % \begin{lemma}\label{lemma:sa:2}
% %   If 
% % \begin{equation*}
% %    \rho \lhd \tau
% % \end{equation*}
% % 
% %   then for all $\rho^*, \bar{\rho}, \bar{\alpha}, \bar{\omega}, E, \theta$ such that
% % \begin{equation*}
% %   \rho = \rho^*\theta
% % \end{equation*}
% % 
% %   where $\mathit{dom}(\theta) \subseteq \bar{\alpha}$, we have that
% %   there exist $\bar{\rho}', \bar{\omega}', E'$ such that
% % 
% % \begin{equation*}
% %   \rho^*; \bar{\rho}; \bar{\alpha}; \bar{\omega}; E \vdash_\mathit{match} \tau \hookrightarrow \bar{\rho}'; \bar{\omega}'; E'
% % \end{equation*}
% % \end{lemma}
% % \end{minipage}
% % }}
% % 
% % \begin{proof}
% % The proof proceeds by induction on the derivation.
% % \begin{description}
% % %===============================================================================
% % \item[(\texttt{M-Simp})]\quad$\tau \lhd \tau$ \ \\
% % %===============================================================================
% %   From the assumption and the definition of $\mathit{mgu}_{\bar{\alpha}}$, it follows that
% % \begin{equation*}
% %   \theta = \mathit{mgu}_{\bar{\alpha}}(\tau^*,\tau)
% % \end{equation*}
% % 
% %   Hence, by rule \mylabel{MTC-Simp}, we conclude
% % \begin{equation*}
% %   \tau^*; \bar{\rho}; \bar{\alpha}; \bar{\omega}; E \vdash_\mathit{match} \tau \hookrightarrow \bar{\rho}\theta; \bar{\omega}; E\theta
% % \end{equation*}
% % 
% % %===============================================================================
% % \item[(\texttt{M-IApp})]\quad$\rho_1 \iarrow \rho_2 \lhd \tau$ \ \\
% % %===============================================================================
% %   From rule \mylabel{MTC-IAbs} and the induction hypothesis, we have that
% % \begin{equation*}
% %   \rho^*_2; \bar{\rho},\rho^*_1; \bar{\alpha}; \bar{\omega},\omega; E \vdash_\mathit{match} \tau \hookrightarrow \bar{\rho}'; \bar{\omega}'; E'
% % \end{equation*}
% % 
% %   Hence, by rule \mylabel{MTC-IAbs}, we conclude
% % \begin{equation*}
% %   \rho^*_1 \iarrow \rho^*_2; \bar{\rho}; \bar{\alpha}; \bar{\omega}; E \vdash_\mathit{match} \tau \hookrightarrow \bar{\rho}'; \bar{\omega}'; E'
% % \end{equation*}
% % \end{description}
% % \end{proof}
% 
%-------------------------------------------------------------------------------
\subsection{Deterministic Resolution is Deterministic}\label{proof:determinism}

%\newcommand{\unamb}{\vdash_{\mathit{unamb}}}

%###############################################################################
{\centering
\fbox{
\begin{minipage}{0.95\columnwidth}
\begin{lemma}\label{lemma:determinism:0}
  If 
\begin{equation*}
  \unamb \tenv
\end{equation*}
  and
\begin{equation*}
  \unamb \rulet
\end{equation*}
  and
\begin{equation*}
  \tenv \ivturns \rulet \leadsto E_1
\end{equation*}
and
\begin{equation*}
  \tenv \ivturns \rulet \leadsto E_2
\end{equation*}
then
\begin{equation*}
  E_1 = E_2
\end{equation*}
\end{lemma}
\end{minipage}
}}

\begin{proof}
From the third and fourth hypotheses of the lemma, the hypothesis
of rule \mylabel{R-Main} and Lemma~\ref{lemma:determinism:1} the desired
result follows
\begin{equation*}
  E_1 = E_2
\end{equation*}
\end{proof}

%###############################################################################
{\centering
\fbox{
\begin{minipage}{0.95\columnwidth}
\begin{lemma}\label{lemma:determinism:1}
  If 
\begin{equation*}
  \unamb \tenv
\end{equation*}
  and
\begin{equation*}
  \unamb \rulet
\end{equation*}
  and
\begin{equation*}
  \bar{\alpha};\tenv \ivturns \rulet \leadsto E_1
\end{equation*}
and
\begin{equation*}
  \bar{\alpha};\tenv \ivturns \rulet \leadsto E_2
\end{equation*}
then
\begin{equation*}
  E_1 = E_2
\end{equation*}
\end{lemma}
\end{minipage}
}}

\begin{proof}
The proof proceeds by induction on the derivation of the third hypothesis.
\begin{description}
\setlength{\itemsep}{1em}
%===============================================================================
\item[\fbox{\texttt{(R-IAbs)}}]\quad$\bar{\alpha};\tenv \ivturns \rulet_1 \iarrow \rulet_2 \leadsto \lambda\relation{x}{||\rulet_1||}.E_1$ \ \\
%===============================================================================

It follows that the lemma's fourth hypothesis is also derived by rule
\mylabel{R-IAbs}. It follows from the lemma's second hypothesis that 
\begin{equation*}
  \unamb \rulet_1 \quad\wedge\quad \unamb \rulet_2
\end{equation*}
From this and lemma's first hypothesis, it follows that
\begin{equation*}
  \unamb \tenv,\rulet_1 \leadsto x
\end{equation*}
From the rule's hypothesis and the induction hypothesis, it follows that
\begin{equation*}
  E_1 = E_2
\end{equation*}
Hence, we may conclude
\begin{equation*}
  \lambda x:|\rulet_1|.E_1 = \lambda x:|\rulet_1|.E_2
\end{equation*}

%===============================================================================
\item[\fbox{\texttt{(R-TAbs)}}]\quad$\bar{\alpha};\tenv \ivturns \forall \alpha. \rho \leadsto \Lambda\alpha.E_1$ \ \\
%===============================================================================

It follows that the lemma's fourth hypothesis is also derived by rule
\mylabel{R-TAbs}. It follows from the lemma's second hypothesis that 
\begin{equation*}
  \unamb \rulet
\end{equation*}
From the lemma's first hypothesis, it follows that
\begin{equation*}
  \unamb \tenv, \alpha
\end{equation*}
From the rule's hypothesis and the induction hypothesis, it follows that
\begin{equation*}
  E_1 = E_2
\end{equation*}
Hence, we may conclude
\begin{equation*}
  \Lambda \alpha.E_1 = \Lambda \alpha.E_2
\end{equation*}

%===============================================================================
\item[\fbox{\texttt{(R-Simp)}}]\quad$\bar{\alpha};\tenv \ivturns \type \leadsto E_1$ \ \\
%===============================================================================

It follows that the lemma's fourth hypothesis is also derived by rule
\mylabel{R-Simp}. We obtain the desired result from Lemma~\ref{lemma:determinism:2}
\begin{equation*}
  E_1 = E_2
\end{equation*}

   
\end{description}
\end{proof}

%###############################################################################
{\centering
\fbox{
\begin{minipage}{0.95\columnwidth}
\begin{lemma}\label{lemma:determinism:2}
  If 
\begin{equation*}
  \unamb \tenv
\end{equation*}
  and
\begin{equation*}
  \unamb \tenv'
\end{equation*}
  and
\begin{equation*}
  \bar{\alpha};\tenv;\tenv' \ivturns \type \leadsto E_1
\end{equation*}
and
\begin{equation*}
  \bar{\alpha};\tenv;\tenv' \ivturns \type \leadsto E_2
\end{equation*}
then
\begin{equation*}
  E_1 = E_2
\end{equation*}
\end{lemma}
\end{minipage}
}}

\begin{proof}
The proof proceeds by induction on the derivation of the third hypothesis.
\begin{description}
\setlength{\itemsep}{1em}
%===============================================================================
\item[\fbox{\texttt{(L-RuleMatch)}}]\quad$\bar{\alpha};\tenv;\tenv',\rulet \leadsto x \ivturns \type \leadsto E_1[\bar{E}_1/\bar{x}]$ \ \\
%===============================================================================
  Then the fourth hypothesis was either derived from rule \mylabel{L-RuleMatch},
  or from rule \mylabel{L-RuleNoMatch}. However, the hypothesis of the latter is
  not satisfied: $\epsilon;E_1;\Sigma_1$ forms a counter-example. Hence, the fourth
  hypothesis is also formed by rule \mylabel{L-RuleMatch}.

  Then it follows from the first hypothesis of the rule and Lemma~\ref{lemma:determinism:3} that
\begin{equation*}
  E_1 = E_2 \quad\wedge\quad \Sigma_1 = \Sigma_2
\end{equation*}
  From the second hypothesis of the rule and Lemma~\ref{lemma:determinism:1} it also follows that
\begin{equation*}
  \bar{E}_1 = \bar{E_2}
\end{equation*}
  Hence, we may conclude
\begin{equation*}
  E_1[\bar{E}_1/\bar{x}] = E_2[\bar{E_2}/\bar{x}]
\end{equation*}


%===============================================================================
\item[\fbox{\texttt{(L-RuleNoMatch)}}]\quad$\bar{\alpha};\tenv;\tenv',\rulet \leadsto x \ivturns \type \leadsto E_1'$ \ \\
%===============================================================================
  Then the fourth hypothesis was either derived from rule \mylabel{L-RuleMatch},
  or from rule \mylabel{L-RuleNoMatch}. However, the hypothesis of the former is
  not satisfied, as it would be a counter-example for the first hypothesis of
  the assumed rule of the third hypothesis. Hence, the fourth hypothesis is also
  formed by rule \mylabel{L-RuleNoMatch}.

  From the second hypothesis of the lemma we derive $\unamb \tenv'$.
  Then from the second hypothesis of the rule and the induction hypothesis we conclude
  the desired result
\begin{equation*}
  E_1' = E_2'
\end{equation*}

%===============================================================================
\item[\fbox{\texttt{(L-Var)}}]\quad$\bar{\alpha};\tenv;\tenv',x:\rulet \ivturns \type \leadsto E_1$ \ \\
%===============================================================================
  Clearly the fourth hypothesis is also derived by rule \mylabel{L-Var}.
  Moreover, from the second hypothesis it follows that $\unamb \tenv'$.
  Hence, from the induction hypothesis we conclude that
\begin{equation*}
  E_1 = E_2
\end{equation*}

%===============================================================================
\item[\fbox{\texttt{(L-TyVar)}}]\quad$\bar{\alpha};\tenv;\tenv',\alpha \ivturns \type \leadsto E_1$ \ \\
%===============================================================================
  Clearly the fourth hypothesis is also derived by rule \mylabel{L-TyVar}.
  Moreover, from the second hypothesis it follows that $\unamb \tenv'$.
  Hence, from the induction hypothesis we conclude that
\begin{equation*}
  E_1 = E_2
\end{equation*}
\end{description}
\end{proof}

We annotated the judgement with the sequence of substitution types $\bar{\suty}$
used to instantiate the universal quantifiers.

{\centering
\fbox{
\begin{minipage}{0.95\columnwidth}
\begin{center}
$\ba{c}
\myruleform{\bar{\suty};\tenv; \rulet~\gbox{\leadsto E} \ivturns \type~\gbox{\leadsto E'}; \Sigma}\\ \\
\mylabel{M-Simp} \quad
          {\epsilon;\tenv; \type~\gbox{\leadsto E} \ivturns \type~\gbox{\leadsto E}; \epsilon} \\ \\
\mylabel{M-IApp} \quad
  \myirule{\bar{\suty};\tenv, \rulet_1 \gbox{\leadsto x}; \rulet_2 ~\gbox{\leadsto E\,x} \ivturns \type~\gbox{\leadsto E'}; \Sigma 
           \quad\quad\quad \gbox{x~\mathit{fresh}}
          }
          {\bar{\suty};\tenv; \rulet_1 \iarrow \rulet_2 ~\gbox{\leadsto E} \ivturns \type~\gbox{\leadsto E'}; \Sigma, \rulet_1~\gbox{\leadsto x}} \\ \\ 
\mylabel{M-TApp} \quad
  \myirule{\bar{\suty};\tenv; \rulet[\suty/\alpha] ~\gbox{\leadsto E\,|\suty|} \ivturns \type~\gbox{\leadsto E'; \Sigma}
           \quad\quad\quad
           \tenv \turns \suty
          }
          {\bar{\suty},\suty;\tenv; \forall \alpha. \rulet ~\gbox{\leadsto E} \ivturns \type~\gbox{\leadsto E'}; \Sigma} \\
\ea
$
\end{center}
\end{minipage}
}}

It is not difficult to see that any derivation of the annotated judgement
is in one to one correspondence with a derivation of the unannotated 
judgement.

The judgement is deterministic.

%###############################################################################
{\centering
\fbox{
\begin{minipage}{0.95\columnwidth}
\begin{lemma}\label{lemma:determinism:3}
  If 
\begin{equation*}
  \bar{\alpha} \unamb \rulet
\end{equation*}
  and
\begin{equation*}
  \bar{\suty}_1 ; \tenv; \rulet[\bar{\suty}_2/\bar{\alpha}]\leadsto E[|\bar{\suty}_2|/\bar{\alpha}] \ivturns \type \leadsto E_1; \Sigma_1
\end{equation*}
and
\begin{equation*}
  \bar{\suty}_1' ; \tenv; \rulet[\bar{\suty}_2'/\bar{\alpha}]\leadsto E[|\bar{\suty}_2'|/\bar{\alpha}] \ivturns \type \leadsto E_2; \Sigma_2
\end{equation*}
then
\begin{equation*}
  \bar{\suty}_1 = \bar{\suty}_1' \quad\wedge\quad \bar{\suty}_2 = \bar{\suty}_2'
  \quad\wedge\quad E_1 = E_2 \quad\wedge\quad \Sigma_1 = \Sigma_2
  \quad\wedge\quad \unamb \Sigma_1
\end{equation*}
% where
% \begin{equation*}
% \begin{array}{c}
%   \myruleform{\bar{\suty}; \rulet \vdash \type} \\ \\
%   \mylabel{P-1} \quad \myirule{}{\epsilon; \type \vdash \type} \\ \\
%   \mylabel{P-2} \quad \myirule{\bar{\suty}; \rulet_2 \vdash \type}{\bar{\suty}; \rulet_1 \iarrow \rulet_2 \vdash \type}
%   \mylabel{P-3} \quad \myirule{\bar{\suty}; \rulet[\suty/\alpha] \vdash \type}{\suty,\bar{\suty}; \forall\alpha.\rulet \vdash \type}
% \end{array}
% \end{equation*}
\end{lemma}
\end{minipage}
}}

\begin{proof}
The proof proceeds by induction on the derivation of the first hypothesis.
\begin{description}
\setlength{\itemsep}{1em}
%===============================================================================
\item[\fbox{\texttt{(UA-Simp)}}]\quad$\bar{\alpha} \unamb \type'$ \ \\
%===============================================================================
  Then the second and third hypothesis of the lemma must have been formed by rule \mylabel{M-Simp}
  and hence 
\begin{equation*}
  \bar{\suty}_1 = \epsilon = \bar{\suty}_1'
\end{equation*}
  
  For the same reason we have that $\type'[\bar{\suty}_2/\bar{\alpha}] = \type = \type'[\bar{\suty}_2'/\bar{\alpha}]$. Since we know that $\bar{\alpha} \subseteq \mathit{ftv}(\type)$, it must follow also that
\begin{equation*}
  \bar{\suty}_2 = \bar{\suty}_2'
\end{equation*}

  As a consequence, we also have that
\begin{equation*}
  E_1 = E[|\bar{\sigma}_2|/\bar{\alpha}] = E[|\bar{\sigma}_2'|/\bar{\alpha}] = E_2
\end{equation*}

  Finally, it also follows from rule \mylabel{M-Simp} that
\begin{equation*}
  \Sigma_1 = \epsilon = \Sigma_2
\end{equation*}
  and trivially
\begin{equation*}
  \unamb \epsilon
\end{equation*}

%===============================================================================
\item[\fbox{\texttt{(UA-IAbs)}}]\quad$\bar{\alpha} \unamb \rulet_1 \iarrow \rulet_2$ \ \\
%===============================================================================
  Then the second and third hypothesis of the lemma must have been formed by rule \mylabel{M-IApp}.
  From their two hypotheses and from the hypothesis of the rule and the induction hypothesis, we obtain
  the desired results
\begin{equation*}
  \bar{\suty}_1 = \bar{\suty}_1' \quad\wedge\quad \bar{\suty}_2 = \bar{\suty}_2'
  \quad\wedge\quad 
  E_1 = E_2
  \quad\wedge\quad
  \Sigma_1, \rulet_1[|\bar{\sigma}_2|/\bar{\alpha}]\leadsto x
  = 
  \Sigma_2, \rulet_1[|\bar{\sigma}_2'|/\bar{\alpha}]\leadsto x
\end{equation*}

  We also derive from the induction hypothesis that $\unamb \Sigma_1$. 
  Since $\bar{\alpha} \unamb \rulet_1 \iarrow \rulet_2$, we also have $\unamb \rulet_1$.
  Hence we also conclude
\begin{equation*}
  \unamb \Sigma_1, \rulet_1 \leadsto x
\end{equation*}

%===============================================================================

\item[\fbox{\texttt{(UA-TAbs)}}]\quad$\bar{\alpha} \unamb \forall \alpha.\rulet$ \ \\
%===============================================================================
  Then the second and third hypothesis of the lemma must have been formed by rule \mylabel{M-TApp},
  with $\bar{\suty}_1 = \bar{\suty}_{1,1},\suty_{1,2}$ and $\bar{\suty}_1' = \bar{\suty}_{1,1}',\suty_{2,2}'$.
  From their two hypotheses and from the hypothesis of the rule and the induction hypothesis, we obtain
\begin{equation*}
  \bar{\suty}_2,\suty_{1,2} = \bar{\suty}_2',\suty_{1,2}' 
  \quad\wedge\quad 
  \bar{\suty}_{1,1} = \bar{\suty}_{1,1}'
  \quad\wedge\quad 
  E_1 = E_2
  \quad\wedge\quad 
  \Sigma_1 = \Sigma_2
  \quad\wedge\quad 
  \unamb \Sigma_1
\end{equation*}
  
  From this we conclude the desired result 
\begin{equation*}
  \bar{\suty}_1 = \bar{\suty}_1' \quad\wedge\quad \bar{\suty}_2 = \bar{\suty}_2'
  \quad\wedge\quad 
  E_1 = E_2
  \quad\wedge\quad 
  \Sigma_1 = \Sigma_2
  \quad\wedge\quad 
  \unamb \Sigma_1
\end{equation*}
   
\end{description}
\end{proof}

%-------------------------------------------------------------------------------
\subsection{Resolution Coherence}\label{proof:coherence}

Deterministic resolution is stable under substitution.

%###############################################################################
{\centering
\fbox{
\begin{minipage}{0.95\columnwidth}
\begin{lemma}\label{lemma:coherence:0}
  If 
\begin{equation*}
   \tenv,\alpha,\tenv' \ivturns \rulet\leadsto E
\end{equation*}
  and
\begin{equation*}
  \tenv \turns \suty
\end{equation*}
  then
\begin{equation*}
  \tenv,\tenv'[\suty/\alpha] \ivturns \rulet[\suty/\alpha] \leadsto E[|\suty|/\alpha]
\end{equation*}
\end{lemma}
\end{minipage}
}}

\begin{proof}
  The hypothesis of rule \label{R-Main} then is
\begin{equation*}
  \mathit{ftv}(\tenv),\alpha,\mathit{ftv}(\tenv'); \tenv,\alpha,\tenv' \ivturns \rulet \leadsto E
\end{equation*}

  From Lemma~\ref{lemma:coherence:1} it follows that
\begin{equation*}
  \mathit{ftv}(\tenv),\mathit{ftv}(\tenv'); \tenv,\tenv'[\suty/\alpha] \ivturns \rulet[\suty/\alpha] \leadsto E[|\suty|/\alpha]
\end{equation*}

  As $\mathit{ftv}(\tenv') = \mathit{ftv}(\tenv'[\suty/\alpha])$, the desired result
  follows from
  rule \mylabel{R-Main}
\begin{equation*}
  \tenv,\tenv'[\suty/\alpha] \ivturns \rulet[\suty/\alpha] \leadsto E[|\suty|/\alpha]
\end{equation*}
\end{proof}

%###############################################################################
{\centering
\fbox{
\begin{minipage}{0.95\columnwidth}
\begin{lemma}\label{lemma:coherence:1}
  If 
\begin{equation*}
   \bar{\alpha},\alpha,\bar{\alpha}'; \tenv,\alpha,\tenv' \ivturns \rulet\leadsto E
\end{equation*}
  and
\begin{equation*}
  \tenv \turns \suty
\end{equation*}
  then
\begin{equation*}
  \bar{\alpha},\bar{\alpha}'; \tenv,\tenv'[\suty/\alpha] \ivturns \rulet[\suty/\alpha] \leadsto E[|\suty|/\alpha]
\end{equation*}
\end{lemma}
\end{minipage}
}}

\begin{proof}
\begin{description}
\setlength{\itemsep}{1em}
%===============================================================================
\item[\fbox{\texttt{(R-IAbs)}}]\quad$\bar{\alpha},\alpha,\bar{\alpha}';\tenv,\alpha,\tenv'
\ivturns \rulet_1 \iarrow \rulet_2 \leadsto \lambda\relation{x}{|\rulet_1|}.E$ \ \\
%===============================================================================
  From the rule's hypothesis and the induction hypothesis we have
\begin{equation*}
\bar{\alpha},\bar{\alpha}';\tenv,(\tenv',\rulet_1\leadsto x)[\suty/\alpha] \ivturns \rulet_2[\suty/\alpha] \leadsto E[|\suty|/\alpha]
\end{equation*}

  From the definition of substitution and rule \mylabel{R-IAbs} we then conclude
\begin{equation*}
\bar{\alpha},\bar{\alpha}';\tenv,\tenv'[\suty/\alpha] \ivturns (\rulet_1 \iarrow \rulet_2)[\suty/\alpha] \leadsto (\lambda x:|\rulet_1|.E)[|\suty|/\alpha]
\end{equation*}

%===============================================================================
\item[\fbox{\texttt{(R-TAbs)}}]\quad$\bar{\alpha},\alpha,\bar{\alpha}';\tenv,\alpha,\tenv'
\ivturns \forall\beta.\rulet \leadsto \Lambda\beta.E$ \ \\
%===============================================================================
  From the rule's hypothesis and the induction hypothesis we have
\begin{equation*}
\bar{\alpha},\bar{\alpha}';\tenv,(\tenv',\beta)[\suty/\alpha] \ivturns \rulet[\suty/\alpha] \leadsto E[|\suty|/\alpha]
\end{equation*}

  From the definition of substitution and rule \mylabel{R-TAbs} we then conclude
\begin{equation*}
\bar{\alpha},\bar{\alpha}';\tenv,\tenv'[\suty/\alpha] \ivturns (\forall\beta.\rulet)[\suty/\alpha] \leadsto (\Lambda\beta.E)[|\suty|/\alpha]
\end{equation*}

%===============================================================================
\item[\fbox{\texttt{(R-Simp)}}]\quad$\bar{\alpha},\alpha,\bar{\alpha}';\tenv,\alpha,\tenv'
\ivturns \type \leadsto E$ \ \\
%===============================================================================
  From the rule's hypothesis and Lemma~\ref{lemma:coherence:2} we conclude
\begin{equation*}
  \bar{\alpha},\bar{\alpha}'; \tenv,\tenv'[\suty/\alpha];\tenv''' \ivturns \type[\suty/\alpha] \leadsto E[|\suty|/\alpha]
\end{equation*}
  and
\begin{equation*}
  R(\tenv;\alpha;\tenv';\tenv,\alpha,\tenv';\tenv''';\suty)
\end{equation*}
  The latter could only have been obtained by ryle \mylabel{R-2}. 
  Hence, we know that $\tenv''' = \tenv,\tenv'[\suty/\alpha]$ and the former is equivalent to
\begin{equation*}
  \bar{\alpha},\bar{\alpha}'; \tenv,\tenv'[\suty/\alpha];\tenv,\tenv'[\suty/\alpha] \ivturns \type[\suty/\alpha] \leadsto E[|\suty|/\alpha]
\end{equation*}

  With this fact we can conclude by rule \mylabel{R-Simp}
\begin{equation*}
\bar{\alpha},\bar{\alpha}';\tenv,\tenv'[\suty/\alpha] \ivturns \type[\suty/\alpha] \leadsto E[|\suty|/\alpha]
\end{equation*}
\end{description}
\end{proof}

%###############################################################################
{\centering
\fbox{
\begin{minipage}{0.95\columnwidth}
\begin{lemma}\label{lemma:coherence:2}
  If 
\begin{equation*}
   \bar{\alpha},\alpha,\bar{\alpha}'; \tenv,\alpha,\tenv';\tenv'' \ivturns \type \leadsto E
\end{equation*}
  and
\begin{equation*}
  \tenv'' \subseteq \tenv,\alpha,\tenv'
\end{equation*}
  and
\begin{equation*}
  \tenv \turns \suty
\end{equation*}
  then
\begin{equation*}
  \bar{\alpha},\bar{\alpha}'; \tenv,\tenv'[\suty/\alpha];\tenv''' \ivturns \type[\suty/\alpha] \leadsto E[|\suty|/\alpha]
\end{equation*}
  and
\begin{equation*}
  R(\tenv;\alpha;\tenv';\tenv'';\tenv''';\suty)
\end{equation*}
\end{lemma}
where
\begin{equation*}
\begin{array}{c}
  \myruleform{R(\tenv;\alpha;\tenv';\tenv'';\tenv''';\suty)} \\ \\
  \mylabel{R-1} \quad \myirule{}{R(\tenv_1,\tenv_2;\alpha;\tenv';\tenv_1;\tenv_1;\suty)} \\ \\
  \mylabel{R-2} \quad \myirule{}{R(\tenv;\alpha;\tenv_1',\tenv_2';\tenv,\alpha,\tenv_1';\tenv,\tenv_1'[\suty/\alpha];\suty)}
\end{array}
\end{equation*}
  
\end{minipage}
}}

\begin{proof}
\begin{description}
\setlength{\itemsep}{1em}
%===============================================================================
\item[\fbox{\texttt{(L-RuleMatch)}}]\quad$
  \bar{\alpha},\alpha,\bar{\alpha}'; \tenv,\alpha,\tenv'; \tenv'',\rulet \leadsto x \ivturns \type \leadsto E
$ \ \\
%===============================================================================
  Then it follows from the first hypothesis of the rule and of Lemma~\ref{lemma:coherence:3}
  that
\begin{equation*}
\tenv,\tenv'[\suty/\alpha]; \rulet[\suty/\alpha] \leadsto E[|\suty|/\alpha] \ivturns \type[\suty/\alpha] \leadsto E'[|\suty|/\alpha]; \bar{\rulet}[\suty/\alpha] \leadsto \bar{x}
\end{equation*}

  Also it follows from the second hypothesis of the rule and of Lemma~\ref{lemma:coherence:1}
  that
\begin{equation*}
  \bar{\alpha},\bar{\alpha}'; \tenv,\tenv'[\suty/\alpha] \ivturns \bar{\rulet}[\suty/\alpha] \leadsto \bar{E}[|\suty|/\alpha]
\end{equation*}

  By combining these two observations with rule \mylabel{L-RuleMatch} we obtain the first desired result
\begin{equation*}
  \bar{\alpha},\bar{\alpha}'; \tenv,\tenv'[\suty/\alpha]; \tenv''',\rulet[\suty/\alpha] \leadsto x \ivturns \type[\suty/\alpha] \leadsto E[|\suty|/\alpha]
\end{equation*}
  
  We obtain the second desired result by case analysis on $\tenv'' \subseteq \tenv,\alpha,\tenv'$:
  \begin{enumerate}
  \item $\tenv = \tenv_1,\rulet \leadsto x,\tenv_2 \quad\wedge\quad \tenv'' = \tenv_1$: \\
  In this case we can use rule \mylabel{R-1} to establish:
\begin{equation*}
  R((\tenv_1,\rulet \leadsto x),\tenv_2;\alpha;\tenv';\tenv_1,\rulet \leadsto x;\tenv_1,\rulet \leadsto x;\sigma)
\end{equation*}
  which is equivalent to
\begin{equation*}
  R(\tenv;\alpha;\tenv';\tenv'',\rulet \leadsto x;\tenv'',\rulet \leadsto x;\sigma)
\end{equation*}

  \item $\tenv' = \tenv_1',\rulet \leadsto x, \tenv_2' \quad\wedge\quad \tenv'' = \tenv,\alpha,\tenv_1'$: \\
  In this case we can use rule \mylabel{R-2} to establish:
\begin{equation*}
  R(\tenv;\alpha;(\tenv_1',\rulet \leadsto x),\tenv_2';\tenv,\alpha,(\tenv_1',\rulet \leadsto x);
  \tenv,(\tenv_1',\rulet \leadsto x)[\suty/\alpha];\sigma)
\end{equation*}
  which is equivalent to
\begin{equation*}
  R(\tenv;\alpha;\tenv';\tenv'',\rulet \leadsto x;\tenv,(\tenv_1',\rulet \leadsto x)[\suty/\alpha];\sigma)
\end{equation*}
  \end{enumerate}

%===============================================================================
\item[\fbox{\texttt{(L-RuleNoMatch)}}]\quad$\bar{\alpha},\alpha,\bar{\alpha}'; \tenv,\alpha,\tenv'; \tenv'',\rulet \leadsto x \ivturns \type \leadsto E'$ \ \\
%===============================================================================
  The rule's first hypothesis states that
\begin{equation*}
  \not\exists \theta, E, \Sigma, \mathit{dom}(\theta) \subseteq (\bar{\alpha},\alpha,\bar{\alpha}'): \theta(\tenv,\alpha,\tenv'); \theta(\rulet)\leadsto x \ivturns \theta(\tau)\leadsto E; \Sigma
\end{equation*}
  Hence, the above also holds when we restrict $\theta$ to be of the form
$\theta' \cdot [\suty/\alpha]$. In this case, the above simplifies to
\begin{equation*}
  \not\exists \theta', E, \Sigma, \mathit{dom}(\theta) \subseteq (\bar{\alpha},\bar{\alpha}'): \theta'(\tenv,\tenv'[\suty/\alpha]); \theta'(\rulet[\suty/\alpha])\leadsto x \ivturns \theta'(\tau[\suty/\alpha])\leadsto E; \Sigma
\end{equation*}

  From the rule's second hypothesis and the induction hypothesis we have
\begin{equation*}
  \bar{\alpha},\bar{\alpha}';\tenv,\tenv'[\suty/\alpha];\tenv''' \ivturns \type[\suty/\alpha] \leadsto E'[|\suty|/\alpha]
\end{equation*}

  With rule \mylabel{L-RuleNoMatch} we combine these two observations into the desired first result
\begin{equation*}
  \bar{\alpha},\bar{\alpha}';\tenv,(\tenv',\rulet \leadsto x)[\suty/\alpha];\tenv''' \ivturns \type[\suty/\alpha] \leadsto E'[|\suty|/\alpha]
\end{equation*}

  Similarly, following the rule's second hypothesis and the induction hypothesis we have:
\begin{equation*}
  R(\tenv;\alpha;\tenv';\tenv'';\tenv''';\suty)
\end{equation*}

  We do a case analysis on the derivation of this judgement.
  \begin{enumerate}
  \item \mylabel{R-1}: \\ Then we have
\begin{equation*}
  \tenv = \tenv_1,x:\rulet,\tenv_2  \quad\wedge\quad \tenv'' = \tenv_1 \quad\wedge\quad \tenv''' = \tenv_1
\end{equation*}
       By rule \mylabel{R-2} we then have
\begin{equation*}
R((\tenv_1,x:\rulet),\tenv_2;\alpha;\tenv';\tenv_1,\rulet \leadsto x;\tenv_1,\rulet\leadsto x;\suty)
\end{equation*}
        which, given all the equations we have, is equivalent to
\begin{equation*}
R(\tenv;\alpha;\tenv';\tenv'',\rulet\leadsto x;\tenv'',\rulet\leadsto x;\suty)
\end{equation*}

  \item \mylabel{R-2}: \\
   Then we have
\begin{equation*}
  \tenv' = \tenv_1',\tenv_2'  \quad\wedge\quad \tenv'' = \tenv,\alpha,\tenv_1' \quad\wedge\quad 
      \tenv''' = \tenv,\tenv_1'[\suty/\alpha]
\end{equation*}

  Since $\tenv'',\rulet \leadsto x \subseteq \tenv,\alpha,\tenv'$, it follows that $\tenv_2' = \rulet\leadsto x,\tenv_{2,2}'$.
  Hence, by rule $\mylabel{R-2}$ we can establish
\begin{equation*}
R(\tenv;\alpha;(\tenv_1',\rulet\leadsto x),\tenv_2';\tenv,\alpha,(\tenv_1',\rulet \leadsto x);\tenv,(\tenv_1',\rulet\leadsto x)[\suty/\alpha];\suty)
\end{equation*}
        which, given all the equations we have, is equivalent to
\begin{equation*}
R(\tenv;\alpha;\tenv';\tenv'',\rulet\leadsto x;\tenv,(\tenv_1',\rulet \leadsto x)[\suty/\alpha];\suty)
\end{equation*}
  \end{enumerate}
%===============================================================================
\item[\fbox{\texttt{(L-Var)}}]\quad$\bar{\alpha},\alpha,\bar{\alpha}';\tenv,\alpha,\tenv';\tenv'', x : \rulet  \ivturns \type \leadsto E$ \ \\
%===============================================================================
  Then following the rule's hypothesis and the induction hypothesis we have:
\begin{equation*}
  \bar{\alpha},\bar{\alpha}';\tenv,\tenv'[\suty/\alpha];\tenv''' \ivturns \type[\suty/\alpha] \leadsto E[|\suty|/\alpha]
\end{equation*}
  By rule \mylabel{L-Var} and the definition of substitution we then have
\begin{equation*}
  \bar{\alpha},\bar{\alpha}';\tenv,\tenv'[\suty/\alpha];\tenv''',x:\rulet[\suty/\alpha] \ivturns \type[\suty/\alpha] \leadsto E[|\suty|/\alpha]
\end{equation*}

  Similarly, following the rule's hypothesis and the induction hypothesis we have:
\begin{equation*}
  R(\tenv;\alpha;\tenv';\tenv'';\tenv''';\suty)
\end{equation*}

  We do a case analysis on the derivation of this judgement.
  \begin{enumerate}
  \item \mylabel{R-1}: \\ Then we have
\begin{equation*}
  \tenv = \tenv_1,x:\rulet,\tenv_2  \quad\wedge\quad \tenv'' = \tenv_1 \quad\wedge\quad \tenv''' = \tenv_1
\end{equation*}
       By rule \mylabel{R-2} we then have
\begin{equation*}
R((\tenv_1,x:\rulet),\tenv_2;\alpha;\tenv';\tenv_1,x:\rulet;\tenv_1,x:\rulet;\suty)
\end{equation*}
        which, given all the equations we have, is equivalent to
\begin{equation*}
R(\tenv;\alpha;\tenv';\tenv'',x:\rulet;\tenv'',x:\rulet;\suty)
\end{equation*}

  \item \mylabel{R-2}: \\
   Then we have
\begin{equation*}
  \tenv' = \tenv_1',\tenv_2'  \quad\wedge\quad \tenv'' = \tenv,\alpha,\tenv_1' \quad\wedge\quad 
      \tenv''' = \tenv,\tenv_1'[\suty/\alpha]
\end{equation*}

  Since $\tenv'',x:\rulet \subseteq \tenv,\alpha,\tenv'$, it follows that $\tenv_2' = x:\rulet,\tenv_{2,2}'$.
  Hence, by rule $\mylabel{R-2}$ we can establish
\begin{equation*}
R(\tenv;\alpha;(\tenv_1',x:\rulet),\tenv_2';\tenv,\alpha,(\tenv_1',x:\rulet);\tenv,(\tenv_1',x:\rulet)[\suty/\alpha];\suty)
\end{equation*}
        which, given all the equations we have, is equivalent to
\begin{equation*}
R(\tenv;\alpha;\tenv';\tenv'',x:\rulet;\tenv,(\tenv_1',x:\rulet)[\suty/\alpha];\suty)
\end{equation*}
  \end{enumerate}


%===============================================================================
\item[\fbox{\texttt{(L-TyVar)}}]\quad$\bar{\alpha},\alpha,\bar{\alpha}';\tenv,\alpha,\tenv';\tenv'',\beta \ivturns \type \leadsto E$ \ \\
%===============================================================================
  Then following the rule's hypothesis and the induction hypothesis we have:
\begin{equation*}
  \bar{\alpha},\bar{\alpha}';\tenv,\tenv'[\suty/\alpha];\tenv''' \ivturns \type[\suty/\alpha] \leadsto E[|\suty|/\alpha]
\end{equation*}
  By rule \mylabel{L-TyVar} and the definition of substitution we then have
\begin{equation*}
  \bar{\alpha},\bar{\alpha}';\tenv,\tenv'[\suty/\alpha];\tenv''',\beta \ivturns \type[\suty/\alpha] \leadsto E[|\suty|/\alpha]
\end{equation*}

  Similarly, following the rule's hypothesis and the induction hypothesis we have:
\begin{equation*}
  R(\tenv;\alpha;\tenv';\tenv'';\tenv''';\suty)
\end{equation*}

  We do a case analysis on the derivation of this judgement.
  \begin{enumerate}
  \item \mylabel{R-1}: \\ Then we have
\begin{equation*}
  \tenv = \tenv_1,\tenv_2  \quad\wedge\quad \tenv'' = \tenv_1 \quad\wedge\quad \tenv''' = \tenv_1
\end{equation*}
  We further distinguish between two mutually exclusive cases:
  \begin{enumerate}
  \item $\tenv_2 = \epsilon$ \\
        It follows that $\alpha = \beta$ and we can establish by means of \mylabel{R-2} that
\begin{equation*}
R(\tenv_1,\tenv_2;\alpha;\epsilon,\tenv';\tenv_1,\tenv_2,\alpha;\tenv_1,\tenv_2,\epsilon[\suty/\alpha];\suty)
\end{equation*}
        which, given all the equations we have, is equivalent to
\begin{equation*}
R(\tenv;\alpha;\tenv';\tenv'',\beta;\tenv;\suty)
\end{equation*}

  \item $\tenv_2 \neq \epsilon$ \\
       Then it follows that $\tenv_2 = \beta,\tenv_{2,2}$ and by rule \mylabel{R-2} we have
\begin{equation*}
R((\tenv_1,\beta),\tenv_{2,2};\alpha;\tenv';\tenv_1,\beta;\tenv_1,\beta;\suty)
\end{equation*}
        which, given all the equations we have, is equivalent to
\begin{equation*}
R(\tenv;\alpha;\tenv';\tenv'',\beta;\tenv_1,\beta;\suty)
\end{equation*}
  \end{enumerate}

  \item \mylabel{R-2}: \\
   Then we have
\begin{equation*}
  \tenv' = \tenv_1',\tenv_2'  \quad\wedge\quad \tenv'' = \tenv,\alpha,\tenv_1' \quad\wedge\quad 
      \tenv''' = \tenv,\tenv_1'[\suty/\alpha]
\end{equation*}

  Since $\tenv'',\beta \subseteq \tenv,\alpha,\tenv'$, it follows that $\tenv_2' = \beta,\tenv_{2,2}'$.
  Hence, by rule $\mylabel{R-2}$ we can establish
\begin{equation*}
R(\tenv;\alpha;(\tenv_1',\beta),\tenv_2';\tenv,\alpha,(\tenv_1',\beta);\tenv,(\tenv_1',\beta)[\suty/\alpha];\suty)
\end{equation*}
        which, given all the equations we have, is equivalent to
\begin{equation*}
R(\tenv;\alpha;\tenv';\tenv'',\beta;\tenv,(\tenv_1',\beta)[\suty/\alpha];\suty)
\end{equation*}
  \end{enumerate}

\end{description}
\end{proof}

%###############################################################################
{\centering
\fbox{
\begin{minipage}{0.95\columnwidth}
\begin{lemma}\label{lemma:coherence:3}
  If 
\begin{equation*}
   \tenv,\alpha,\tenv'; \rulet \leadsto E \ivturns \type \leadsto E'; \bar{\rulet} \leadsto \bar{x}
\end{equation*}
  and
\begin{equation*}
  \tenv \turns \suty
\end{equation*}
  then
\begin{equation*}
   \tenv,\tenv'[\suty/\alpha]; \rulet[\suty/\alpha] \leadsto E[|\suty|/\alpha] \ivturns \type[\suty/\alpha] \leadsto E'[|\suty|/\alpha]; \bar{\rulet}[\suty/\alpha] \leadsto \bar{x}
\end{equation*}
\end{lemma}
\end{minipage}
}}

\begin{proof}
\begin{description}
\setlength{\itemsep}{1em}
%===============================================================================
\item[\fbox{\texttt{(M-Simp)}}]\quad$\tenv,\alpha,\tenv'; \type \leadsto E \ivturns \type \leadsto E; \epsilon$ \ \\
%===============================================================================
  The desired conclusion follows directly from rule \mylabel{M-Simp}
\begin{equation*}
  \tenv,\tenv'[\suty/\alpha]; \type[\suty/\alpha] \leadsto E[|\suty/\alpha|] \ivturns \type[\suty/\alpha] \leadsto E[|\suty/\alpha|]; \epsilon
\end{equation*}

%===============================================================================
\item[\fbox{\texttt{(M-IApp)}}]\quad$\tenv,\alpha,\tenv'; \rulet_1 \iarrow \rulet_2 \leadsto E \ivturns \type \leadsto E'; \Sigma, \rulet_1 \leadsto x$ \ \\
%===============================================================================
  From the rule's hypothesis and the induction hypothesis we have 
\begin{equation*}
  \tenv,(\tenv',\rulet_1 \leadsto x)[\suty/\alpha]; \rulet_2[\suty/\alpha] \leadsto (E x)[|\suty|/\alpha] \ivturns \type[\suty/\alpha] \leadsto E'[|\suty|/\alpha]; \Sigma[\suty/\alpha]
\end{equation*}

  Then from the definition of substutition and rule \mylabel{M-IApp} we conclude
\begin{equation*}
  \tenv,\tenv'[\suty/\alpha]; (\rulet_1 \iarrow \rulet_2)[\suty/\alpha] \leadsto E[|\suty|/\alpha] \ivturns \type[\suty/\alpha] \leadsto E'[|\suty|/\alpha]; (\Sigma,\rulet_1 \leadsto x)[\suty/\alpha]
\end{equation*}

%===============================================================================
\item[\fbox{\texttt{(M-TApp)}}]\quad$\tenv,\alpha,\tenv'; \forall\beta.\rulet \leadsto E \ivturns \type \leadsto E'; \Sigma$ \ \\
%===============================================================================
  From the rule's first hypothesis and the induction hypothesis we have 
\begin{equation*}
  \tenv,\tenv'[\suty/\alpha]; \rulet[\suty'/\beta][\suty/\alpha] \leadsto (E |\suty'|)[|\suty|/\alpha] \ivturns \type[\suty/\alpha] \leadsto E'[|\suty|/\alpha]; \Sigma[\suty/\alpha]
\end{equation*}

  From the rule's second hypothesis (and the preservation of well-typing under type-susbstitution) we have
\begin{equation*}
  \tenv,\tenv'[\suty/\alpha] \vdash \suty'[\suty'/\alpha]
\end{equation*}

  From these two facts we conclude by rule \mylabel{M-TApp}, reasoning modulo
  the definition of substitution
\begin{equation*}
  \tenv,\tenv'[\suty/\alpha]; (\forall\beta.\rulet)[\suty/\alpha] \leadsto E[|\suty|/\alpha] \ivturns \type[\suty/\alpha] \leadsto E'[|\suty|/\alpha]; \Sigma[\suty/\alpha]
\end{equation*}

\end{description}
\end{proof}

%-------------------------------------------------------------------------------
\subsection{Soundness of the Algorithm wrt Deterministic Resolution}

%###############################################################################
{\centering
\fbox{
\begin{minipage}{0.95\columnwidth}
\begin{lemma}\label{lemma:asoundness:0}
  If 
\begin{equation*}
  \tenv \alg \rulet \leadsto E
\end{equation*}
  then
\begin{equation*}
  \tenv \ivturns \rulet \leadsto E
\end{equation*}
\end{lemma}
\end{minipage}
}}

\begin{proof}
  From the hypothesis it follows that 
\begin{equation*}
  \mathit{tyvars}(\tenv);\tenv \alg \rulet \leadsto E
\end{equation*}
  Hence, by Lemma~\ref{lemma:asoundness:1} and rule \mylabel{R-Main} the desired conclusion follows
\begin{equation*}
  \tenv \ivturns \rulet \leadsto E
\end{equation*}
\end{proof}

%###############################################################################
{\centering
\fbox{
\begin{minipage}{0.95\columnwidth}
\begin{lemma}\label{lemma:asoundness:1}
  If 
\begin{equation*}
  \bar{\alpha};\tenv \alg \rulet \leadsto E
\end{equation*}
  then
\begin{equation*}
  \bar{\alpha};\tenv \ivturns \rulet \leadsto E
\end{equation*}
\end{lemma}
\end{minipage}
}}

\begin{proof}
The lemma follows from the isomorphism between the 
rule sets of the two judgements and from Lemma~\ref{lemma:asoundness:2}.
\end{proof}

%###############################################################################
{\centering
\fbox{
\begin{minipage}{0.95\columnwidth}
\begin{lemma}\label{lemma:asoundness:2}
  If 
\begin{equation*}
  \bar{\alpha};\tenv;\tenv' \alg \rulet \leadsto E
\end{equation*}
  then
\begin{equation*}
  \bar{\alpha};\tenv;\tenv' \ivturns \rulet \leadsto E
\end{equation*}
\end{lemma}
\end{minipage}
}}

\begin{proof}
The proof proceeds by induction on the derivation of the hypothesis.
\begin{description}
\setlength{\itemsep}{1em}
%===============================================================================
\item[\fbox{\texttt{(AL-RuleMatch)}}]\quad$\bar{\alpha};\tenv; \tenv', \rulet~{\leadsto x} \alg \type~{\leadsto E[\bar{E}/\bar{x}]}$ \ \\
%===============================================================================
From the rule's first hypothesis and Lemma~\ref{lemma:asoundness:3} we have
\begin{equation*}
\tenv; \rulet \leadsto E \ivturns \type \leadsto E'; \bar{\rulet} \leadsto \bar{x}
\end{equation*}
Then, using Lemma~\ref{lemma:asoundness:1} and rule \mylabel{L-RuleMatch} we conclude
\begin{equation*}
\bar{\alpha};\tenv; \tenv', \rulet~{\leadsto x} \ivturns \type~{\leadsto E[\bar{E}/\bar{x}]}
\end{equation*}

%===============================================================================
\item[\fbox{\texttt{(AL-RuleNoMatch)}}]\quad$\bar{\alpha};\tenv;\tenv', \rulet~{\leadsto x}\alg \type~{\leadsto E'}$ \ \\
%===============================================================================
  From the rule's second hypothesis and the induction hypothesis we have
\begin{equation*}
  \bar{\alpha};\tenv;\tenv' \ivturns \type~{\leadsto E'}
\end{equation*}

  Then from the rule's first hypothesis and the negation of Lemma~\ref{lemma:asoundness:5}, we have:
\begin{equation*}
\not\exists E, \Sigma: \bar{\alpha};\tenv;\rulet \leadsto x; \epsilon \alg \type \leadsto E; \Sigma
\end{equation*}
  By Lemma~\ref{lemma:asoundness:3} we thus have
\begin{equation*}
\not\exists \theta, E, \Sigma, \mathit{dom}(\theta) \subseteq \bar{\alpha}: \theta(\tenv);\theta(\rulet) \leadsto x \ivturns \type \leadsto E; \Sigma
\end{equation*}
  Hence with rule \mylabel{L-RuleNoMatch} we conclude
\begin{equation*}
  \bar{\alpha};\tenv;\tenv',\rulet \leadsto x \ivturns \type~{\leadsto E'}
\end{equation*}

%===============================================================================
\item[\fbox{\texttt{(AL-Var)}}]\quad$\bar{\alpha};\tenv;\tenv',x:\rulet \alg \type~{\leadsto E}$ \ \\
%===============================================================================
From the rule's hypothesis and the induction hypothesis we obtain
\begin{equation*}
  \bar{\alpha};\tenv;\tenv' \ivturns \type~{\leadsto E}
\end{equation*}
By rule \mylabel{L-Var} we conclude
\begin{equation*}
  \bar{\alpha};\tenv;\tenv',x:\rulet \ivturns \type~{\leadsto E}
\end{equation*}

%===============================================================================
\item[\fbox{\texttt{(AL-TyVar)}}]\quad$\bar{\alpha};\tenv;\tenv',\alpha \alg \type~{\leadsto E}$ \ \\
%===============================================================================
From the rule's hypothesis and the induction hypothesis we obtain
\begin{equation*}
  \bar{\alpha};\tenv;\tenv' \ivturns \type~{\leadsto E}
\end{equation*}
By rule \mylabel{L-TyVar} we conclude
\begin{equation*}
  \bar{\alpha};\tenv;\tenv',\alpha \ivturns \type~{\leadsto E}
\end{equation*}

\end{description}
\end{proof}

We assume that the judgement is decorated with an additional argument, the substitution
for the $\bar{\alpha}$ type variables.
%###############################################################################
{\centering
\fbox{
\begin{minipage}{0.95\columnwidth}
\begin{lemma}\label{lemma:asoundness:3}
  If 
\begin{equation*}
  \bar{\alpha};\tenv;\rulet \leadsto E; \Sigma \alg \type \leadsto E'; \Sigma',\theta(\Sigma); \theta
\end{equation*}
  and
\begin{equation*}
  \mathit{dom}(\theta) \subseteq \bar{\alpha}
\end{equation*}
  then
\begin{equation*}
  \theta(\tenv); \theta(\rulet) \leadsto |\theta|(E) \ivturns \theta(\type) \leadsto E'; \Sigma'
\end{equation*}
\end{lemma}
\end{minipage}
}}

\begin{proof}
The proof proceeds by induction on the derivation of the first hypothesis.
\begin{description}
\setlength{\itemsep}{1em}
%===============================================================================
\item[\fbox{\texttt{(AM-Simp)}}]\quad$\bar{\alpha}; \tenv; \type' \leadsto E; \Sigma \alg \type \leadsto |\theta|(E); \epsilon,\theta(\Sigma);\theta$ \ \\
%===============================================================================
From the hypothesis of the rule and Lemma~\ref{lemma:asoundness:6} it follows that
$\theta(\type') = \theta(\type)$. Hence, the target judgement can be rewritten as
\begin{equation*}
  \theta(\tenv); \theta(\type') \leadsto |\theta|(E) \ivturns \theta(\type') \leadsto |\theta|(E); \epsilon
\end{equation*}
This follows from rule \mylabel{M-Simp}.

%===============================================================================
\item[\fbox{\texttt{(AM-IApp)}}]\quad$\bar{\alpha}; \tenv; \rulet_1 \iarrow \rulet_2~{\leadsto E}; \Sigma \alg \type~{\leadsto E'}; \Sigma',\theta(\rulet_1) \leadsto x, \theta(\Sigma);\theta$ \ \\
%===============================================================================
  From the hypothesis of the rule and the induction hypothesis, we have that
\begin{equation*}
 \theta(\tenv,\rulet_1 \leadsto x); \theta(\rulet_2) \leadsto |\theta|(E\,x) \ivturns \theta(\type) \leadsto E'; \Sigma' 
\end{equation*}
  By rule \mylabel{M-IApp} we may then conclude
\begin{equation*}
 \theta(\tenv); \theta(\rulet_1 \iarrow \rulet_2) \leadsto |\theta|(E) \ivturns \theta(\type) \leadsto E'; \Sigma',\theta(\rulet_1) \leadsto x
\end{equation*} 

%===============================================================================
\item[\fbox{\texttt{(AM-TApp)}}]\quad$\bar{\alpha}; \tenv; \forall \alpha. \rulet~{\leadsto E}; \Sigma \alg \type~{\leadsto E'}; \Sigma',\theta(\Sigma);\theta$ \ \\
%===============================================================================
  Then it follows from the rule's hypothesis and from the induction hypothesis that
\begin{equation*}
  \theta(\tenv); \theta(\rulet) \leadsto |\theta|(E\,\alpha) \ivturns \theta(\type) \leadsto E'; \Sigma'
\end{equation*}
  Hence, it follows from rule \mylabel{M-TApp} that
\begin{equation*}
  \theta(\tenv); \theta(\forall \alpha.\rulet) \leadsto |\theta|(E) \ivturns \theta(\type) \leadsto E'; \Sigma'
\end{equation*}

\end{description}
\end{proof}


%###############################################################################
{\centering
\fbox{
\begin{minipage}{0.95\columnwidth}
\begin{lemma}\label{lemma:asoundness:5}
  If 
\begin{equation*}
  \bar{\alpha};\tenv;\rulet \leadsto E; \Sigma \alg \type \leadsto E'; \Sigma'
\end{equation*}
  then
\begin{equation*}
  \bar{\alpha}; \rulet \coh \type
\end{equation*}
\end{lemma}
\end{minipage}
}}

\begin{proof}
  The derivation of the conclusion is obtained by erasing the irrelevant arguments
  from the derivation of the hypothesis.
\end{proof}


%###############################################################################
{\centering
\fbox{
\begin{minipage}{0.95\columnwidth}
\begin{lemma}\label{lemma:asoundness:6}
  If 
\begin{equation*}
  \theta = \mgu{\type}{\type'}
\end{equation*}
  then
\begin{equation*}
  \theta(\type) = \theta(\type')
\end{equation*}
  and
\begin{equation*}
  \mathit{dom}(\theta) \subseteq \bar{\alpha}
\end{equation*}

\end{lemma}
\end{minipage}
}}

\begin{proof}
Straightforward induction on the derivation.
\end{proof}

%-------------------------------------------------------------------------------
\subsection{Completeness of the Algorithm wrt Deterministic Resolution}

%###############################################################################
{\centering
\fbox{
\begin{minipage}{0.95\columnwidth}
\begin{lemma}\label{lemma:acompleteness:0}
  If 
\begin{equation*}
  \tenv \ivturns \rulet \leadsto E
\end{equation*}
  then
\begin{equation*}
  \tenv \alg \rulet \leadsto E
\end{equation*}
\end{lemma}
\end{minipage}
}}

\begin{proof}
  From the hypothesis it follows that 
\begin{equation*}
  \mathit{tyvars}(\tenv);\tenv \ivturns \rulet \leadsto E
\end{equation*}
  Hence, by Lemma~\ref{lemma:acompleteness:1} and rule \mylabel{AR-Main} the desired conclusion follows
\begin{equation*}
  \tenv \alg \rulet \leadsto E
\end{equation*}
\end{proof}

%###############################################################################
{\centering
\fbox{
\begin{minipage}{0.95\columnwidth}
\begin{lemma}\label{lemma:acompleteness:1}
  If 
\begin{equation*}
  \bar{\alpha};\tenv \ivturns \rulet \leadsto E
\end{equation*}
  then
\begin{equation*}
  \bar{\alpha};\tenv \alg \rulet \leadsto E
\end{equation*}
\end{lemma}
\end{minipage}
}}

\begin{proof}
The lemma follows from the isomorphism between the 
rule sets of the two judgements and from Lemma~\ref{lemma:acompleteness:2}.
\end{proof}

%###############################################################################
{\centering
\fbox{
\begin{minipage}{0.95\columnwidth}
\begin{lemma}\label{lemma:acompleteness:2}
  If 
\begin{equation*}
  \bar{\alpha};\tenv;\tenv' \ivturns \rulet \leadsto E
\end{equation*}
  then
\begin{equation*}
  \bar{\alpha};\tenv;\tenv' \alg \rulet \leadsto E
\end{equation*}
\end{lemma}
\end{minipage}
}}

\begin{proof}
The proof proceeds by induction on the derivation of the hypothesis.
\begin{description}
\setlength{\itemsep}{1em}
%===============================================================================
\item[\fbox{\texttt{(L-RuleMatch)}}]\quad$\bar{\alpha};\tenv; \tenv', \rulet~{\leadsto x} \ivturns \type~{\leadsto E[\bar{E}/\bar{x}]}$ \ \\
%===============================================================================
From the rule's first hypothesis and Lemma~\ref{lemma:acompleteness:3} we have
\begin{equation*}
\epsilon; \tenv; \rulet \leadsto x; \epsilon \alg \type \leadsto E'; \bar{\rulet} \leadsto \bar{x}
\end{equation*}
Then, using Lemma~\ref{lemma:acompleteness:1} and rule \mylabel{AL-RuleMatch} we conclude
\begin{equation*}
\bar{\alpha};\tenv; \tenv', \rulet~{\leadsto x} \alg \type~{\leadsto E[\bar{E}/\bar{x}]}
\end{equation*}

%===============================================================================
\item[\fbox{\texttt{(L-RuleNoMatch)}}]\quad$\bar{\alpha};\tenv;\tenv', \rulet~{\leadsto x}\ivturns \type~{\leadsto E'}$ \ \\
%===============================================================================
  From the rule's second hypothesis and the induction hypothesis we have
\begin{equation*}
  \bar{\alpha};\tenv;\tenv' \alg \type~{\leadsto E'}
\end{equation*}

  From the rule's first hypothesis and the negation of Lemma~\ref{lemma:acompleteness:4}, we have:
\begin{equation*}
\bar{\alpha}; \rulet \not\coh \type
\end{equation*}
  Hence with rule \mylabel{AL-RuleNoMatch} we conclude
\begin{equation*}
  \bar{\alpha};\tenv;\tenv',\rulet \leadsto x \alg \type~{\leadsto E'}
\end{equation*}

%===============================================================================
\item[\fbox{\texttt{(L-Var)}}]\quad$\bar{\alpha};\tenv;\tenv',x:\rulet \ivturns \type~{\leadsto E}$ \ \\
%===============================================================================
From the rule's hypothesis and the induction hypothesis we obtain
\begin{equation*}
  \bar{\alpha};\tenv;\tenv' \alg \type~{\leadsto E}
\end{equation*}
By rule \mylabel{AL-Var} we conclude
\begin{equation*}
  \bar{\alpha};\tenv;\tenv',x:\rulet \alg \type~{\leadsto E}
\end{equation*}

%===============================================================================
\item[\fbox{\texttt{(L-TyVar)}}]\quad$\bar{\alpha};\tenv;\tenv',\alpha \ivturns \type~{\leadsto E}$ \ \\
%===============================================================================
From the rule's hypothesis and the induction hypothesis we obtain
\begin{equation*}
  \bar{\alpha};\tenv;\tenv' \alg \type~{\leadsto E}
\end{equation*}
By rule \mylabel{AL-TyVar} we conclude
\begin{equation*}
  \bar{\alpha};\tenv;\tenv',\alpha \alg \type~{\leadsto E}
\end{equation*}

\end{description}
\end{proof}

%###############################################################################
{\centering
\fbox{
\begin{minipage}{0.95\columnwidth}
\begin{lemma}\label{lemma:acompleteness:3}
  If 
\begin{equation*}
  \theta_1(\tenv); \theta_1(\rulet) \leadsto |\theta_1|(E) \ivturns \theta_1(\type) \leadsto |\theta_1|(E'); \theta_1(\Sigma')
\end{equation*}
  and
\begin{equation*}
  \mathit{dom}(\theta_1) \subseteq \bar{\alpha}
\end{equation*}
  then
\begin{equation*}
  \bar{\alpha};\tenv;\rulet \leadsto E; \Sigma \alg \type \leadsto |\theta_2|(E'); \theta_2(\Sigma',\Sigma)
\end{equation*}
  and
\begin{equation*}
  \mathit{dom}(\theta_2) \subseteq \bar{\alpha}
\end{equation*}
  and
\begin{equation*}
  \theta_1 \sqsubseteq \theta_2
\end{equation*}
\end{lemma}
\end{minipage}
}}

\begin{proof}
The proof proceeds by induction on the derivation of the first hypothesis.
\begin{description}
\setlength{\itemsep}{1em}
%===============================================================================
\item[\fbox{\texttt{(M-Simp)}}]\quad$\theta_1(\tenv); \theta_1(\type') \leadsto |\theta_1|(E) \ivturns \theta_1(\type) \leadsto |\theta_1|(E); \theta_1(\epsilon)$ \hfill where $\theta_1(\tenv) = \theta_1(\type')$. \\
%===============================================================================

From Lemma~\ref{lemma:acompleteness:5} and rule \mylabel{AM-Simp} we then have
\begin{equation*}
  \bar{\alpha}; \tenv; \type' \leadsto E; \Sigma \alg \type \leadsto \theta_2(E); \theta_2(\Sigma)
\end{equation*}

%===============================================================================
\item[\fbox{\texttt{(M-IApp)}}]\quad$\theta_1(\tenv); \theta_1(\rulet_1 \iarrow \rulet_2)~{\leadsto |\theta_1|(E)}\ivturns \theta_1(\type)~{\leadsto |\theta_1|(E')}; \theta_1(\Sigma',\rulet_1 \leadsto x)$ \ \\
%===============================================================================
  From the hypothesis of the rule and the induction hypothesis, we have that
\begin{equation*}
 \bar{\alpha};\tenv,\rulet_1 \leadsto x;\rulet_2 \leadsto E\,x; \rulet_1 \leadsto x, \Sigma \alg \type \leadsto |\theta_2|(E'); \theta_2(\Sigma',\rulet_1 \leadsto x, \Sigma) 
\end{equation*}
  By rule \mylabel{AM-IApp} we may then conclude
\begin{equation*}
 \bar{\alpha};\tenv;\rulet_1 \iarrow \rulet_2 \leadsto E; \Sigma \alg \type \leadsto |\theta_2|(E'); \theta_2(\Sigma',\rulet_1 \leadsto x, \Sigma) 
\end{equation*} 

%===============================================================================
\item[\fbox{\texttt{(M-TApp)}}]\quad$\theta_1(\tenv); \theta_1(\forall\alpha.\rulet) \leadsto |\theta_1|(E) \ivturns \theta_1(\type) \leadsto |\theta_1|(E'); \theta_1(\Sigma')$ \ \\
%===============================================================================
  From the hypothesis of the rule and the induction hypothesis, we have that
\begin{equation*}
 \bar{\alpha},\alpha;\tenv;\rulet \leadsto E\,\alpha; \Sigma \alg \type \leadsto |\theta_2|(E'); \theta_2(\Sigma',\Sigma) 
\end{equation*}
  Hence, it follows from rule \mylabel{AM-TApp} that
\begin{equation*}
 \bar{\alpha};\tenv;\forall \alpha.\rulet \leadsto E; \Sigma \alg \type \leadsto |\theta_2|(E'); \theta_2(\Sigma',\Sigma) 
\end{equation*}

\end{description}
\end{proof}

%###############################################################################
{\centering
\fbox{
\begin{minipage}{0.95\columnwidth}
\begin{lemma}\label{lemma:acompleteness:4}
  If 
\begin{equation*}
  \bar{\alpha}; \rulet \coh \type
\end{equation*}
  then for all $E, \tenv, \Sigma$ there exist $E', \Sigma'$ such that
\begin{equation*}
  \bar{\alpha};\tenv;\rulet \leadsto E; \Sigma \alg \type \leadsto E'; \Sigma'
\end{equation*}
\end{lemma}
\end{minipage}
}}

\begin{proof}
  The proof is straightforward induction on the derivation. The conclusion's judgement
  is an annotated version of the hypothesis' judgement.
\end{proof}

%###############################################################################
{\centering
\fbox{
\begin{minipage}{0.95\columnwidth}
\begin{lemma}\label{lemma:acompleteness:5}
  If 
\begin{equation*}
  \theta(\type) = \theta(\type')
\end{equation*}
  and
\begin{equation*}
  \mathit{dom}(\theta) \subseteq \bar{\alpha}
\end{equation*}
  then
\begin{equation*}
  \theta' = \mgu{\type}{\type'}
\end{equation*}
  and
\begin{equation*}
  \mathit{dom}(\theta') \subseteq \bar{\alpha}
\end{equation*}
  and
\begin{equation*}
  \theta \sqsubseteq \theta'
\end{equation*}
\end{lemma}
\end{minipage}
}}

% \newcommand{\vdashr}[4]{#1; #2 \vdash_r #3 \leadsto #4}
% 
% \begin{proof}
% The proof proceeds by induction on the $\bar{\alpha};\env \vdash_r \rho \leadsto E$ derivation.
% \begin{description}
% \renewcommand{\itemsep}{10mm}
% %= = = = = = = = = = = = = = = = = = = = = = = = = = = = = = = = = = = = = = = = 
% \item[\texttt{(R-IAbs)}] \quad
%     $\vdashr{\bar{\alpha}}{\env}{\rho_1 \iarrow \rho_2}{\lambda\relation{x}{||\rho_1||}.E}$ \ \\
% %= = = = = = = = = = = = = = = = = = = = = = = = = = = = = = = = = = = = = = = = 
% From rule \mylabel{R-IAbs} and the induction hypothesis, it follows that
% \begin{equation*}
% \vdashr{\bar{\alpha}}{\theta(\env,\rho_1\leadsto x)}{\theta(\rho_2)}{\theta(E)}
% \end{equation*}
% Hence, using rule \mylabel{R-IAbs}, we conclude
% \begin{equation*}
% \vdashr{\bar{\alpha}}{\theta(\env)}{\theta(\rho_1\iarrow\rho_2)}{\theta(\lambda\relation{x}{||\rho_1||}.E)}
% \end{equation*}
% %= = = = = = = = = = = = = = = = = = = = = = = = = = = = = = = = = = = = = = = = 
% \item[\texttt{(R-TAbs)}] \quad
%     $\vdashr{\bar{\alpha}}{\env}{\forall \alpha. \rho}{\Lambda\alpha.E}$ \ \\
% %= = = = = = = = = = = = = = = = = = = = = = = = = = = = = = = = = = = = = = = = 
% From rule \mylabel{R-TAbs} and the induction hypothesis, it follows that
% \begin{equation*}
% \vdashr{\bar{\alpha},\alpha}{\theta(\env)}{\theta(\rho)}{\theta(E)}
% \end{equation*}
% Hence, using rule \mylabel{R-TAbs}, we conclude
% \begin{equation*}
% \vdashr{\bar{\alpha}}{\theta(\env)}{\theta(\forall \alpha. \rho)}{\theta(\Lambda\alpha.E)}
% \end{equation*}
% %= = = = = = = = = = = = = = = = = = = = = = = = = = = = = = = = = = = = = = = = 
% \item[\texttt{(R-Simp)}] \quad
%     $\vdashr{\bar{\alpha}}{\env}{\tau}{E}$ \ \\
% %= = = = = = = = = = = = = = = = = = = = = = = = = = = = = = = = = = = = = = = = 
% We distinguish two cases. Firstly, we assume
% that $\theta(\tau)$ yields the simple type $\tau'$.
% Then, it follows from rule \mylabel{R-Simp} and Lemma~\ref{} (TODO) that
% \begin{equation*}
% \elookup{\theta(\env)}{\theta(\tau)} = \theta(\rho)~\leadsto x
% \end{equation*}
% Similarly, it follows from rule \mylabel{R-Simp} and Lemma~\ref{} (TODO) that
% \begin{equation*}
% \bar{\alpha};\theta(\env); \theta(\rho)~\leadsto x
%    \turns_\downarrow \theta(\type)~\leadsto \theta(E)
% \end{equation*}
% Hence, using rule \mylabel{R-Simp} we conclude
% \begin{equation*}
% \vdashr{\bar{\alpha}}{\theta(\env)}{\theta(\tau)}{\theta(E)}
% \end{equation*}
% 
% Secondly, we assume that $\theta(\tau) = \forall \bar{\beta}.\tau'$.
% This happens only if $\tau = \alpha$.
% In this case, we can show:
% \begin{equation*}
% \vdashr{\bar{\alpha}}{\theta(\env)}{\theta(\tau)}{E'} \quad \wedge \quad E' \equiv_{\alpha,H} \theta(E)
% \end{equation*}
% if we take $E' = \Lambda\bar{\beta}.\theta(E)\,\bar{\beta}$
% by repeated application of rule \mylabel{R-TAbs} and showing that
% \begin{equation*}
% \vdashr{\bar{\alpha},\bar{\beta}}{\theta(\env)}{\tau'}{\theta(E)\,\bar{\beta}}
% \end{equation*}
% 
% %   \item $\theta(\tau) = \tau'$
% %   \item $\theta(\tau) = \forall \bar{\alpha'}.\tau'$
% \end{description}
% \end{proof}


% \begin{lemma}\label{lemma:substitution:lhd}
% The $\rho \lhd \tau$ judgement is stable under substitution.
% \[\forall \rho, \tau, \alpha, \suty: 
%     \rho \lhd \tau \quad \rightarrow \quad \theta(\rho) \lhd \head{\theta(\tau)}
% \]
% where $\theta = [\suty/\alpha]$.
% \end{lemma}
% \begin{proof}
% The proof proceeds by induction on the $\rho \lhd \tau$ derivation.
% \begin{description}
% \renewcommand{\itemsep}{10mm}
% %= = = = = = = = = = = = = = = = = = = = = = = = = = = = = = = = = = = = = = = = 
% \item[\texttt{(M-IApp)}] \quad
%     $\rho_1 \iarrow \rho_2 \lhd \tau$ \ \\
% %= = = = = = = = = = = = = = = = = = = = = = = = = = = = = = = = = = = = = = = = 
% From rule \mylabel{M-IApp} and the induction hypothesis it follows that:
% \begin{equation*}
% \theta(\rho_2) \lhd \head{\theta(\tau)}
% \end{equation*}
% Then it follows from rule \mylabel{M-IApp} that:
% \begin{equation*}
% \theta(\rho_1) \iarrow \theta(\rho_2) \lhd \head{\theta(\tau)}
% \end{equation*}
% Hence, because $\theta(\rho_1) \iarrow \theta(\rho_2) = \theta(\rho_1 \iarrow \rho_2)$.
% we conclude:
% \begin{equation*}
% \theta(\rho_1 \iarrow \rho_2) \lhd \head{\theta(\tau)}
% \end{equation*}
% 
% %= = = = = = = = = = = = = = = = = = = = = = = = = = = = = = = = = = = = = = = = 
% \item[\texttt{(M-TApp)}] \quad
%     $\forall \alpha. \rho \lhd \tau$ \ \\
% %= = = = = = = = = = = = = = = = = = = = = = = = = = = = = = = = = = = = = = = = 
% From rule \mylabel{M-TApp} and the induction hypothesis it follows that:
% \begin{equation*}
% \theta(\rho[\suty'/\alpha']) \lhd \head{\theta(\tau)}
% \end{equation*}
% We can commute the two substitutions as follows because their
% domains are disjoint:
% \begin{equation*}
% \theta(\rho)[\theta(\suty')/\alpha'] \lhd \head{\theta(\tau)}
% \end{equation*}
% Then, by rule \mylabel{M-TApp}, we get:
% \begin{equation*}
% \forall \alpha. \theta(\rho) \lhd \head{\theta(\tau)}
% \end{equation*}
% Finally, because $\forall \alpha. \theta(\rho) = \theta(\forall \alpha. \rho)$,
% we conclude:
% \begin{equation*}
% \theta(\forall \alpha. \rho) \lhd \head{\theta(\tau)}
% \end{equation*}
% 
% %= = = = = = = = = = = = = = = = = = = = = = = = = = = = = = = = = = = = = = = = 
% \item[\texttt{(M-Simp)}] \quad
%     $\tau \lhd \tau$ \ \\
% %= = = = = = = = = = = = = = = = = = = = = = = = = = = = = = = = = = = = = = = = 
% We distinguish three cases:
% Firstly, we consider the case where $\tau = \rho_1 \arrow \rho_2$.
% Then, $\theta(\tau) = \head{\theta(\tau)}$ by rule \mylabel{M-Simp}
% we trivially have:
% \begin{equation*}
% \theta(\tau) \lhd \head{\theta(\tau)}
% \end{equation*}
% 
% Secondly, we consider the case where $\tau = \alpha' \neq \alpha$.
% Then, $\head(\theta(\tau)) = \head{\tau} = \tau$ and by rule 
% \mylabel{M-Simp} we trivally have:
% \begin{equation*}
% \theta(\tau) \lhd \head{\theta(\tau)}
% \end{equation*}
% 
% Thirdly, we consider the case where $\tau = \alpha$. Then $\theta(\tau) =
% \suty$.  We have that $\suty$ is of the general form $\forall
% \bar{\beta}.\tau'$. Hence, $\head{\theta(\tau)} = \tau'$.
% Then we can establish the desired goal:
% \begin{equation*}
% \forall \bar{\beta}.\tau' \lhd \tau'
% \end{equation*}
% by repeated application of rule \mylabel{M-TApp}
% choosing substitutions of the form $[\beta/\beta]$
% and the trivial base case $\tau' \lhd \tau'$ shown by rule
% \mylabel{M-Simp}.
% 
% \end{description}
% \end{proof}
% 
% \begin{lemma}\label{lemma:substitution:lookup}
% The $\elookup{\env}{\type} = \rho \leadsto x$ judgement is stable under substitution.
% \[\forall \env, \bar{\alpha}, \type, \rho, \alpha, \suty: 
%     \elookup{\env}{\type} = \rho \leadsto x
%     \quad\rightarrow\quad
%     \elookup{\theta(\env)}{\head{\theta(\type)}} = \theta(\rho) \leadsto x
% \]
% where $\theta = [\suty/\alpha]$.
% \end{lemma}
% \begin{proof}
% The proof proceeds by induction on the $\elookup{\env}{\type} = \rho \leadsto x$ derivation.
% \begin{description}
% \renewcommand{\itemsep}{10mm}
% %= = = = = = = = = = = = = = = = = = = = = = = = = = = = = = = = = = = = = = = = 
% \item[\texttt{(L-Head)}] \quad
%     $\elookup{(\env,\rho \leadsto x)}{\type} = \rho \leadsto x$ \ \\
% %= = = = = = = = = = = = = = = = = = = = = = = = = = = = = = = = = = = = = = = = 
% From rule \mylabel{L-Head} it follows that
% \begin{equation*}
% \rho \lhd \tau
% \end{equation*}
% Hence, from Lemma~\ref{lemma:substitution:lhd}, we know that:
% \begin{equation*}
% \theta(\rho) \lhd \head{\theta(\type)}
% \end{equation*}
% Thus we obtain with rule \mylabel{L-Head}:
% \begin{equation*}
% \elookup{(\theta(\env),\theta(\rho) \leadsto x)}{\head{\theta(\type)}} = \theta(\rho) \leadsto x
% \end{equation*}
% As $(\theta(\env),\theta(\rho) \leadsto x) = \theta(\env,\rho \leadsto x)$,
% this yields the desired result:
% \begin{equation*}
% \elookup{\theta(\env,\rho \leadsto x)}{\head{\theta(\type)}} = \theta(\rho) \leadsto x
% \end{equation*}
% 
% %= = = = = = = = = = = = = = = = = = = = = = = = = = = = = = = = = = = = = = = = 
% \item[\texttt{(L-Tail)}] \quad
%     $\elookup{(\env,\rho_1 \leadsto x)}{\type} = \rho_2 \leadsto y$ \ \\
% %= = = = = = = = = = = = = = = = = = = = = = = = = = = = = = = = = = = = = = = = 
% From rule \mylabel{L-Tail} and the induction hypothesis it follows that
% \begin{equation*}
% \elookup{\theta(\env)}{\head{\theta(\type)}} = \theta(\rho_2) \leadsto y
% \end{equation*}
% 
% Also from rule \mylabel{L-Tail} we have
% \begin{equation*}
% \forall \theta': 
% 	\theta'(\rulet_1) \mathop{\not\!\!\lhd} \head{\theta'(\type)}
% \end{equation*}
% As a consequence this also holds for every choice $\theta' = \theta'' \cdot \theta$.
% \begin{equation*}
% \forall \theta'': 
% 	\theta''(\theta(\rulet_1)) \mathop{\not\!\!\lhd} \head{\theta''(\theta(\type))}
% \end{equation*}
% Now we apply the property that $\mathrm{hd} \cdot \theta'' = \mathrm{hd} \cdot \theta'' \cdot \mathrm{hd}$:
% \begin{equation*}
% \forall \theta'': 
% 	\theta''(\theta(\rulet_1)) \mathop{\not\!\!\lhd} \head{\theta''(\head{\theta(\type)})}
% \end{equation*}
% We now have all the necessary ingredients to apply rule \mylabel{L-Tail}
% and to conclude:
% \begin{equation*}
% \elookup{(\theta(\env),\theta(\rho_1) \leadsto x)}{\head{\theta(\type)}} = \theta(\rho_2) \leadsto y
% \end{equation*}
% Finally, as $(\theta(\env),\theta(\rho_1) \leadsto x) = \theta(\env, \rho_1
% \leadsto x)$, we obtain the desired result:
% \begin{equation*}
% \elookup{\theta(\env,\rho_1 \leadsto x)}{\head{\theta(\type)}} = \theta(\rho_2) \leadsto y
% \end{equation*}
% 
% \end{description}
% \end{proof}
